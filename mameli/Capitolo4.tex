\chapter{Capitolo 4. Approssimazione di funzioni}
\section{Esercizi del libro}
\subsection{Esercizio 1}
Sia $f(x) = 4x^{2} - 12x +1$. Determinare $p(x) \in \Pi_{4} $ che interpola $f(x)$ sulle ascisse $x_{i} = i, i = 0, \ldots 4$.
\subsection{Esercizio 2} Dimostrare che il seguente algoritmo, \\
SCRIVERE ALGORITMO\\
valuta il polinomio (4.4) nel punto x, se il vettore $a$ contiene i coefficienti del polinomio $p(x)$ (Osservare che in Matlab i vettori hanno indice che parte da 1, invece che da 0).
\subsection{Esercizio 3}
Dimostrare il lemma 4.1
\subsection{Esercizio 4}
Dimostrare il lemma 4.2
\subsection{Esercizio 5}
Dimostrare il lemma 4.4
\subsection{Esercizio 6}
Costruire una function Matlab che implementi in modo efficiente l'Algoritmo 4.1
\subsection{Esercizio 7}
Dimostrare che il seguente algoritmo, che riceve in ingresso i vettori $x$ e $f$ prodotti dalla function dell'Esercizio 4.6, valuta il corrispondente polinomio interpolante di Newton in un punto $xx$
assegnato.\\
SCRIVERE ALGORITMO\\
Quale è il suo costo computazionale? Confrontarlo con quello dell'Algoritmo 4.1. Costruire, quindi, una corrispondente function Matlab che lo implementi efficientemente (complementare la possibilità che $xx$ sia un vettore)
\subsection{Esercizio 8}
Costruire una funciton Matlab che implementi efficientemente l'Algoritmo 4.2.