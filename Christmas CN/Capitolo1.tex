\documentclass[20pt,a4paper]{book}

\usepackage[italian]{babel}
\usepackage[T1]{fontenc}
\usepackage[latin1]{inputenc}

\begin{document}  %INIZIO DOCUMENTO
 
\textbf{\Large{Esercizio 1.01}}
Sia \begin{math}{x=\pi\simeq 3.1415=\widehat{x}}\end{math}. Calcolare il corrispondente errore
relativo \begin{math}{\varepsilon_{x}}\end{math}. Verificare che il numero di cifre decimali 
corrette nella rappresentazione approssimata di x mediante \begin{math}{\overline{x}}\end{math}
\'e all'incirca dato da: \begin{math}{-log10(|\varepsilon_{x}|)}\end{math}.

\textit{Soluzione}
Dalla formula \begin{math}{\varepsilon_{x}=\frac{\widehat{x}-x}{x}}\end{math}, ponendo \begin{math}
{x=\pi}\end{math} e \begin{math}{\widehat{x}=3.1415}\end{math} si ricava \begin{math}{\varepsilon_{x}=
-2.9493\cdot10^{-5}}\end{math}. Quindi il numero di cifre decimali rappresentate in modo corretto \'e \begin{math}{-log10(|\varepsilon_{x}|)\simeq 4.5303}\end{math}. 
Infatti 4 \'e proprio il numero di cifre decimali dell'approssimazione di \begin{math}{\pi}\end{math} rappresentate correttamente.

 
\vspace{10mm}

\textbf{\Large{Esercizio 1.08}}
Quante cifre binarie sono utilizzate per rappresentare, mediante 
arrotondamento, la mantissa di un numero, sapendo che la precisione
di macchina \'e \begin{math}\simeq  4.66\cdot10 ^{-10}\end{math}?

\textit{Soluzione} 

Per il Teorema 1.4, vale la relazione \begin{math} u=\frac{1}{2}b^{1-m} \end{math}
in caso di arrotondamento (dove u indica la precisione di macchina e b la base). 
Dato che b=2 e u=\begin{math}4.66\cdot10 ^{-10}\end{math}, si ricava che 
m=\begin{math}log_2 4.66\cdot10 ^{2}\simeq\end{math} 31.


\vspace{10mm}

\textbf{\Large{Esercizio 1.09}}
Dimostrare che, detta u la precisione di macchina utilizzata,\begin{math}-log_2 u\end{math}
fornisce, approssimativamente, il numero di cifre decimali correttamente rappresentate dalla
mantissa.

\textit{Soluzione} 

\\Arrotondamento:
\begin{math} u=\frac{1}{2}b^{1-m} \end{math} si eseguono i seguenti passaggi\\
\begin{math} 2u=b^{1-m}\end{math}\\
\begin{math} 1-m= log_b 2u\end{math}\\
\begin{math}m=1-log_b 2u\end{math}\\
\begin{math}m=1-log_b u-log_b 2\end{math}\\
e si pone b=10 ottenendo
\begin{math}m=1-log_{10} u-log_{10} 2\end{math}.\\
Dato che  \begin{math}1-log_{10} 2\simeq 0.7\end{math},  \'e possibile approssimare m cos\'i 
\begin{math}m=-log_2 u\end{math}

\vspace{2mm}

\\Troncamento:
\begin{math} u=\ b^{1-m} \end{math} si eseguono i seguenti passaggi\\
\begin{math}-log_{10}b^{1-m}=r\end{math}\\
\begin{math} (m-1)log_{10}b=r\end{math}\\
e si pone b=10 ottendendo
\begin{math} m=1-log_{10}u\end{math}\\


\vspace{10mm}

\textbf{\Large{Esercizio 1.15}}
Eseguire l'analisi dell'errore (relativo) dei due seguenti algoritmi per calcolare la somma
di tre numeri

1. \begin{math}{( x \oplus y)\oplus z}\end{math}

2. \begin{math}{ x \oplus (y\oplus z)}\end{math}

\textit{Soluzione} 

1. Dalla relazione \begin{math}{r\varepsilon_{r}=(x\varepsilon_{x}+y\varepsilon_{y})+z\varepsilon_{z}}\end{math}
si ricava \\   
\begin{math}{\varepsilon_{r}=\frac{(x\varepsilon_{x}+y\varepsilon_{y})}{(x+y)+z)}}\end{math}

Se \begin{math}{\varepsilon_{max}}\end{math} rappresenta il massimo tra \begin{math}{\varepsilon_{x},
\varepsilon_{y},\varepsilon_{z}}\end{math} , si ottiene

\begin{math}{\varepsilon_{max}=\frac{|(x+y)|+|z|}{|(x+y)+z)|}}\cdot\varepsilon_{max}\end{math}

2. Si procede in modo analogo, ottenendo per\'o

\begin{math}{\varepsilon_{max}=\frac{|x|+|(y+z)|}{|x+(y+z)|}}\cdot\varepsilon_{max}\end{math}

\vspace{10mm}

\textbf{\Large{Esercizio 1.16}}
Dimostrare che il numero di condizionamento del problema del calcolo di \begin{math}{y=\sqrt{x}}\end{math} \'e \begin{math}{k=\frac{1}{2}}\end{math}.

\textit{Soluzione} 
In questo caso, poich\'e \begin{math}{f(x)=\sqrt{x}}\end{math}, sappiamo che \begin{math}{f'(x)=\frac{1}{2\sqrt{x}}}\end{math}.
Sfruttando la seguente relazione \begin{math}{|\varepsilon_{y}|\simeq|f'(x)\cdot\frac{x}{y}|\cdot|\varepsilon_{x}|\equiv k\cdot|\varepsilon_{x}|}\end{math}, si ricava  \begin{math}{k=|f'(x)\cdot\frac{x}{y}|}\end{math}
Sostituendo poi \begin{math}{y=\sqrt{x}}\end{math} e \begin{math}{f'(x)=\frac{1}{2\sqrt{x}}}\end{math}, si ottiene \begin{math}{k=|\frac{x}{2\sqrt{x}\sqrt{x}}|=\frac{1}{2}}\end{math}.

\vspace{10mm}

\textbf{\Large{Esercizio 1.17}}
\\(Cancellazione Numerica) \\Si supponga di dover calcolare l'espressione \begin{math}{y=0.12345678-0.12341234\equiv0.00004444}\end{math} utilizzando una rappresentazione decimale con arrotondamento alla quarta cifra significativa. Comparare il risultato esatto con quello ottenuto in aritmetica finita, e determinare la perdita di cifre significative derivante dall'operazione effettuata. Verificare che questo risultato e' in accordo con l'analisi di condizionamento.

\textit{Soluzione} 
I due addendi sono di segno discorde e sono vicini in valore assoluto: il problema \'e malcondizionato e siamo in presenza della cancellazione numerica. 
Ponendo \begin{math}{x_{1}=0.12345678}\end{math} e \begin{math}{x_{2}=0.12345678}\end{math}, il risultato di \begin{math}{y=x_{1}-x_{2}}\end{math} in aritmetica esatta vale 0.00004444 \begin{math}{(\simeq 4\cdot 10^{-5})}\end{math}.
In aritmetica finita, arrotondando alla quarta cifra significativa, abbiamo \begin{math}{\widehat{x_{1}}=0.1235}\end{math}, \begin{math}{\widehat{x_{2}}=0.1234}\end{math} e quindi \begin{math}{\widehat{y}=\widehat{x_{1}}-\widehat{x_{2}}=0.001(\simeq10\cdot10^{5})}\end{math}. 
Confrontando i due risultati, si nota un'evidente perdita di cifre significative.
Nel caso approssimato, il problema risulta malcondizionato.
Abbiamo \begin{math} k= \frac{|x_{1}| + |x_{2}|}{|x_{1}+x_{2}|} = \frac{0.12345678+0.12341234}{0.00004444} \simeq \frac{0.3}{4*10^{-5}} \simeq 10^{4}\end{math}
Avendo inoltre \begin{math} u=\frac{1}{2} 10^{-4} \simeq 10^{-4}\end{math} si ha il caso degenere dove  \begin{math} k\simequ_{-1} \end{math}.



\end{document} 