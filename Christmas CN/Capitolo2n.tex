\documentclass[20pt,a4paper]{book}
\usepackage{amsmath}

\usepackage[italian]{babel}
\usepackage[T1]{fontenc}
\usepackage[latin1]{inputenc}

\begin{document} 
\textbf{\Large{Esercizio 2.1 }} Definire una procedura iterativa basata sul metodo di Newton
per determinare \begin{math}\sqrt{\alpha}\end{math}, per un assegnato \begin{math}\alpha \end{math} > 0. Costruire una tabella delle
approssimazioni relativa al caso \begin{math}\alpha = x_{0} \end{math} = 2. Comparare, infine, con la tabella:

\vspace{3mm}

\begin{tabular}{|l|l|l|l|}
\hline
n & xn \\
\hline
0 & 2 \\
\hline
1 & 1.5 \\
\hline
2 & 1.416666666666...\\
\hline
3 & 1.414215686274...\\
\hline
4 & 1.414213562374...\\
\hline
\end{tabular}

\vspace{3mm}
che rappresenta le approssimazioni di \begin{math} \sqrt{2} \end{math} fornite da \begin{math} x_{n+1}=\frac{1}{2}(x_{n}+\frac{2}{x_{n}})\end{math}, n=0,1,2..., x_{0}=2.


\textit{Soluzione}
Per calcolare \sqrt{\alpha} si considera la funzione f(x)=x^{2}-\alpha dato che si annulla per x=\sqrt{\alpha} e x=-\sqrt{\alpha}. Sappiamo che la derivata prima è f'(x)= 2x.

\\Utilizzando questa funzione diventa, l'iterazione del metodo di Newton diventa:

\begin{math}
x_{i+1}=x_{i}-\frac{f(x_{i})}{f'(x_{i})}=x_{i}-\frac{x_{i}^{2}-\alpha}{2x_{i}}=x_{i}-\frac{x_{i}(x_{i}-\frac{\alpha}{x_{i}})}{2x_{i}}=\frac{2x_{i}-x_{i}+\frac{\alpha}{x_{i}}}{2}=\frac{1}{2}(x_{i}+\frac{\alpha}{x_{i}})\end{math} per \begin{math}i=0,1,2..\end{math}

L'esecuzione di tale metodo, con \begin{math} \alpha= 2= x_{0}=2 \end{math} produce le seguenti approssimazioni:

\vspace{2mm}

\begin{tabular}{|l|l|l|l|}
\hline
0 & 2.00000000000000\\
\hline
1 & 1.50000000000000\\
\hline
2 & 1.41666666666667\\
\hline
3 & 1.41421568627451\\
\hline
4 & 1.41421356237469\\
\hline
5 & 1.41421356237309\\
\hline
\end{tabular}

\vspace{2mm}

\\L'algoritmo termina alla sesta iterazione, con un'approssimazione di \begin{math}\sqrt{2}\end{math} dell'ordine di 10^{-16}.



\vspace{10mm}


\textbf{\Large{Esercizio 2.2 }} Generalizzare il risultato del precedente esercizio, derivando una
procedura iterativa basata sul metodo di Newton per determinare \begin{math} ^{n}\sqrt{\alpha} \end{math} per un assegnato \begin{math} \alpha \end{math}>0.

\textit{Soluzione}
In questo caso scegliamo come funzione \begin{math} f(x)=x^{n}-\alpha \end{math}, la quale si annulla per x=\begin{math}^{n}\sqrt{\alpha}\end{math} e la cui derivata vale \begin{math} f'(x)=nx^{n-1}\end{math}.
\\Considerando la procedure iterativa del metodo di Newton, ricaviamo: \\


\begin{math} x_{i+1}=x_{i}-\frac{f(x_{i})}{f'(x_{i})}=x_{i}-\frac{x^{n}-\alpha}{nx^{n-1}}=
\frac{nx_{i}^{n}-x_{i}^{n}+\alpha}{nx_{i}^{n-1}}=(\frac{1}{n})((n-1)x_{i}+\frac{\alpha}{x_{i}^{n-1}})\end{math}

\vspace{10mm}



\textbf{\Large{Esercizio 2.3 }} In analogia con quanto visto nell' esercizio 2.1, definire una procedura iterativa basata sul metodo delle secanti per determinare \begin{math} \sqrt{\alpha} \end{math}. Confrontare con l'esercizio 1.4.

\textit{Soluzione}
Si considera la funzione  \begin{math}f(x)=x^{2}-\alpha\end{math}, che si annulla in \begin{math} x=\sqrt{\alpha}\end{math} e \begin{math} x=-\sqrt{\alpha}\end{math}.

Considerando il metodo iterativo delle secanti, ricaviamo: \\

\begin{math} x_{i+1}=\frac{f(x_{i})x_{i-1}-f(x_{i-1})x_{i}}{f(x_{i})-f(x_{i-1})}=
\frac{(x_{i}^{2}-\alpha)x_{i-1}-(x_{i-1}^{2}-\alpha)x_{i}}{x_{i}^{2}-\alpha-x_{i-1}^{2}+\alpha}=
 \frac{x_{i}^{2}x_{i-1}-\alpha x_{i-1}-x_{i-1}^{2}x_{i}+\alpha x_{i}}{x_{i}^{2}-x_{i-1}^{2}}= \\
\frac{x_{i}x_{i-1}(x_{i}-x_{i-1})+\alpha(x_{i}-x_{i-1})}{(x_{i}-x_{i-1})(x_{i}+x_{i-1})}=
\frac{(x_{i}-x_{i-1})(x_{i}x_{i-1}+\alpha)}{(x_{i}-x_{i-1})(x_{i}+x_{i-1})}=\frac{x_{i}x_{i-1}+\alpha}{x_{i}+x_{i-1}} \end{math}  con i=0,1,2.. 

\vspace{2mm}

Con l'esecuzione del metodo, scegliendo \begin{math} \alpha= x_{0}=2 \end{math} si ottengono tali approssimazioni con una tolleranza pari alla precisione di macchina: \\


\vspace{2mm}

\begin{tabular}{|l|l|l|l|}
\hline
i & x_{i}\\
\hline
0 & 2 \\
\hline
1 & 1.5 \\
\hline
2 & 1.428571428571429.. \\
\hline
3 & 1.414634146341463.. \\
\hline
4 & 1.414215686274510.. \\
\hline
5 & 1.414213562688869.. \\
\hline
6 & 1.414213562373095.. \\
\hline
7 & 1.414213562373095.. \\
\hline
\end{tabular}


\vspace{2mm}

Osservando i valori, si nota che la convergenza risulta piu' lenta rispetto al metodo di Newton dell’ esercizio 2.1. Dalla sesta iterazione in poi, l'algoritmo termina fornendo un'approssimazione con un errore dell'ordine di 10^{-16}.


\vspace{10mm}

\textbf{\Large{Esercizio 2.4 }} Discutere la convergenza del metodo di Newton, applicato
per determinare le radici dell'equazione \begin{math}x^{3}-5x \end{math} , in funzione della scelta del punto iniziale \begin{math}x_{0}\end{math}.

\textit{Soluzione}
La funzione \begin{math}f(x)= x^{3}-5x\end{math} ha tre radici:\\
\begin{math} x_{0}=0, x_{1}=\sqrt{5}, x_{2}=-\sqrt{5}\end{math}.
Se scegliamo come punto di innesto 1 o -1 il metodo non converge e applichiamo il metodo di Newton, possiamo notare che questo non converge.

\\ \begin{math} x_{0}=-1\end{math}\\
\vspace{2mm}

\begin{tabular}{|l|l|l|l|}
\hline
i & x_{i}\\
\hline
1 & -1 \\
\hline
2 & 1 \\
\hline
3 & -1 \\
\hline
4 & 1 \\
\hline
\end{tabular}


\vspace{2mm}


\begin{math} x_{0}=1\end{math}\\

\vspace{2mm}

\begin{tabular}{|l|l|l|l|}
\hline
i & x_{i}\\
\hline
1 & 1 \\
\hline
2 & -1 \\
\hline
3 & 1 \\
\hline
4 & -1 \\
\hline
\end{tabular}

 
\vspace{2mm} 
 
Se invece il punto di innesto e' ad esempio 2, si ha la convergenza al valore 2.23...

\begin{math} x_{0}=2\end{math} \\

\vspace{2mm} 

\begin{tabular}{|l|l|l|l|}
\hline
i & x_{i}\\
\hline
1 & 2.237639989074023 \\
\hline
2 & 2.236069632529971 \\
\hline
3 & 2.236067977501627 \\
\hline
4 & 2.236067977499790 \\
\hline
\end{tabular}


\vspace{10mm}

\textbf{\Large{Esercizio 2.6}} E' possibile, nel caso delle funzioni del precedente esercizio, utilizzare il metodo di bisezione per determinare lo zero?

\textit{Soluzione}
Le funzioni del precedente esercizio sono sempre positive e si annullano in x=1. Visto che non si verifica la condizione f(a)f(b)< 0 per qualche a,b (Teorema degli Zeri), non e' possibile utilizzare il metodo di bisezione.

\vspace{10mm}



\textbf{\Large{Esercizio 2.7}} Costruire una tabella in cui si comparano, a partire dallo stesso punto iniziale x0=0, e per valori decrescenti della tolleranza \textbf{tolx}, il numero di iterazioni richieste per la convergenza dei metodi di Newton, corde e secanti, utilizzati per determinare lo zero della funzione \begin{math}f(x)=x-cos(x) \end{math}

\textit{Soluzione}

\begin{tabular}{|l|l|l|l|}
\hline
tolx & Newton & Corde & Secanti\\
\hline
\begin{math}10^{-1}\end{math} & x = 0:73911 , i = 2 & x = 0:70137 , i = 5 & x = 0:7363 , i = 2\\
\hline
\begin{math}10^{-2}\end{math} & x = 0:73909 , i = 3 & x = 0:7356 , i = 11 & x = 0:73912 , i = 3\\
\hline
\begin{math}10^{-3}\end{math} & x = 0:73909 , i = 3 & x = 0:73876 , i = 17 & x = 0:73909 , i = 4\\
\hline
\begin{math}10^{-4}\end{math} & x = 0:73909 , i = 3 & x = 0:73905 , i = 23 & x = 0:73909 , i = 4\\
\hline
\begin{math}10^{-5}\end{math} & x = 0:73909 , i = 4 & x = 0:73908 , i = 29 & x = 0:73909 , i = 5\\
\hline
\begin{math}10^{-6}\end{math} & x = 0:73909 , i = 4 & x = 0:73909 , i = 34 & x = 0:73909 , i = 5\\
\hline
\begin{math}10^{-7}\end{math} & x = 0:73909 , i = 4 & x = 0:73909 , i = 40 & x = 0:73909 , i = 5\\
\hline
\begin{math}10^{-8}\end{math} & x = 0:73909 , i = 4 & x = 0:73909 , i = 46 & x = 0:73909 , i = 6\\
\hline
\begin{math}10^{-9}\end{math} & x = 0:73909 , i = 4 & x = 0:73909 , i = 52 & x = 0:73909 , i = 6\\
\hline
\begin{math}10^{-10}\end{math} & x = 0:73909 , i = 5 & x = 0:73909 , i = 58 & x = 0:73909 , i = 6\\
\hline
\begin{math}10^{-11}\end{math} & x = 0:73909 , i = 5 & x = 0:73909 , i = 64 & x = 0:73909 , i = 6\\
\hline
\begin{math}10^{-12}\end{math} & x = 0:73909 , i = 5 & x = 0:73909 , i = 69 & x = 0:73909 , i = 6\\
\hline
\begin{math}10^{-13}\end{math} & x = 0:73909 , i = 5 & x = 0:73909 , i = 75 & x = 0:73909 , i = 7\\
\hline
\begin{math}10^{-14}\end{math} & x = 0:73909 , i = 5 & x = 0:73909 , i = 81 & x = 0:73909 , i = 7\\
\hline
\begin{math}10^{-15}\end{math} & x = 0:73909 , i = 5 & x = 0:73909 , i = 87 & x = 0:73909 , i = 7\\
\hline
\end{tabular}

\vspace{2mm}

I metodi di Newton e delle secanti convergono molto rapidamente alla soluzione, rispetto al metodo delle corde. Se pero' si osserva il tempo di esecuzione, si deduce che i metodi delle corde e delle secanti hanno un tempo di esecuzione medio per step inferiore.

\vspace{10mm}

\textbf{\Large{Esercizio 2.8}} Completare i confronti del precedente esercizio inserendo quelli con il metodo di bisezione, con intervallo di confidenza [0,1]? 

\textit{Soluzione}
La seguente tabella contiene i risultati dell’esecuzione del metodo di bisezione con intervallo
di confidenza iniziale [0, 1]:

\begin{math} f(x)=x-cos(x)\end{math}\\

\begin{tabular}{|l|l|l|l|}
\hline
tolx & approssimazione & iterazioni\\
\hline
\begin{math}10^{-1}\end{math} & x = 0.75 & i = 1\\
\hline
\begin{math}10^{-2}\end{math} & x = 0.7344 & i = 5\\
\hline
\begin{math}10^{-3}\end{math} & x = 0.7383 & i = 8\\
\hline
\begin{math}10^{-4}\end{math} & x = 0.7390 & i = 11\\
\hline
\begin{math}10^{-5}\end{math} & x = 0.7391 & i = 15\\
\hline
\begin{math}10^{-6}\end{math} & x = 0.7391 & i = 18\\
\hline
\begin{math}10^{-7}\end{math} & x = 0.7391 & i = 19\\
\hline
\begin{math}10^{-8}\end{math} & x = 0.7391 & i = 23\\
\hline
\begin{math}10^{-9}\end{math} & x = 0.7391 & i = 27\\
\hline
\begin{math}10^{-10}\end{math} & x = 0.7391 & i = 29\\
\hline
\begin{math}10^{-11}\end{math} & x = 0.7391 & i = 34\\
\hline
\begin{math}10^{-12}\end{math} & x = 0.7391 & i = 38\\
\hline
\begin{math}10^{-13}\end{math} & x = 0.7391 & i = 40\\
\hline
\begin{math}10^{-14}\end{math} & x = 0.7391 & i = 43\\
\hline
\begin{math}10^{-15}\end{math} & x = 0.7391 & i = 48\\
\hline
\end{tabular}

\vspace{2mm}

In questo caso il metodo di bisezione converge piu' lentamente dei metodi di Newton e delle secanti, ma
anche piu' rapidamente rispetto al metodo delle corde.\\
Nonostante questo, il metodo di bisezione ha un minor costo computazionale rispetto ai metodi delle secanti e di Newton.


\vspace{10mm}

\textbf{\Large{Esercizio 2.9}} Quali controlli introdurreste, negli Algoritmi 2.4-2.6, al fine di rendere piu' "robuste" le corrispondenti iterazioni?

\textit{Soluzione}

Nel metodo di Newton si dovrebbe inserire un controllo sulla derivata f'(x): quest'ultima infatti deve essere diversa da 0, perche' altrimenti avremmo una divisione con denominatore nullo e si verrebbe a verificare un errore.

if ( f1x==0) 
error (La derivata deve essere diversa da 0);
end

\vspace{3mm}

Per quanto riguarda il metodo delle secanti, vogliamo evitare anche in questo caso una divisione con denominatore nullo. Quindi imponiamo che fx-fx0 sia diverso da 0.

if (fx-fx0==0) 
error (Denominatore uguale a 0);
end

\vspace{3mm}

Nell'algoritmo di Aitken, deve essere imposto che x-2*x1+x0 sia diverso da zero.

if (x-2*x1+x0==0) 
error (Denominatore uguale a 0);
end

\vspace{3mm}

Infine, si possono aggiungere dei controlli sul numero massimo di iterazioni eseguite e sull' errore commesso (rispetto ad un tolx fissato).


\vspace{10mm}

\end{document} 




 
