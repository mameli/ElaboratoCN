
\documentclass[20pt,a4paper]{book}
\usepackage{amsmath}

\usepackage[italian]{babel}
\usepackage[T1]{fontenc}
\usepackage[latin1]{inputenc}

\begin{document}  


\textbf{\Large{Esercizio 5.1}}  
Calcolare il numero di condizionamento dell'integrale \begin{math}\int_{0}^{e^{21}} sin \sqrt{x} dx\end{math}. Questo problema e' ben condizionato o mal condizionato? 
 
\textit{Soluzione}
Per quanto riguarda il fattore di condizionamento k, sappiamo che vale la relazione 0=b-a. In questo caso abbiamo \begin{math}b=e^{21}\end{math} e a=0, quindi k=\begin{math}e^{21}\end{math}.
\\Il problema risulta malcondizionato.

\vspace{10mm}

\textbf{\Large{Esercizio 5.2}}  
Derivare dalla formula di Newton-Cotes, i coecienti della formula dei trapezi e della formula di Simpson.
 
\textit{Soluzione}
\\1)Formula dei trapezi (n=1):

\begin{math} c_{0,1}=\int_{0}^{1}\frac{t-0}{1-0}dt=\int_{0}^{1}t dt=\frac{t^{2}}{2}\mid_{0}^{1}=\frac{1}{2} 
\\c_{1,1}=1-c_{0,1}=1-\frac{1}{2}=\frac{1}{2} \end{math}

\vspace{3mm}

1)Formula di Simpson (n=2):

\begin{math} c_{0,2}=\int_{0}^{2}\frac{(t-1)(t-2)}{(0-1)(0-2)}dt=\frac{1}{2}\int_{0}^{2}t^{2}-3t+2 dt=\frac{1}{2}(\frac{t^{3}}{3}-\frac{3}{2}t^{2}+2t)\mid_{0}^{2}=\frac{1}{3}\end{math}
\\Eseguendo lo stesso procedimento si ottiene: \begin{math}
\\c_{1,2}=\frac{4}{3}
\\c_{2,2}=\frac{1}{3}\end{math}


\vspace{10mm}

\textbf{\Large{Esercizio 5.3}}  
Verifcare la stima dell'errore per il metodo dei trapezi e di Simpson.
 
\textit{Soluzione}
\\1)Metodo dei trapezi (n=1, k=1):

\begin{math}\nu_{1}=\int_{0}^{1}\Pi_{j=0}^{1}(t-j) dt=(\frac{t^{3}}{3}-\frac{t^{2}}{2})\mid_{0}^{1}=-\frac{1}{6} \end{math}

\\da cui:

\\\begin{math}E_{1}(f)=\nu_{1}\frac{f^{(2)}(\xi)}{2!}(\frac{b-a}{1})^{3}=-\frac{f^{(2)}(\xi)(b-a)^{3}}{12}\end{math}


\vspace{3mm}

1)Metodo di Simpson (n=2, k=2):


\begin{math}\nu_{2}=\int_{0}^{2}\Pi_{j=0}^{2}(t-j) dt=(\frac{t^{5}}{5}-\frac{3t^{4}}{4}+\frac{2t^{3}}{3})\mid_{0}^{2}=-\frac{4}{15} \end{math}

\\da cui:

\\\begin{math}E_{2}(f)=\nu_{2}\frac{f^{(4)}(\xi)}{4!}(\frac{b-a}{2})^{5}=-\frac{1}{90}f^{(4)}(\xi)\frac{(b-a)^{5}}{2^{5}}\end{math} 


\vspace{10mm}

\textbf{\Large{Esercizio 5.8}}
Come e' classificabile, dal punto di vista del condizionamento il problema \begin{math}\int_{\frac{1}{2}}^{100}-2x^{-3}cos(x^{-2})dx\equiv sin(10^{-4})-sin(4)\end{math}?

 
\textit{Soluzione}
Dato che b=100 e a=\begin{math}\frac{1}{2}\end{math} abbiamo k=100-\begin{math}\frac{1}{2}=99.5\gg 1\end{math} quindi il problema e' malcondizionato.


\vspace{10mm}





\end{document} 

