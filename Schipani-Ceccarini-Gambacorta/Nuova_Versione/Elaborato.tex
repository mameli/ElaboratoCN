\documentclass[a4paper, 10pt]{book}
\usepackage{amsmath}
\usepackage{amsfonts}
\usepackage{amssymb}
\usepackage[italian]{babel}
\usepackage[utf8x]{inputenc}
\usepackage{mathtools}
\usepackage{epigraph}
\usepackage{listings}
\usepackage{amsfonts, mathtools, placeins, floatflt, xfrac, multirow, indentfirst, fancyhdr,
            amssymb, latexsym, amsthm, eucal, float, tabularx, mdframed, bigdelim, blkarray,
            relsize, lettrine, xcolor, tikz, caption, geometry, perpage}
\usepackage[Sonny]{fncychap}
\usepackage{ornament}






\lstloadlanguages{Matlab, Fortran}
\lstset{
  language = Matlab,
  frameround = fttt,
  frame = single,
  numbers = left,
  stepnumber = 2,
  breaklines = true,
  numberstyle=\tiny\color{gray},
}


\title{ELABORATO DI CALCOLO NUMERICO}
\author{Federico Schipani \\ Tommaso Ceccarini \\ Giuliano Gambacorta \\}

\begin{document}
\maketitle
\tableofcontents

\cleardoublepage

\thispagestyle{empty}
\vspace{\stretch{1}}
\begin{flushright}
\epigraph{L'errore assoluto da un informazione relativa, l'errore relativo da un informazione assoluta!}{Sestini, Brugnano, Magherini}
\end{flushright}
\vspace{\stretch{2}}
\cleardoublepage
\documentclass[20pt,a4paper]{book}

\usepackage[italian]{babel}
\usepackage[T1]{fontenc}
\usepackage[latin1]{inputenc}

\begin{document}  %INIZIO DOCUMENTO
 
\textbf{\Large{Esercizio 1.01}}
Sia \begin{math}{x=\pi\simeq 3.1415=\widehat{x}}\end{math}. Calcolare il corrispondente errore
relativo \begin{math}{\varepsilon_{x}}\end{math}. Verificare che il numero di cifre decimali 
corrette nella rappresentazione approssimata di x mediante \begin{math}{\overline{x}}\end{math}
\'e all'incirca dato da: \begin{math}{-log10(|\varepsilon_{x}|)}\end{math}.

\textit{Soluzione}
Dalla formula \begin{math}{\varepsilon_{x}=\frac{\widehat{x}-x}{x}}\end{math}, ponendo \begin{math}
{x=\pi}\end{math} e \begin{math}{\widehat{x}=3.1415}\end{math} si ricava \begin{math}{\varepsilon_{x}=
-2.9493\cdot10^{-5}}\end{math}. Quindi il numero di cifre decimali rappresentate in modo corretto \'e \begin{math}{-log10(|\varepsilon_{x}|)\simeq 4.5303}\end{math}. 
Infatti 4 \'e proprio il numero di cifre decimali dell'approssimazione di \begin{math}{\pi}\end{math} rappresentate correttamente.

 
\vspace{10mm}

\textbf{\Large{Esercizio 1.08}}
Quante cifre binarie sono utilizzate per rappresentare, mediante 
arrotondamento, la mantissa di un numero, sapendo che la precisione
di macchina \'e \begin{math}\simeq  4.66\cdot10 ^{-10}\end{math}?

\textit{Soluzione} 

Per il Teorema 1.4, vale la relazione \begin{math} u=\frac{1}{2}b^{1-m} \end{math}
in caso di arrotondamento (dove u indica la precisione di macchina e b la base). 
Dato che b=2 e u=\begin{math}4.66\cdot10 ^{-10}\end{math}, si ricava che 
m=\begin{math}log_2 4.66\cdot10 ^{2}\simeq\end{math} 31.


\vspace{10mm}

\textbf{\Large{Esercizio 1.09}}
Dimostrare che, detta u la precisione di macchina utilizzata,\begin{math}-log_2 u\end{math}
fornisce, approssimativamente, il numero di cifre decimali correttamente rappresentate dalla
mantissa.

\textit{Soluzione} 

\\Arrotondamento:
\begin{math} u=\frac{1}{2}b^{1-m} \end{math} si eseguono i seguenti passaggi\\
\begin{math} 2u=b^{1-m}\end{math}\\
\begin{math} 1-m= log_b 2u\end{math}\\
\begin{math}m=1-log_b 2u\end{math}\\
\begin{math}m=1-log_b u-log_b 2\end{math}\\
e si pone b=10 ottenendo
\begin{math}m=1-log_{10} u-log_{10} 2\end{math}.\\
Dato che  \begin{math}1-log_{10} 2\simeq 0.7\end{math},  \'e possibile approssimare m cos\'i 
\begin{math}m=-log_2 u\end{math}

\vspace{2mm}

\\Troncamento:
\begin{math} u=\ b^{1-m} \end{math} si eseguono i seguenti passaggi\\
\begin{math}-log_{10}b^{1-m}=r\end{math}\\
\begin{math} (m-1)log_{10}b=r\end{math}\\
e si pone b=10 ottendendo
\begin{math} m=1-log_{10}u\end{math}\\


\vspace{10mm}

\textbf{\Large{Esercizio 1.15}}
Eseguire l'analisi dell'errore (relativo) dei due seguenti algoritmi per calcolare la somma
di tre numeri

1. \begin{math}{( x \oplus y)\oplus z}\end{math}

2. \begin{math}{ x \oplus (y\oplus z)}\end{math}

\textit{Soluzione} 

1. Dalla relazione \begin{math}{r\varepsilon_{r}=(x\varepsilon_{x}+y\varepsilon_{y})+z\varepsilon_{z}}\end{math}
si ricava \\   
\begin{math}{\varepsilon_{r}=\frac{(x\varepsilon_{x}+y\varepsilon_{y})}{(x+y)+z)}}\end{math}

Se \begin{math}{\varepsilon_{max}}\end{math} rappresenta il massimo tra \begin{math}{\varepsilon_{x},
\varepsilon_{y},\varepsilon_{z}}\end{math} , si ottiene

\begin{math}{\varepsilon_{max}=\frac{|(x+y)|+|z|}{|(x+y)+z)|}}\cdot\varepsilon_{max}\end{math}

2. Si procede in modo analogo, ottenendo per\'o

\begin{math}{\varepsilon_{max}=\frac{|x|+|(y+z)|}{|x+(y+z)|}}\cdot\varepsilon_{max}\end{math}

\vspace{10mm}

\textbf{\Large{Esercizio 1.16}}
Dimostrare che il numero di condizionamento del problema del calcolo di \begin{math}{y=\sqrt{x}}\end{math} \'e \begin{math}{k=\frac{1}{2}}\end{math}.

\textit{Soluzione} 
In questo caso, poich\'e \begin{math}{f(x)=\sqrt{x}}\end{math}, sappiamo che \begin{math}{f'(x)=\frac{1}{2\sqrt{x}}}\end{math}.
Sfruttando la seguente relazione \begin{math}{|\varepsilon_{y}|\simeq|f'(x)\cdot\frac{x}{y}|\cdot|\varepsilon_{x}|\equiv k\cdot|\varepsilon_{x}|}\end{math}, si ricava  \begin{math}{k=|f'(x)\cdot\frac{x}{y}|}\end{math}
Sostituendo poi \begin{math}{y=\sqrt{x}}\end{math} e \begin{math}{f'(x)=\frac{1}{2\sqrt{x}}}\end{math}, si ottiene \begin{math}{k=|\frac{x}{2\sqrt{x}\sqrt{x}}|=\frac{1}{2}}\end{math}.

\vspace{10mm}

\textbf{\Large{Esercizio 1.17}}
\\(Cancellazione Numerica) \\Si supponga di dover calcolare l'espressione \begin{math}{y=0.12345678-0.12341234\equiv0.00004444}\end{math} utilizzando una rappresentazione decimale con arrotondamento alla quarta cifra significativa. Comparare il risultato esatto con quello ottenuto in aritmetica finita, e determinare la perdita di cifre significative derivante dall'operazione effettuata. Verificare che questo risultato e' in accordo con l'analisi di condizionamento.

\textit{Soluzione} 
I due addendi sono di segno discorde e sono vicini in valore assoluto: il problema \'e malcondizionato e siamo in presenza della cancellazione numerica. 
Ponendo \begin{math}{x_{1}=0.12345678}\end{math} e \begin{math}{x_{2}=0.12345678}\end{math}, il risultato di \begin{math}{y=x_{1}-x_{2}}\end{math} in aritmetica esatta vale 0.00004444 \begin{math}{(\simeq 4\cdot 10^{-5})}\end{math}.
In aritmetica finita, arrotondando alla quarta cifra significativa, abbiamo \begin{math}{\widehat{x_{1}}=0.1235}\end{math}, \begin{math}{\widehat{x_{2}}=0.1234}\end{math} e quindi \begin{math}{\widehat{y}=\widehat{x_{1}}-\widehat{x_{2}}=0.001(\simeq10\cdot10^{5})}\end{math}. 
Confrontando i due risultati, si nota un'evidente perdita di cifre significative.
Nel caso approssimato, il problema risulta malcondizionato.
Abbiamo \begin{math} k= \frac{|x_{1}| + |x_{2}|}{|x_{1}+x_{2}|} = \frac{0.12345678+0.12341234}{0.00004444} \simeq \frac{0.3}{4*10^{-5}} \simeq 10^{4}\end{math}
Avendo inoltre \begin{math} u=\frac{1}{2} 10^{-4} \simeq 10^{-4}\end{math} si ha il caso degenere dove  \begin{math} k\simequ_{-1} \end{math}.



\end{document} 
\documentclass[20pt,a4paper]{book}

\usepackage[italian]{babel}
\usepackage[T1]{fontenc}
\usepackage[latin1]{inputenc}

\begin{document}  %INIZIO DOCUMENTO
 
\textbf{\Large{Esercizio 1.01}}
Sia \begin{math}{x=\pi\simeq 3.1415=\widehat{x}}\end{math}. Calcolare il corrispondente errore
relativo \begin{math}{\varepsilon_{x}}\end{math}. Verificare che il numero di cifre decimali 
corrette nella rappresentazione approssimata di x mediante \begin{math}{\overline{x}}\end{math}
\'e all'incirca dato da: \begin{math}{-log10(|\varepsilon_{x}|)}\end{math}.

\textit{Soluzione}
Dalla formula \begin{math}{\varepsilon_{x}=\frac{\widehat{x}-x}{x}}\end{math}, ponendo \begin{math}
{x=\pi}\end{math} e \begin{math}{\widehat{x}=3.1415}\end{math} si ricava \begin{math}{\varepsilon_{x}=
-2.9493\cdot10^{-5}}\end{math}. Quindi il numero di cifre decimali rappresentate in modo corretto \'e \begin{math}{-log10(|\varepsilon_{x}|)\simeq 4.5303}\end{math}. 
Infatti 4 \'e proprio il numero di cifre decimali dell'approssimazione di \begin{math}{\pi}\end{math} rappresentate correttamente.

 
\vspace{10mm}

\textbf{\Large{Esercizio 1.08}}
Quante cifre binarie sono utilizzate per rappresentare, mediante 
arrotondamento, la mantissa di un numero, sapendo che la precisione
di macchina \'e \begin{math}\simeq  4.66\cdot10 ^{-10}\end{math}?

\textit{Soluzione} 

Per il Teorema 1.4, vale la relazione \begin{math} u=\frac{1}{2}b^{1-m} \end{math}
in caso di arrotondamento (dove u indica la precisione di macchina e b la base). 
Dato che b=2 e u=\begin{math}4.66\cdot10 ^{-10}\end{math}, si ricava che 
m=\begin{math}log_2 4.66\cdot10 ^{2}\simeq\end{math} 31.


\vspace{10mm}

\textbf{\Large{Esercizio 1.09}}
Dimostrare che, detta u la precisione di macchina utilizzata,\begin{math}-log_2 u\end{math}
fornisce, approssimativamente, il numero di cifre decimali correttamente rappresentate dalla
mantissa.

\textit{Soluzione} 

\\Arrotondamento:
\begin{math} u=\frac{1}{2}b^{1-m} \end{math} si eseguono i seguenti passaggi\\
\begin{math} 2u=b^{1-m}\end{math}\\
\begin{math} 1-m= log_b 2u\end{math}\\
\begin{math}m=1-log_b 2u\end{math}\\
\begin{math}m=1-log_b u-log_b 2\end{math}\\
e si pone b=10 ottenendo
\begin{math}m=1-log_{10} u-log_{10} 2\end{math}.\\
Dato che  \begin{math}1-log_{10} 2\simeq 0.7\end{math},  \'e possibile approssimare m cos\'i 
\begin{math}m=-log_2 u\end{math}

\vspace{2mm}

\\Troncamento:
\begin{math} u=\ b^{1-m} \end{math} si eseguono i seguenti passaggi\\
\begin{math}-log_{10}b^{1-m}=r\end{math}\\
\begin{math} (m-1)log_{10}b=r\end{math}\\
e si pone b=10 ottendendo
\begin{math} m=1-log_{10}u\end{math}\\


\vspace{10mm}

\textbf{\Large{Esercizio 1.15}}
Eseguire l'analisi dell'errore (relativo) dei due seguenti algoritmi per calcolare la somma
di tre numeri

1. \begin{math}{( x \oplus y)\oplus z}\end{math}

2. \begin{math}{ x \oplus (y\oplus z)}\end{math}

\textit{Soluzione} 

1. Dalla relazione \begin{math}{r\varepsilon_{r}=(x\varepsilon_{x}+y\varepsilon_{y})+z\varepsilon_{z}}\end{math}
si ricava \\   
\begin{math}{\varepsilon_{r}=\frac{(x\varepsilon_{x}+y\varepsilon_{y})}{(x+y)+z)}}\end{math}

Se \begin{math}{\varepsilon_{max}}\end{math} rappresenta il massimo tra \begin{math}{\varepsilon_{x},
\varepsilon_{y},\varepsilon_{z}}\end{math} , si ottiene

\begin{math}{\varepsilon_{max}=\frac{|(x+y)|+|z|}{|(x+y)+z)|}}\cdot\varepsilon_{max}\end{math}

2. Si procede in modo analogo, ottenendo per\'o

\begin{math}{\varepsilon_{max}=\frac{|x|+|(y+z)|}{|x+(y+z)|}}\cdot\varepsilon_{max}\end{math}

\vspace{10mm}

\textbf{\Large{Esercizio 1.16}}
Dimostrare che il numero di condizionamento del problema del calcolo di \begin{math}{y=\sqrt{x}}\end{math} \'e \begin{math}{k=\frac{1}{2}}\end{math}.

\textit{Soluzione} 
In questo caso, poich\'e \begin{math}{f(x)=\sqrt{x}}\end{math}, sappiamo che \begin{math}{f'(x)=\frac{1}{2\sqrt{x}}}\end{math}.
Sfruttando la seguente relazione \begin{math}{|\varepsilon_{y}|\simeq|f'(x)\cdot\frac{x}{y}|\cdot|\varepsilon_{x}|\equiv k\cdot|\varepsilon_{x}|}\end{math}, si ricava  \begin{math}{k=|f'(x)\cdot\frac{x}{y}|}\end{math}
Sostituendo poi \begin{math}{y=\sqrt{x}}\end{math} e \begin{math}{f'(x)=\frac{1}{2\sqrt{x}}}\end{math}, si ottiene \begin{math}{k=|\frac{x}{2\sqrt{x}\sqrt{x}}|=\frac{1}{2}}\end{math}.

\vspace{10mm}

\textbf{\Large{Esercizio 1.17}}
\\(Cancellazione Numerica) \\Si supponga di dover calcolare l'espressione \begin{math}{y=0.12345678-0.12341234\equiv0.00004444}\end{math} utilizzando una rappresentazione decimale con arrotondamento alla quarta cifra significativa. Comparare il risultato esatto con quello ottenuto in aritmetica finita, e determinare la perdita di cifre significative derivante dall'operazione effettuata. Verificare che questo risultato e' in accordo con l'analisi di condizionamento.

\textit{Soluzione} 
I due addendi sono di segno discorde e sono vicini in valore assoluto: il problema \'e malcondizionato e siamo in presenza della cancellazione numerica. 
Ponendo \begin{math}{x_{1}=0.12345678}\end{math} e \begin{math}{x_{2}=0.12345678}\end{math}, il risultato di \begin{math}{y=x_{1}-x_{2}}\end{math} in aritmetica esatta vale 0.00004444 \begin{math}{(\simeq 4\cdot 10^{-5})}\end{math}.
In aritmetica finita, arrotondando alla quarta cifra significativa, abbiamo \begin{math}{\widehat{x_{1}}=0.1235}\end{math}, \begin{math}{\widehat{x_{2}}=0.1234}\end{math} e quindi \begin{math}{\widehat{y}=\widehat{x_{1}}-\widehat{x_{2}}=0.001(\simeq10\cdot10^{5})}\end{math}. 
Confrontando i due risultati, si nota un'evidente perdita di cifre significative.
Nel caso approssimato, il problema risulta malcondizionato.
Abbiamo \begin{math} k= \frac{|x_{1}| + |x_{2}|}{|x_{1}+x_{2}|} = \frac{0.12345678+0.12341234}{0.00004444} \simeq \frac{0.3}{4*10^{-5}} \simeq 10^{4}\end{math}
Avendo inoltre \begin{math} u=\frac{1}{2} 10^{-4} \simeq 10^{-4}\end{math} si ha il caso degenere dove  \begin{math} k\simequ_{-1} \end{math}.



\end{document} 
\chapter{Radici di una equazione}
\section{Esercizi del libro}
\subsection{Esercizio 2.1}
Definire una procedura iterativa basata sul metodo di Newton per determinare $\sqrt{a}$, per un assegnato $a>0$. Costruire una tabella dell'approssimazioni relativa al caso $a=x_{0}=2$ (Comparare con la tabella 1.1)
\subsection{Esercizio 2.2}
Generalizzare il risultato del precedente esercizio, derivando una procedura iterativa basata sul metodo di Newton per determinare $\sqrt[n]{a}$ per un assegnato $a>0$
\subsection{Esercizio 2.3}
In analogia con quanto visto nell'Esercizio 2.1, definire una procedura iterativa basata sul metodo delle secanti per determinare $\sqrt{a}$. Confrontare con l'esercizio 1.4.
\subsection{Esercizio 2.4}
Discutere la convergenza del metodo di Newton, applicato per determinare le radici dell'equazione (2.11) in funzione della scelta del punto iniziale $x_{0}$
\subsection{Esercizio 2.5}
Comparare il metodo di Newton (2.9), il metodo di Newton modificato (2.16) ed il metodo di accellerazione di Aitken (2.17), per approsimare gli zeri delle funzioni:\\
\center $f_{1}(x) = (x-1)^{10}, \qquad f_{2}(x)=(x-1)^{10}e^{x} $
\flushleft per valori decrescenti della tolleranza $\mathtt{tolx}$. Utilizzare, in tutti i casi, il punto iniziale $x_{0}=10$.
\subsection{Esercizio 2.6}
È possibile, nel caso delle funzioni del precedente esercizio utilizzare il metodo di bisezione per determinare lo zero?
\subsection{Esercizio 2.7}
Costruire una tabella in cui si comparano, a partire dallo stesso punto iniziale $\mathtt{x_{0} = 0}$, e per valori decrescenti della tolleranza $\mathtt{tolx}$, il numero di iterazioni richieste per la convergenza dei metodi di Newton, corde e secanti, utilizzati per determinare lo zero della funzione\\
\center $f(x) = x - \cos{x}$
\flushleft
\subsection{Esercizio 2.8}
Completare i confronti del precedente esercizio inserendo quelli con il metodo di bisezione, con intervallo di confidenza iniziale $ [0,1]$.
\subsection{Esercizio 2.9}
Quali controlli introdurreste, negli algoritmi 2.4-2.6, al fine di rendere più "robuste" le corrispondenti iterazioni?

\chapter{Radici di una equazione}
\section{Esercizi del libro}
\subsection{Esercizio 2.1}
Definire una procedura iterativa basata sul metodo di Newton per determinare $\sqrt{a}$, per un assegnato $a>0$. Costruire una tabella dell'approssimazioni relativa al caso $a=x_{0}=2$ (Comparare con la tabella 1.1)
\subsection{Esercizio 2.2}
Generalizzare il risultato del precedente esercizio, derivando una procedura iterativa basata sul metodo di Newton per determinare $\sqrt[n]{a}$ per un assegnato $a>0$
\subsection{Esercizio 2.3}
In analogia con quanto visto nell'Esercizio 2.1, definire una procedura iterativa basata sul metodo delle secanti per determinare $\sqrt{a}$. Confrontare con l'esercizio 1.4.
\subsection{Esercizio 2.4}
Discutere la convergenza del metodo di Newton, applicato per determinare le radici dell'equazione (2.11) in funzione della scelta del punto iniziale $x_{0}$
\subsection{Esercizio 2.5}
Comparare il metodo di Newton (2.9), il metodo di Newton modificato (2.16) ed il metodo di accellerazione di Aitken (2.17), per approsimare gli zeri delle funzioni:\\
\center $f_{1}(x) = (x-1)^{10}, \qquad f_{2}(x)=(x-1)^{10}e^{x} $
\flushleft per valori decrescenti della tolleranza $\mathtt{tolx}$. Utilizzare, in tutti i casi, il punto iniziale $x_{0}=10$.
\subsection{Esercizio 2.6}
È possibile, nel caso delle funzioni del precedente esercizio utilizzare il metodo di bisezione per determinare lo zero?
\subsection{Esercizio 2.7}
Costruire una tabella in cui si comparano, a partire dallo stesso punto iniziale $\mathtt{x_{0} = 0}$, e per valori decrescenti della tolleranza $\mathtt{tolx}$, il numero di iterazioni richieste per la convergenza dei metodi di Newton, corde e secanti, utilizzati per determinare lo zero della funzione\\
\center $f(x) = x - \cos{x}$
\flushleft
\subsection{Esercizio 2.8}
Completare i confronti del precedente esercizio inserendo quelli con il metodo di bisezione, con intervallo di confidenza iniziale $ [0,1]$.
\subsection{Esercizio 2.9}
Quali controlli introdurreste, negli algoritmi 2.4-2.6, al fine di rendere più "robuste" le corrispondenti iterazioni?

\chapter{Capitolo 3}
\section{Esercizi del libro}
\subsection{Esercizio 3.1}
Riscrivere gli Algoritmi 3.1-3.4 in modo da controllare che la matrice dei coefficenti sia non singolare.
\subsection{Esercizio 3.2}
Dimostrare che la somma ed il prodotto di matricitriangolari inferiori(superiori), � una matrice triangolare inferiore (superiore).
\subsection{Esercizio 3.3}
Dimostrare che il prodotto di due matrici triangolari inferiori a diagonale unitaria � a sua volta una matrice triangolare inferiore a digonale unitaria
\subsection{Esercizio 3.4}
Dimostrare che la matrice inversa di una matrice triangolare inferiore � as ua volta triangolare inferiore. Dimostraarare inoltrre che, se la matrice ha diagonale unitaria, tale � anche la diagonale della sua inversa.
\subsection{Esercizio 3.5}
Dimostrare i lemmi 3.2 e 3.3.
\subsection{Esercizio 3.6}
Dimostrare che il numero di flop richiesti dall'algoritmo 3.5 � dato da (3.25)
\subsection{Esercizio 3.7}
Scrivere una function matlab che implementi efficientemente l'algoritmo 3.5 per calcolare la fattorizzazione LU di una matrice.
\subsection{Esercizio 3.8}
Scrivere una function Matlab che, avendo in ingresso la matrice A riscritta dall'algoritmo 3.g, ed un vettore x contenente i termini noti del sistema lineare (3.1), ne calcoli efficientemente la soluzione.
\subsection{Esercizio 3.9}
Dimostrare i lemmi 3.4 e 3.5
\subsection{Esercizio 3.10}
Completare la dimostrazione del Teorema 3.6
\subsection{Esercizio 3.11}
Dimostrare che , se A � non singolare, le matrici $A^{T}A$ e $AA^{T}$ sono sdp.
\subsection{Esercizio 3.12}

\subsection{Esercizio 3.13}
\subsection{Esercizio 3.14}
\chapter{Capitolo 3}
\section{Esercizi del libro}
\subsection{Esercizio 3.1}
Riscrivere gli Algoritmi 3.1-3.4 in modo da controllare che la matrice dei coefficenti sia non singolare.
\subsection{Esercizio 3.2}
Dimostrare che la somma ed il prodotto di matricitriangolari inferiori(superiori), � una matrice triangolare inferiore (superiore).
\subsection{Esercizio 3.3}
Dimostrare che il prodotto di due matrici triangolari inferiori a diagonale unitaria � a sua volta una matrice triangolare inferiore a digonale unitaria
\subsection{Esercizio 3.4}
Dimostrare che la matrice inversa di una matrice triangolare inferiore � as ua volta triangolare inferiore. Dimostraarare inoltrre che, se la matrice ha diagonale unitaria, tale � anche la diagonale della sua inversa.
\subsection{Esercizio 3.5}
Dimostrare i lemmi 3.2 e 3.3.
\subsection{Esercizio 3.6}
Dimostrare che il numero di flop richiesti dall'algoritmo 3.5 � dato da (3.25)
\subsection{Esercizio 3.7}
Scrivere una function matlab che implementi efficientemente l'algoritmo 3.5 per calcolare la fattorizzazione LU di una matrice.
\subsection{Esercizio 3.8}
Scrivere una function Matlab che, avendo in ingresso la matrice A riscritta dall'algoritmo 3.g, ed un vettore x contenente i termini noti del sistema lineare (3.1), ne calcoli efficientemente la soluzione.
\subsection{Esercizio 3.9}
Dimostrare i lemmi 3.4 e 3.5
\subsection{Esercizio 3.10}
Completare la dimostrazione del Teorema 3.6
\subsection{Esercizio 3.11}
Dimostrare che , se A � non singolare, le matrici $A^{T}A$ e $AA^{T}$ sono sdp.
\subsection{Esercizio 3.12}

\subsection{Esercizio 3.13}
\subsection{Esercizio 3.14}

\documentclass[20pt,a4paper]{book}

\usepackage[italian]{babel}
\usepackage[T1]{fontenc}
\usepackage[latin1]{inputenc}

\begin{document}  %INIZIO DOCUMENTO

\textbf{\Large{Esercizio 4.1}}
Sia f(x)=\begin{math}4x^{2}\end{math}-12x+1.\\
Determinare \begin{math} p(x)\in\pi_{4} \end{math} che interpola f(x) sulle ascisse \begin{math}{i}=i \end{math}, i=0,...,4.

\textit{Soluzione}
Caso Lagrange:
\\Per prima cosa si calcolano gli \begin{math}f(x_{i})\end{math} per ogni i=0,...,4:\\
f(0)=1\\
f(1)=-7\\
f(2)=-7\\
f(3)=1\\
f(4)=17\\

Adesso calcoliamo \begin{math}L_{kn}(x)\end{math} con k=0,...,4 e n=4:\\
\begin{math}L_{04}\end{math}=\begin{math}\frac{(x-1)(x-2)(x-3)(x-4)}{24}\end{math}\\

\begin{math}L_{14}\end{math}=\begin{math}\frac{(-x)(x-2)(x-3)(x-4)}{6}\end{math}\\
\begin{math}L_{24}\end{math}=\begin{math}\frac{(x)(x-1)(x-3)(x-4)}{4}\end{math}\\
\begin{math}L_{34}\end{math}=\begin{math}\frac{(-x)(x-1)(x-2)(x-4)}{6}\end{math}\\
\begin{math}L_{44}\end{math}=\begin{math}\frac{(x)(x-1)(x-2)(x-3)}{24}\end{math}\\

A questo punto possiamo scrivere \begin{math} p(x)\in\pi_{4} \end{math}= \begin{math}\frac{(x-1)(x-2)(x-3)(x-4)}{24}\end{math}-7\begin{math}\frac{(-x)(x-2)(x-3)(x-4)}{6}\end{math}-7\begin{math}\frac{(x)(x-1)(x-3)(x-4)}{4}\end{math}+\begin{math}\frac{(-x)(x-1)(x-2)(x-4)}{6}\end{math}+17\begin{math}\frac{(x)(x-1)(x-2)(x-3)}{24}\end{math}

\vspace{2mm}
Eseguendo i calcoli, si ottiene il polinomio \begin{math}p(x)=4x^{2}-12x+1\end{math}

\vspace{10mm}


\textbf{\Large{Esercizio 4.14}}
Quali diventano le ascisse di Chebyshev, per un problema definito su un generico intervallo [a,b]?

\textit{Soluzione}

Nel caso a=-1 e b=1, la formula per il calcolo delle ascisse di Chebyshev è:
\begin{math}\x x_{i}^{(k)}=cos(\frac{(2i+1)\pi}{2k})\end{math}  con k grado del polinomio e i=0,...,k.

Nel caso generico, la formula diventa:
\begin{math}\x x_{i}^{(k)}=\frac{a+b}{2}+\frac{b-a}{2}cos(\frac{(2i+1)\pi}{2k})\end{math}  con k grado del polinomio e i=0,...,k.



\vspace{10mm}



\end{document}


\documentclass[20pt,a4paper]{book}

\usepackage[italian]{babel}
\usepackage[T1]{fontenc}
\usepackage[latin1]{inputenc}

\begin{document}  %INIZIO DOCUMENTO

\textbf{\Large{Esercizio 4.1}}
Sia f(x)=\begin{math}4x^{2}\end{math}-12x+1.\\
Determinare \begin{math} p(x)\in\pi_{4} \end{math} che interpola f(x) sulle ascisse \begin{math}{i}=i \end{math}, i=0,...,4.

\textit{Soluzione}
Caso Lagrange:
\\Per prima cosa si calcolano gli \begin{math}f(x_{i})\end{math} per ogni i=0,...,4:\\
f(0)=1\\
f(1)=-7\\
f(2)=-7\\
f(3)=1\\
f(4)=17\\

Adesso calcoliamo \begin{math}L_{kn}(x)\end{math} con k=0,...,4 e n=4:\\
\begin{math}L_{04}\end{math}=\begin{math}\frac{(x-1)(x-2)(x-3)(x-4)}{24}\end{math}\\

\begin{math}L_{14}\end{math}=\begin{math}\frac{(-x)(x-2)(x-3)(x-4)}{6}\end{math}\\
\begin{math}L_{24}\end{math}=\begin{math}\frac{(x)(x-1)(x-3)(x-4)}{4}\end{math}\\
\begin{math}L_{34}\end{math}=\begin{math}\frac{(-x)(x-1)(x-2)(x-4)}{6}\end{math}\\
\begin{math}L_{44}\end{math}=\begin{math}\frac{(x)(x-1)(x-2)(x-3)}{24}\end{math}\\

A questo punto possiamo scrivere \begin{math} p(x)\in\pi_{4} \end{math}= \begin{math}\frac{(x-1)(x-2)(x-3)(x-4)}{24}\end{math}-7\begin{math}\frac{(-x)(x-2)(x-3)(x-4)}{6}\end{math}-7\begin{math}\frac{(x)(x-1)(x-3)(x-4)}{4}\end{math}+\begin{math}\frac{(-x)(x-1)(x-2)(x-4)}{6}\end{math}+17\begin{math}\frac{(x)(x-1)(x-2)(x-3)}{24}\end{math}

\vspace{2mm}
Eseguendo i calcoli, si ottiene il polinomio \begin{math}p(x)=4x^{2}-12x+1\end{math}

\vspace{10mm}


\textbf{\Large{Esercizio 4.14}}
Quali diventano le ascisse di Chebyshev, per un problema definito su un generico intervallo [a,b]?

\textit{Soluzione}

Nel caso a=-1 e b=1, la formula per il calcolo delle ascisse di Chebyshev è:
\begin{math}\x x_{i}^{(k)}=cos(\frac{(2i+1)\pi}{2k})\end{math}  con k grado del polinomio e i=0,...,k.

Nel caso generico, la formula diventa:
\begin{math}\x x_{i}^{(k)}=\frac{a+b}{2}+\frac{b-a}{2}cos(\frac{(2i+1)\pi}{2k})\end{math}  con k grado del polinomio e i=0,...,k.



\vspace{10mm}



\end{document}

\chapter{Capitolo 5. Formule di quadratura}
\label{chap:Capitolo 5. Formule di quadratura}

\subsection{Esercizio 1}
\label{sub:es1}
\emph{Calcolare il numero di condizionamento dell'integrale
$$\int_0^{e^{21}}\sin\sqrt{x}\;dx.$$
Questo problema è ben condizionato o è malcondizionato?}


\subsection{Esercizio 2}
\label{sub:es2}
\emph{Derivare, dalla (5.5), i coefficienti della formula dei trapezi (5.6) e della formula di Simpson (5.7).}

\subsection{Esercizio 3}
\label{sub:es3}
\emph{Verificare, utilizzando il risultato del Teorema (5.2) le (5.9) e (5.10).}

\subsection{Esercizio 4}
\label{sub:es4}
\emph{Scrivere una \lstinline{function} \textsc{Matlab} che implementi efficientemente la formula dei trapezi composita (5.11).}

\subsection{Esercizio 5}
\label{sub:es5}
\emph{Scrivere una \lstinline{function} \textsc{Matlab} che implementi efficientemente la formula di Simpson composita (5.13).}

\subsection{Esercizio 6}
\label{sub:es6}
\emph{Implementare efficientemente in \textsc{Matlab} la formula adattativa dei trapezi.}

\subsection{Esercizio 7}
\label{sub:es7}
\emph{Implementare efficientemente in \textsc{Matlab} la formula adattativa di Simpson.}

\subsection{Esercizio 8}
\label{sub:es8}
\emph{Come è classificabile, dal punto di vista del condizionamento, il seguente problema?
			$$\int_{\frac{1}{2}}^{100}-2x^{-3}\cos\left(x^{-2}\right)\;\mathrm{d}x\equiv\sin\left(10^{-4}\right)-\sin(4)$$}

\subsection{Esercizio 9}
\label{sub:es9}
\emph{Utilizzare le \lstinline{function} degli Esercizi 5.4 e 5.6 per il calcolo dell'integrale
      $$\int_{\frac{1}{2}}^{100}-2x^{-3}\cos\left(x^{-2}\right)\;\mathrm{d}x\equiv\sin\left(10^{-4}\right)-\sin(4),$$
			indicando gli errori commessi.
      Si utilizzi $n=1000,2000,\dots,10000$ per la formula dei trapezi composita e
      $tol=10^{-1},10^{-2},\dots,10^{-5}$ per la formula dei trapezi adattativa (indicando anche il numero di punti).}

\subsection{Esercizio 10}
\label{sub:es10}
\emph{Utilizzare le \lstinline{function} degli Esercizi 5.5 e 5.7 per il calcolo dell'integrale
      $$\int_{\frac{1}{2}}^{100}-2x^{-3}\cos\left(x^{-2}\right)\;\mathrm{d}x\equiv\sin\left(10^{-4}\right)-\sin(4),$$
			indicando gli errori commessi.
      Si utilizzi $n=1000,2000,\dots,10000$ per la formula di Simpson composita e
      $tol=10^{-1},10^{-2},\dots,10^{-5}$ per la formula di Simpson adattativa (indicando anche il numero di punti).}

\chapter{Capitolo 5. Formule di quadratura}
\label{chap:Capitolo 5. Formule di quadratura}

\subsection{Esercizio 1}
\label{sub:es1}
\emph{Calcolare il numero di condizionamento dell'integrale
$$\int_0^{e^{21}}\sin\sqrt{x}\;dx.$$
Questo problema è ben condizionato o è malcondizionato?}


\subsection{Esercizio 2}
\label{sub:es2}
\emph{Derivare, dalla (5.5), i coefficienti della formula dei trapezi (5.6) e della formula di Simpson (5.7).}

\subsection{Esercizio 3}
\label{sub:es3}
\emph{Verificare, utilizzando il risultato del Teorema (5.2) le (5.9) e (5.10).}

\subsection{Esercizio 4}
\label{sub:es4}
\emph{Scrivere una \lstinline{function} \textsc{Matlab} che implementi efficientemente la formula dei trapezi composita (5.11).}

\subsection{Esercizio 5}
\label{sub:es5}
\emph{Scrivere una \lstinline{function} \textsc{Matlab} che implementi efficientemente la formula di Simpson composita (5.13).}

\subsection{Esercizio 6}
\label{sub:es6}
\emph{Implementare efficientemente in \textsc{Matlab} la formula adattativa dei trapezi.}

\subsection{Esercizio 7}
\label{sub:es7}
\emph{Implementare efficientemente in \textsc{Matlab} la formula adattativa di Simpson.}

\subsection{Esercizio 8}
\label{sub:es8}
\emph{Come è classificabile, dal punto di vista del condizionamento, il seguente problema?
			$$\int_{\frac{1}{2}}^{100}-2x^{-3}\cos\left(x^{-2}\right)\;\mathrm{d}x\equiv\sin\left(10^{-4}\right)-\sin(4)$$}

\subsection{Esercizio 9}
\label{sub:es9}
\emph{Utilizzare le \lstinline{function} degli Esercizi 5.4 e 5.6 per il calcolo dell'integrale
      $$\int_{\frac{1}{2}}^{100}-2x^{-3}\cos\left(x^{-2}\right)\;\mathrm{d}x\equiv\sin\left(10^{-4}\right)-\sin(4),$$
			indicando gli errori commessi.
      Si utilizzi $n=1000,2000,\dots,10000$ per la formula dei trapezi composita e
      $tol=10^{-1},10^{-2},\dots,10^{-5}$ per la formula dei trapezi adattativa (indicando anche il numero di punti).}

\subsection{Esercizio 10}
\label{sub:es10}
\emph{Utilizzare le \lstinline{function} degli Esercizi 5.5 e 5.7 per il calcolo dell'integrale
      $$\int_{\frac{1}{2}}^{100}-2x^{-3}\cos\left(x^{-2}\right)\;\mathrm{d}x\equiv\sin\left(10^{-4}\right)-\sin(4),$$
			indicando gli errori commessi.
      Si utilizzi $n=1000,2000,\dots,10000$ per la formula di Simpson composita e
      $tol=10^{-1},10^{-2},\dots,10^{-5}$ per la formula di Simpson adattativa (indicando anche il numero di punti).}

\chapter{Calcolo del Google Pagerank}
\label{chap:Google}

\subsection{Esercizio 1}
\label{sub:Es1}
[Teorema di Gershgorin]
      Dimostrare che gli autovalori di una matrice
      $A=(a_{ij})\in\mathbb{C}^{n n}$
      \emph{sono contenuti nell'insieme}
			\[
				\mathcal{D}=\bigcup_{i=1}^n\mathcal{D}_i,\qquad \mathcal{D}_i=\left\{\lambda\in\mathbb{C}:|\lambda-a_{ii}|\leq\sum_{\begin{subarray}{c}
					j=1\\
					j\neq i
				\end{subarray}}^n|a_{ij}|\right\},\quad i=1,\dots,n.
			\]

\subsection{Esercizio 2}
\label{sub:Es2}
\emph{
      Utilizzare il metodo delle potenze per approssimare l'autovalore dominante della matrice
			\[
				A_n=\begin{pmatrix}
					2 & -1 & &\\
					-1 & 2 & \ddots &\\
					& \ddots & \ddots & -1\\
					& & -1 & 2
				\end{pmatrix}\in\mathbb{R}^{n\times n},
			\]
			per valori crescenti di $n$. Verificare numericamente che questo è dato da $2\left(1+\cos\frac{\pi}{n+1}\right)$.
}

\subsection{Esercizio 3}
\label{sub:es3}
\emph{Dimostrare i Corollari 6.2 e 6.3.}

\subsection{Esercizio 4}
\label{sub:es4}
\emph{Dimostrare il Teorema 6.9.}

\subsection{Esercizio 5}
\label{sub:es5}
\emph{Tenendo conto della (6.10), riformulare il metodo delle potenze (6.11) per il calcolo del \textit{Google pagerank} come metodo iterativo definito da uno splitting regolare.}

\subsection{Esercizio 6}
\label{sub:es6}
\emph{Dimostrare che il metodo di Jacobi converge asintoticamente in un numero minore di iterazioni, rispetto al metodo delle potenze (6.11) per il calcolo del \textit{Google pagerank}.}

\subsection{Esercizio 7}
\label{sub:es7}
\emph{Dimostrare che, se $A$ è diagonale dominante, per riga o per colonna, il metodo di Jacobi è convergente.}

\subsection{Esercizio 8}
\label{sub:es8}
\emph{Dimostrare che, se $A$ è diagonale dominante, per riga o per colonna, il metodo di Gauss-Seidel è convergente.}

\subsection{Esercizio 9}
\label{sub:es9}
\emph{Se $A$ è \textit{sdp}, il metodo di Gauss-Seidel risulta essere convergente.
      Dimostrare questo risultato nel caso (assai più semplice) in cui l'autovalore di massimo modulo della matrice di iterazione sia reale.\\
			(\underline{Suggerimento:} considerare il sistema lineare equivalente
			$$(D^{-\frac{1}{2}}AD^{-\frac{1}{2}})(D^{\frac{1}{2}}\underline{x})=(D^{-\frac{1}{2}}\underline{b}),\qquad D^{\frac{1}{2}}=diag(\sqrt{a_{11}},\dots,\sqrt{a_{nn}}),$$
			la cui matrice dei coefficienti è ancora \textit{sdp} ma ha diagonale unitaria,
      ovvero del tipo $I-L-L^T$. Osservare quindi che, per ogni vettore reale $\underline{v}$ di norma $1$,si ha:
      $\underline{v}^TL\underline{v}=\underline{v}^TL^T\underline{v}=\frac{1}{2}\underline{v}^T(L+L^T)\underline{v}<\frac{1}{2}$.)}

\subsection{Esercizio 10}
\label{sub:es10}
\emph{Con riferimento ai vettori errore (6.16) e residuo (6.17) dimostrare che, se
			\begin{equation}
				\label{criterioArrestoSplitting}
				||\underline{r_k}||\leq\varepsilon||\underline{b}||,
			\end{equation}
			allora
			$$||\underline{e_k}||\leq\varepsilon k(A)||\underline{\hat{x}}||,$$
			dove $k(A)$ denota, al solito, il numero di condizionamento della matrice $A$.
      Concludere che, per sistemi lineari malcondizionati, anche la risoluzione iterativa (al pari di quella diretta) risulta essere più problematica.}

\subsection{Esercizio 11}
\label{sub:es11}
\emph{Calcolare il polinomio caratteristico della matrice
			\[
				\begin{pmatrix}
					0 & \dots & 0 & \alpha\\
					1 & \ddots & & 0\\
					& \ddots & \ddots & \vdots\\
					0 & & 1 & 0
				\end{pmatrix}\in\mathbb{R}^{n\times n}.
			\]}

\subsection{Esercizio 12}
\label{sub:es12}
\emph{Dimostrare che i metodi di Jacobi e Gauss-Seidel possono essere utilizzati per la risoluzione del sistema lineare (gli elementi non indicati sono da intendersi nulli)
			\[
				\begin{pmatrix}
					1 & & & -\frac{1}{2}\\
					-1 & 1 & &\\
					& \ddots & \ddots &\\
					& & -1 & 1
				\end{pmatrix}\underline{x}=\begin{pmatrix}
					\frac{1}{2}\\
					0\\
					\vdots\\
					0
				\end{pmatrix}\in\mathbb{R}^n,
			\]
			la cui soluzione è $\underline{x}=(1,\dots,1)^T\in\mathbb{R}^n$.
      Confrontare il numero di iterazioni richieste dai due metodi per soddisfare lo stesso criterio di arresto (6.19),
      per valori crescenti di $n$ e per tolleranze $\varepsilon$ decrescenti. Riportare i risultati ottenuti in una tabella $(n/\varepsilon)$.
      }

\chapter{Calcolo del Google Pagerank}
\label{chap:Google}

\subsection{Esercizio 1}
\label{sub:Es1}
[Teorema di Gershgorin]
      Dimostrare che gli autovalori di una matrice
      $A=(a_{ij})\in\mathbb{C}^{n n}$
      \emph{sono contenuti nell'insieme}
			\[
				\mathcal{D}=\bigcup_{i=1}^n\mathcal{D}_i,\qquad \mathcal{D}_i=\left\{\lambda\in\mathbb{C}:|\lambda-a_{ii}|\leq\sum_{\begin{subarray}{c}
					j=1\\
					j\neq i
				\end{subarray}}^n|a_{ij}|\right\},\quad i=1,\dots,n.
			\]

\subsection{Esercizio 2}
\label{sub:Es2}
\emph{
      Utilizzare il metodo delle potenze per approssimare l'autovalore dominante della matrice
			\[
				A_n=\begin{pmatrix}
					2 & -1 & &\\
					-1 & 2 & \ddots &\\
					& \ddots & \ddots & -1\\
					& & -1 & 2
				\end{pmatrix}\in\mathbb{R}^{n\times n},
			\]
			per valori crescenti di $n$. Verificare numericamente che questo è dato da $2\left(1+\cos\frac{\pi}{n+1}\right)$.
}

\subsection{Esercizio 3}
\label{sub:es3}
\emph{Dimostrare i Corollari 6.2 e 6.3.}

\subsection{Esercizio 4}
\label{sub:es4}
\emph{Dimostrare il Teorema 6.9.}

\subsection{Esercizio 5}
\label{sub:es5}
\emph{Tenendo conto della (6.10), riformulare il metodo delle potenze (6.11) per il calcolo del \textit{Google pagerank} come metodo iterativo definito da uno splitting regolare.}

\subsection{Esercizio 6}
\label{sub:es6}
\emph{Dimostrare che il metodo di Jacobi converge asintoticamente in un numero minore di iterazioni, rispetto al metodo delle potenze (6.11) per il calcolo del \textit{Google pagerank}.}

\subsection{Esercizio 7}
\label{sub:es7}
\emph{Dimostrare che, se $A$ è diagonale dominante, per riga o per colonna, il metodo di Jacobi è convergente.}

\subsection{Esercizio 8}
\label{sub:es8}
\emph{Dimostrare che, se $A$ è diagonale dominante, per riga o per colonna, il metodo di Gauss-Seidel è convergente.}

\subsection{Esercizio 9}
\label{sub:es9}
\emph{Se $A$ è \textit{sdp}, il metodo di Gauss-Seidel risulta essere convergente.
      Dimostrare questo risultato nel caso (assai più semplice) in cui l'autovalore di massimo modulo della matrice di iterazione sia reale.\\
			(\underline{Suggerimento:} considerare il sistema lineare equivalente
			$$(D^{-\frac{1}{2}}AD^{-\frac{1}{2}})(D^{\frac{1}{2}}\underline{x})=(D^{-\frac{1}{2}}\underline{b}),\qquad D^{\frac{1}{2}}=diag(\sqrt{a_{11}},\dots,\sqrt{a_{nn}}),$$
			la cui matrice dei coefficienti è ancora \textit{sdp} ma ha diagonale unitaria,
      ovvero del tipo $I-L-L^T$. Osservare quindi che, per ogni vettore reale $\underline{v}$ di norma $1$,si ha:
      $\underline{v}^TL\underline{v}=\underline{v}^TL^T\underline{v}=\frac{1}{2}\underline{v}^T(L+L^T)\underline{v}<\frac{1}{2}$.)}

\subsection{Esercizio 10}
\label{sub:es10}
\emph{Con riferimento ai vettori errore (6.16) e residuo (6.17) dimostrare che, se
			\begin{equation}
				\label{criterioArrestoSplitting}
				||\underline{r_k}||\leq\varepsilon||\underline{b}||,
			\end{equation}
			allora
			$$||\underline{e_k}||\leq\varepsilon k(A)||\underline{\hat{x}}||,$$
			dove $k(A)$ denota, al solito, il numero di condizionamento della matrice $A$.
      Concludere che, per sistemi lineari malcondizionati, anche la risoluzione iterativa (al pari di quella diretta) risulta essere più problematica.}

\subsection{Esercizio 11}
\label{sub:es11}
\emph{Calcolare il polinomio caratteristico della matrice
			\[
				\begin{pmatrix}
					0 & \dots & 0 & \alpha\\
					1 & \ddots & & 0\\
					& \ddots & \ddots & \vdots\\
					0 & & 1 & 0
				\end{pmatrix}\in\mathbb{R}^{n\times n}.
			\]}

\subsection{Esercizio 12}
\label{sub:es12}
\emph{Dimostrare che i metodi di Jacobi e Gauss-Seidel possono essere utilizzati per la risoluzione del sistema lineare (gli elementi non indicati sono da intendersi nulli)
			\[
				\begin{pmatrix}
					1 & & & -\frac{1}{2}\\
					-1 & 1 & &\\
					& \ddots & \ddots &\\
					& & -1 & 1
				\end{pmatrix}\underline{x}=\begin{pmatrix}
					\frac{1}{2}\\
					0\\
					\vdots\\
					0
				\end{pmatrix}\in\mathbb{R}^n,
			\]
			la cui soluzione è $\underline{x}=(1,\dots,1)^T\in\mathbb{R}^n$.
      Confrontare il numero di iterazioni richieste dai due metodi per soddisfare lo stesso criterio di arresto (6.19),
      per valori crescenti di $n$ e per tolleranze $\varepsilon$ decrescenti. Riportare i risultati ottenuti in una tabella $(n/\varepsilon)$.
      }

\end{document}
