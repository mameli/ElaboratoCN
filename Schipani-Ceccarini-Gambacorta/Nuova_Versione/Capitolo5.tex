\chapter{Formule di quadratura}
\label{chap:Cap5}

\section{Esercizio 1}
\label{sub:es1}
\emph{Calcolare il numero di condizionamento dell'integrale
$$\int_0^{e^{21}}\sin\sqrt{x}\;dx.$$
Questo problema è ben condizionato o è malcondizionato?}\\
\underline{Soluzione:}\\
$$ I(f)=\int_{0}^{e^{21}}\sin\sqrt{x}\;dx $$
Poichè $\kappa = b-a$, dove $[a,b]$ è l'intervallo d'integrazione, in questo caso si
ha che $\kappa = {e^{21}}-0$ ed il problema risulta mal condizionato.

\section{Esercizio 2}
\label{sub:es2}
\emph{Derivare, dalla (5.5), i coefficienti della formula dei trapezi (5.6) e della formula di Simpson (5.7).}\\
\underline{Soluzione:}\\
Per generare i coefficenti si usa la formula (5.5) del libro.
\begin{itemize}
\item Per la formula dei trapezi si usa $n=1$ ed i coefficenti risultano così:
\[
c_{1,1}=\int_{0}^{1}\frac{t-0}{1-0}dt=\int_{0}^{1}t\, dt=\frac{t^{2}}{2}|_{0}^{1}=\frac{1}{2}
\]
\[
c_{0,1}=1-c_{1,1}=\frac{1}{2}
\]

\item Per la formula di Simpson si usa $n=2$ ed i coefficenti risultano così:
\[
c_{1,2}=\int_{0}^{2}\frac{t-0}{1-0}\frac{t-2}{1-2}dt=\int_{0}^{2}t(t-2)dt=\left[\frac{t^{3}}{3}-t^{2}\right]_{0}^{2}=\frac{4}{3}
\]
\[
c_{22}=\int_{0}^{2}\frac{t-0}{2-0}\frac{t-1}{2-1}dt=\int_{0}^{2}\frac{t(t-1)}{2}dt=\left[\frac{t^{3}}{6}-\frac{t^{2}}{4}\right]_{0}^{2}=\frac{1}{3}
\]
\[
c_{0,2}=1-c_{1,2}-c_{2,2}=\frac{1}{3}
\]
\end{itemize}

\section{Esercizio 3}
\label{sub:es3}
\emph{Verificare, utilizzando il risultato del Teorema (5.2) le (5.9) e (5.10).}\\
\underline{Soluzione:}\\
Per $n=1$ si ha che:\\
$$\nu_{n}=\int_{0}^{1}\prod_{0}^{1}=\int_{0}^{1}t(t-1)dt=\int_{0}^{1}t_{2}-t\;dt=\left[\frac{t^{3}}{3}-\frac{t^2}{2}\right]_{0}^{1}=-\frac{1}{6}$$\\
da cui:\\
$$E_{1}(f)=-\frac{1}{6}\frac{f^{(2)}}{2!}(\xi)(b-a)^{3}=-\frac{1}{12} f^{(2)}(\xi)(b-a)^{3},\qquad \xi\in[a,b].$$
\section{Esercizio 4}
\label{sub:es4}
\emph{Scrivere una \lstinline{function} \textsc{Matlab} che implementi efficientemente la formula dei trapezi composita (5.11).}\\
\underline{Soluzione:}\\
\lstinputlisting[language=Matlab]{code/trapeziComposita.m}
\section{Esercizio 5}
\label{sub:es5}
\emph{Scrivere una \lstinline{function} \textsc{Matlab} che implementi efficientemente la formula di Simpson composita (5.13).}\\
\underline{Soluzione:}\\
\lstinputlisting[language=Matlab]{code/simpsonComposita.m}
\section{Esercizio 6}
\label{sub:es6}
\emph{Implementare efficientemente in \textsc{Matlab} la formula adattativa dei trapezi.}\\
\underline{Soluzione:}\\
\lstinputlisting[language=Matlab]{code/trapeziAdattativa.m}

\section{Esercizio 7}
\label{sub:es7}
\emph{Implementare efficientemente in \textsc{Matlab} la formula adattativa di Simpson.}\\
\underline{Soluzione:}\\
\lstinputlisting[language=Matlab]{code/simpsonAdattativa.m}
\section{Esercizio 8}
\label{sub:es8}
\emph{Come è classificabile, dal punto di vista del condizionamento, il seguente problema?
			$$\int_{\frac{1}{2}}^{100}-2x^{-3}\cos\left(x^{-2}\right)\;\mathrm{d}x\equiv\sin\left(10^{-4}\right)-\sin(4)$$}\\
\underline{Soluzione:}\\

\section{Esercizio 9}
\label{sub:es9}
\emph{Utilizzare le \lstinline{function} degli Esercizi 5.4 e 5.6 per il calcolo dell'integrale
      $$\int_{\frac{1}{2}}^{100}-2x^{-3}\cos\left(x^{-2}\right)\;\mathrm{d}x\equiv\sin\left(10^{-4}\right)-\sin(4),$$
			indicando gli errori commessi.
      Si utilizzi $n=1000,2000,\dots,10000$ per la formula dei trapezi composita e
      $tol=10^{-1},10^{-2},\dots,10^{-5}$ per la formula dei trapezi adattativa (indicando anche il numero di punti).}\\
			\underline{Soluzione:}\\

			\begin{center}\begin{tabular}{c|c|c}
			\hline\multicolumn{3}{c}{Trapezi Composita}\\\hline
			$n$ & $I$ & $E_1^{(n)}$\\\hline
			1000&6.6401e-001&9.2897e-002\\
			2000&7.3077e-001&2.6131e-002\\
			3000&7.4507e-001&1.1836e-002\\
			4000&7.5020e-001&6.7007e-003\\
			5000&7.5260e-001&4.3009e-003\\
			6000&7.5391e-001&2.9914e-003\\
			7000&7.5470e-001&2.1998e-003\\
			8000&7.5522e-001&1.6853e-003\\
			9000&7.5557e-001&1.3321e-003\\
			10000&7.5582e-001&1.0793e-003
			\end{tabular}\end{center}
			\begin{center}
			\begin{tabular}{c|c|c|c}
			\hline\multicolumn{4}{c}{Trapezi adattativa}\\\hline
			$tol$ & $I$ & $E_1^{(n)}$ & points\\\hline
				1.0e-001&7.5143e-001&5.4696e-003&159\\
				1.0e-002&7.5563e-001&1.2676e-003&471\\
				1.0e-003&7.5657e-001&3.3005e-004&1567\\
				1.0e-004&7.5684e-001&6.5936e-005&4851
			\end{tabular}\end{center}


\section{Esercizio 10}
\label{sub:es10}
\emph{Utilizzare le \lstinline{function} degli Esercizi 5.5 e 5.7 per il calcolo dell'integrale
      $$\int_{\frac{1}{2}}^{100}-2x^{-3}\cos\left(x^{-2}\right)\;\mathrm{d}x\equiv\sin\left(10^{-4}\right)-\sin(4),$$
			indicando gli errori commessi.
      Si utilizzi $n=1000,2000,\dots,10000$ per la formula di Simpson composita e
      $tol=10^{-1},10^{-2},\dots,10^{-5}$ per la formula di Simpson adattativa (indicando anche il numero di punti).}\\
			\underline{Soluzione:}\\
			\begin{center}\begin{tabular}{c|c|c}
			\hline\multicolumn{3}{c}{Simpson composita}\\\hline
			$n$ & $I$ & $E_1^{(n)}$\\\hline
			 1000 	  &7.0132e-001 	  &5.5580e-002\\
			 2000 	  &7.5303e-001 	  &3.8753e-003\\
			 3000 	  &7.5617e-001 	  &7.2977e-004\\
			 4000 	  &7.5668e-001 	  &2.2403e-004\\
			 5000 	  &7.5681e-001 	  &9.0209e-005\\
			 6000 	  &7.5686e-001 	  &4.3062e-005\\
			 7000 	  &7.5688e-001 	  &2.3094e-005\\
			 8000 	  &7.5689e-001 	  &1.3479e-005\\
			 9000 	  &7.5689e-001 	  &8.3892e-006\\
			 10000 	  &7.5690e-001 	  &5.4921e-006
			\end{tabular}\end{center}\begin{center}
			\begin{tabular}{c|c|c|c}
			\hline\multicolumn{4}{c}{Simpson Adattativa}\\\hline
			$tol$ & $I$ & $E_1^{(n)}$ & points\\\hline
			 1.0e-001 	  &7.5701e-001 	  &1.1164e-004 	  &49\\
			 1.0e-002 	  &7.5671e-001 	  &1.9384e-004 	  &65\\
			 1.0e-003 	  &7.5690e-001 	  &4.8068e-006 	  &93\\
			 1.0e-004 	  &7.5688e-001 	  &1.7808e-005 	  &181\\
			 1.0e-005 	  &7.5690e-001 	  &4.8337e-006 	  &309
			\end{tabular}\end{center}
