\chapter{Capitolo 3}
\section{Esercizi del libro}
\subsection{Esercizio 3.1}
\emph{Riscrivere gli Algoritmi 3.1-3.4 in modo da controllare che la matrice dei coefficenti sia non singolare.}
\subsection{Esercizio 3.2}
\emph{Dimostrare che la somma ed il prodotto di matrici triangolari inferiori (superiori), è una matrice triangolare inferiore (superiore).}
\subsection{Esercizio 3.3}
\emph{Dimostrare che il prodotto di due matrici triangolari inferiori a diagonale unitaria è a sua volta una matrice triangolare inferiore a diagonale unitaria.}
\subsection{Esercizio 3.4}
\emph{Dimostrare che la matrice inversa di una matrice triangolare inferiore è as ua volta triangolare inferiore. Dimostrare inoltre che, se la matrice ha diagonale unitaria, tale è anche la diagonale della sua inversa.}
\subsection{Esercizio 3.5}
\emph{Dimostrare i lemmi 3.2 e 3.3.}
\subsection{Esercizio 3.6}
\emph{Dimostrare che il numero di flop richiesti dall'algoritmo 3.5 è dato da (3.25)}
\subsection{Esercizio 3.7}
\emph{Scrivere una function matlab che implementi efficientemente l'algoritmo 3.5 per calcolare la fattorizzazione LU di una matrice.}
\subsection{Esercizio 3.8}
\emph{Scrivere una function Matlab che, avendo in ingresso la matrice A riscritta dall'algoritmo 3.5, ed un vettore x contenente i termini noti del sistema lineare (3.1), ne calcoli efficientemente la soluzione.}
\subsection{Esercizio 3.9}
\emph{Dimostrare i lemmi 3.4 e 3.5}
\subsection{Esercizio 3.10}
\emph{Completare la dimostrazione del Teorema 3.6}
\subsection{Esercizio 3.11}
\emph{Dimostrare che , se A è non singolare, le matrici $A^{T}A$ e $AA^{T}$ sono sdp.}
\subsection{Esercizio 3.12}
\emph{Dimostrare che se A $\in \mathbb{R}^{m \times n} $ con $ m \geq n = rank(A)$, allora la matrice $A^{T} A$ è sdp.}
\subsection{Esercizio 3.13}
\emph{Data una matrice A $\in \mathbb{R}^{n \times n} $, dimostrare che essa può essere scritta come
$$A = \frac{1}{2} (A+A^{T})+\frac{1}{2} (A-A^{T}) \equiv A_{s} + A_{a}$$
dove $ A_{s} = A_{s}^{T} $ è detta parte simmetrica di A, mentre $A_{a} = - A^{T}_{a} $ è detta parte asimmetrica di A. Dimostrare inoltre che, dato un generico vettore $ x \in \mathbb{R}^{n}$, risulta} $$ x^{T} Ax = x^{T} A_{s} x $$
 \subsection{Esercizio 3.14}
\emph{Dimostrare la consistenza delle formule (3.29).}
\subsection{Esercizio 3.15}
\emph{Dimostrare che il numero di flop richiesti dall'algoritmo 3.6 è dato da 3.30 }
\subsection{Esercizio 3.16}
\emph{Scrivere una function Matlab che implementi efficientemente l'algoritmo di fattorizzazione $LDL^{T}$ per matrici sdp}
\subsection{Esercizio 3.17}
\emph{Scrivere una funzione Matlab che, avendo in ingresso la matrice A prodotta dalla precedente funziona, contenente la fattorizzazione $LDL^{T}$ della matrice sdp originaria, ed un vettore di termini noti, x, calcoli efficientemente la soluzione del corrispondente sistema lineare.}
\subsection{Esercizio 3.18}
\emph{Utilizzare la function dell'esercizio 3.16 per verificare che la matrice}\\
   $$\begin{pmatrix}
      1 & 1 & 1 & 1 \\
      1 & 2 & 2 & 2 \\
      1 & 2 & 1 & 1 \\
      1 & 2 & 1 & 2 \\
   \end{pmatrix}$$
\emph{non è sdp.}
\subsection{Esercizio 3.19}
\emph{Dimostrare che, al passo i-esimo di eliminazione di Gauss con pivoting parziale, si ha $a_{k_{i}i}^{(i)} \neq 0 $, se A è non singolare.}
\subsection{Esercizio 3.20}
\emph{Con riferimento alla matrice A definita nella (3.24), qual è la matrice di permutazione P che rende PA fattorizzatile LU? Chi sono, in tal caso, i fattori L e U?}
\subsection{Esercizio 3.21}
\emph{Scrivere una function Matlab che implementi efficientemente l'algoritmo di fattorizzazione LU con pivoting parziale.}
\subsection{Esercizio 3.22}
\emph{Scrivere una funzione matlab che, avendo in ingresso la matrice A prodotta dalla precedente function, contenente la fattorizzazione LU della matrice permutata, il vettore p contenente l'informazione relativa alla corrispondente matrice di permutazione, ed un vettore di termini noti, x, calcoli efficientemente la soluzione del corrispondente sistema lineare.}
\subsection{Esercizio 3.23}
\emph{Costruire alcuni esempi di applicazione delle funzioni degli esercizi 3.21 e 3.22}
\subsection{Esercizio 3.24}
\emph{Le seguenti istruzioni \textsc{Matlab} sono equivalenti a risolvere il dato sistema $2\times 2$ con l'utilizzo del pivoting e non, rispettivamente. Spiegarne il differente risultato ottenuto. Concludere che l'utilizzo del pivoting migliora, in generale, la prognosi degli errori in aritmetica finita.}
\lstinputlisting[frame=none, stepnumber=0, nolol=true]{code/matlab3_24.m}
\subsection{Esercizio 3.25}
\emph{Si consideri la seguente matrice \textit{bidiagonale inferiore}}
\[
  A=
  \begin{pmatrix}
    1 & & &\\
    100 & 1 & &\\
    & \ddots & \ddots &\\
    & & 100 & 1
  \end{pmatrix}_{10\times 10}.
\]
\emph{Calcolare $k_{\infty}(A)$. Confrontate il risultato con quello fornito dalla \lstinline{function cond} di \textsc{Matlab}. Dimostrare, e verificare, che $k_{\infty}(A)=k_1(A)$.}
\subsection{Esercizio 3.26}
\emph{Si consideri i seguenti vettori di $\mathbb{R}^{10}$,}
\[
  \underline{b}=
  \begin{pmatrix}
    1\\
    101\\
    \vdots\\
    101
  \end{pmatrix},
  \quad \underline{c}=0.1\cdot
  \begin{pmatrix}
    1\\
    101\\
    \vdots\\
    101
  \end{pmatrix},
\]
\emph{ed i seguenti sistemi lineari
$$A\underline{x}=\underline{b},\qquad A\underline{y}=\underline{c},$$
in cui $A$ è la matrice definita nel precedente Esercizio \ref{es:3.25}. Verificare che le soluzioni di questi sistemi lineari sono, rispettivamente, date da:}
\[
  \underline{x}=
  \begin{pmatrix}
    1\\
    \vdots\\
    1
  \end{pmatrix},
  \qquad \underline{y}=
  \begin{pmatrix}
    0.1\\
    \vdots\\
    0.1
  \end{pmatrix}.
\]
\emph{Confrontare questi vettori con quelli calcolato dalle seguenti due serie di istruzioni \textsc{Matlab},}
\lstinputlisting[frame=none, stepnumber=0, nolol=true]{code/matlab3_26.m}
\emph{che implementano, rispettivamente, le risoluzioni dei due sistemi lineari. Spiegare i risultati ottenuti, alla luce di quanto visto in Sezione 3.7.}
\subsection{Esercizio 3.27}
\emph{Dimostrare che il numero di \texttt{flop} richiesti dall'Algoritmo di fattorizzazione $QR$ di Householder è dato da 3.60.}
\subsection{Esercizio 3.28}
\emph{Definendo il vettore $\underline{\hat{v}}=\frac{\underline{v}}{v_1}$, verificare che \lstinline{beta}, come definito nel seguente algoritmo per la fattorizzazione $QR$ di Householder, corrisponde alla quantità $\frac{2}{\underline{\hat{v}}^T\underline{\hat{v}}}$}
\subsection{Esercizio 3.29}
\emph{Scrivere una \lstinline{function} \textsc{Matlab} che implementi efficientemente l'algoritmo di fattorizzazione $QR$, mediante il metodo di Householder (vedi l'Algoritmo descritto nell'Esercizio precedente).}
\subsection{Esercizio 3.30}
\emph{Scrivere una \lstinline{function} \textsc{Matlab} che, avendo in ingresso la matrice $A$ prodotta dalla \lstinline{function} del precedente Esercizio, contenente la fattorizzazione $QR$ della matrice originaria, e un corrispondente vettore di termini noti $\underline{b}$, calcoli efficientemente la soluzione del sistema lineare sovradeterminato.}
\subsection{Esercizio 3.31}
\emph{Utilizzare le \lstinline{function} degli Esercizi 3.29 e 3.30 per calcolare la soluzione ai minimi quadrati di (3.51), ed il corrispondente residuo, nel caso in cui}
\[
  A=
  \begin{pmatrix}
    3 & 2 & 1\\
    1 & 2 & 3\\
    1 & 2 & 1\\
    2 & 1 & 2
  \end{pmatrix},
  \qquad \underline{b}=
  \begin{pmatrix}
    10\\
    10\\
    10\\
    10
  \end{pmatrix}.
\]
\subsection{Esercizio 3.32}
\emph{Calcolare i coefficienti della equazione della retta
$$r(x)=a_1x+a_2,$$
che meglio approssima i dati prodotti dalle seguenti istruzioni \textsc{Matlab}:}
\lstinputlisting[frame=none, stepnumber=0, nolol=true]{code/matlab3_32a.m}
\emph{Riformulare il problema come minimizzazione della norma Euclidea di un corrispondente vettore residuo. Calcolare la soluzione utilizzando le \lstinline{function} sviluppate negli Esercizi 3.29 e 3.30, e confrontarla con quella ottenuta dalle seguenti istruuzioni \textsc{Matlab}:}
\lstinputlisting[frame=none, stepnumber=0, nolol=true]{code/matlab3_32b.m}
\emph{Confrontare i risultati che si ottengono per i seguenti valori del parametro \lstinline{gamma}:
$$0.5,0.1,0.05,0.01,0.005,0.001.$$
Quale è la soluzione che si ottiene nel limite \lstinline{gamma} $\rightarrow 0$?}
\subsection{Esercizio 3.33}
\emph{Determinare il punto di minimo della funzione $f(x_1,x_2)=x_1^4+x_1(x_1+x_2)+(1-x_2)^2$, utilizzando il metodo di Newton (3.63) per calcolarne il punto stazionario.}
\subsection{Esercizio 3.34}
\emph{Uno dei metodi di base per la risoluzione di equazioni differenziali ordinarie, $y'(t)=f(t,y(t))$, $t\in[t_0,T]$, $y(t_0)=y_0$ (problema di Cauchy di prim'ordine), è il metodo di Eulero implicito,
$$y_n=y_{n-1}+hf_n,\qquad n=1,2,\dots,N\equiv\frac{T-t_0}{h},$$
in cui $y_n\approx y(t_n)$, $f_n=f(t_n,y_n)$, $t_n=t_0+nh$. Utilizzare questo metodo nel caso in cui $t_0=0$, $T=10$, $N=100$, $y_0=(1,2)^T$ e, se $y=(x_1,x_2)^T$, $f(t,y)=(-10^3x_1+\sin x_1\cos x_2, -2x_2+\sin x_1 \cos x_2)^T$. Utilizzare il metodo (3.64) per la risoluzione dei sistemi nonlineari richiesti.}
