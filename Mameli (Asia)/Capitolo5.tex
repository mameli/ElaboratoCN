\chapter{Formule di quadratura}
\label{chap:Cap5}

\section{Esercizio 1}
\label{sub:es1}
\emph{Calcolare il numero di condizionamento dell'integrale
$$\int_0^{e^{21}}\sin\sqrt{x}\;dx.$$
Questo problema è ben condizionato o è malcondizionato?}
\begin{sol}
	$\kappa$ definisce il numero di condizionamento ed è dato da $\kappa=b-a$\\
	per cui abbiamo $\kappa=b-a=e^{21}-0=e^{21}>10^{9}$,il	problema è, quindi, mal condizionato.
\end{sol}

\sectionline{black}{88}

\section{Esercizio 2}
\label{sub:es2}
\emph{Derivare, dalla (5.5), i coefficienti della formula dei trapezi (5.6) e della formula di Simpson (5.7).}
\begin{sol}
	\begin{itemize}
		\item Formula dei trapezi, $n=1$:
			\begin{itemize}
				\item $c_{1,1}=\int_0^1{\frac{t-0}{1-0}\:dt}=\int_0^1{t}=\left.\frac{t^2}{2}\right|_0^1=\frac{1}{2}$,\\
				\item $c_{0,1}=1-c_{1,1}=\frac{1}{2}$ \\
				per la formula dei coefficienti di Newton Cotes\\
				\vspace{1em}
				\begin{equation}
					c_{k,n}=\int_0^n \prod_{
						\begin{subarray}{c}
							j=0\\
							j\neq k
						\end{subarray}
						}^{n}\frac{t-j}{k-j} dt, \qquad k=0,\dots,n,
				\end{equation}
			\end{itemize}
			Segue $I_1(f)=(b-a)\left(\frac{1}{2}f(a)+\frac{1}{2}f(b)\right)=\frac{b-a}{2}\left(f(a)+f(b)\right)$.
		\item Formula di Simpson, $n=2$:
			\begin{itemize}
				\item $c_{1,2}=\int_0^2{\frac{t-0}{1-0}\frac{t-2}{1-2}\:dt}=\int_2^0{t(t-2)}=\left.\left(\frac{t^3}{3}-t^2\right)\right|_0^2=\frac{4}{3}$,
				\item $c_{2,2}=\int_0^2{\frac{t-0}{2-0}\frac{t-1}{2-1}\:dt}=\int_2^0{\frac{t(t-1)}{2}}=\left.\left(\frac{t^3}{6}-\frac{t^2}{4}\right)\right|_0^2=\frac{1}{3}$,
				\item $c_{0,2}=1-c_{1,2}-c_{2,2}=\frac{1}{3}$\\
				per la formula dei coefficienti di Newton Cotes
			\end{itemize}
			Segue $I_2(f)=\frac{b-a}{2}\left(\frac{1}{3}f(a)+\frac{4}{3}f\left(\frac{a+b}{2}\right)+\frac{1}{2}f(b)\right)=$\\ $=\frac{b-a}{6}\left(f(a)+4f\left(\frac{a+b}{2}\right)+f(b)\right)$.
	\end{itemize}
\end{sol}

\sectionline{black}{88}

\section{Esercizio 3}
\label{sub:es3}
\emph{Verificare, utilizzando il risultato del Teorema (5.2) le (5.9) e (5.10).}
\begin{sol}
	\begin{itemize}
		\item Formula dei trapezi, $n=1$:
			\begin{itemize}
				\item $k=1$,
				\item $\nu_1=\int_0^1{\prod_{j=0}^1{(t-j)}\:dt}=\int_0^1{(t(t-1))\:dt}=\left.\left(\frac{t^3}{3}-\frac{t^2}{2}\right)\right|_0^1=-\frac{1}{6}$.
			\end{itemize}
			Segue $E_1(f)=\nu_1\frac{f^{(2)}(\xi)}{2!}\left(\frac{b-a}{1}\right)^3=-\frac{1}{12}f^{(2)}(\xi)(b-a)^3$.
		\item Formula di Simpson, $n=2$:
			\begin{itemize}
				\item $k=2$,
				\item $\nu_2=\int_0^2{t\prod_{j=0}^2{(t-j)}\:dt}=\int_0^1{(t^2(t-1)(t-2))\:dt}=$\\$=\left.\left(\frac{t^5}{5}-3\frac{t^4}{4}+2\frac{t^3}{3}\right)\right|_0^1=-\frac{4}{15}$.
			\end{itemize}
			Segue $E_2(f)=\nu_2\frac{f^{(4)}(\xi)}{4!}\left(\frac{b-a}{2}\right)^5=-\frac{1}{90}f^{(4)}(\xi)\left(\frac{b-a}{2}\right)^5$.
	\end{itemize}
\end{sol}

\sectionline{black}{88}

\section{Esercizio 4}
\label{sub:es4}
\emph{Scrivere una \lstinline{function} \textsc{Matlab} che implementi efficientemente la formula dei trapezi composita (5.11).}
\begin{sol}
	\lstinputlisting[caption={Formula dei trapezi composita.}, label=lst:trapeziComposita]{code/trapeziComposita.m}
\end{sol}

\sectionline{black}{88}

\section{Esercizio 5}
\label{sub:es5}
\emph{Scrivere una \lstinline{function} \textsc{Matlab} che implementi efficientemente la formula di Simpson composita (5.13).}
\begin{sol}
	\lstinputlisting[caption={Formula di Simpson composita.}, label=lst:simpsonComposita]{code/simpsonComposita.m}
	L'espressione implementata è
	$$I_2^{(n)}(f)=\frac{b-a}{3n}(\sum_{i=1}^{n/2}(4f_{2i-1}+2f_{2i})+f_0-f_n).$$
\end{sol}

\sectionline{black}{88}

\section{Esercizio 6}
\label{sub:es6}
\emph{Implementare efficientemente in \textsc{Matlab} la formula adattativa dei trapezi.}
\begin{sol}
	\lstinputlisting[caption={Formula dei trapezi adattativa.}, label=lst:trapeziAdattativa]{code/trapeziAdattativa.m}
\end{sol}

\sectionline{black}{88}

\section{Esercizio 7}
\label{sub:es7}
\emph{Implementare efficientemente in \textsc{Matlab} la formula adattativa di Simpson.}
\begin{sol}
	\lstinputlisting[caption={Formula di Simpson adattativa.}, label=lst:simpsonAdattativa]{code/simpsonAdattativa.m}
\end{sol}

\sectionline{black}{88}

\section{Esercizio 8}
\label{sub:es8}
\emph{Come è classificabile, dal punto di vista del condizionamento, il seguente problema?
			$$\int_{\frac{1}{2}}^{100}-2x^{-3}\cos\left(x^{-2}\right)\;\mathrm{d}x\equiv\sin\left(10^{-4}\right)-\sin(4)$$}
\begin{sol}
	\lstinputlisting[caption={Esercizio \ref{es:5.8}.}, label=lst:es5.8]{code/es5_8.m}
	\normalfont
	Per la \ref{kappa} risulta $\kappa=b-a=100-1/2=99.5$, il problema è dunque mal condizionato.
	Visto che la funzione ha rapide variazioni in $(1,10)\subset[1/2,100]$ è preferibile
	utilizzare le formule di adattative invece che le formule composite;
\end{sol}

\sectionline{black}{88}

\section{Esercizio 9}
\label{sub:es9}
\emph{Utilizzare le \lstinline{function} degli Esercizi 5.4 e 5.6 per il calcolo dell'integrale
      $$\int_{\frac{1}{2}}^{100}-2x^{-3}\cos\left(x^{-2}\right)\;\mathrm{d}x\equiv\sin\left(10^{-4}\right)-\sin(4),$$
			indicando gli errori commessi.
      Si utilizzi $n=1000,2000,\dots,10000$ per la formula dei trapezi composita e
      $tol=10^{-1},10^{-2},\dots,10^{-5}$ per la formula dei trapezi adattativa (indicando anche il numero di punti).}
\begin{sol}
	\lstinputlisting[caption={Esercizio 5.9.}, label=lst:es5.9]{code/es5_9.m}
	\normalfont
	$ $\\
	\begin{center}\begin{tabular}{c|c|c}
	\hline\multicolumn{3}{c}{Formula composita dei trapezi}\\\hline
	$n$ & $I$ & $E_1^{(n)}$\\\hline
	1000&6.6401e-001&9.2897e-002\\
	2000&7.3077e-001&2.6131e-002\\
	3000&7.4507e-001&1.1836e-002\\
	4000&7.5020e-001&6.7007e-003\\
	5000&7.5260e-001&4.3009e-003\\
	6000&7.5391e-001&2.9914e-003\\
	7000&7.5470e-001&2.1998e-003\\
	8000&7.5522e-001&1.6853e-003\\
	9000&7.5557e-001&1.3321e-003\\
	10000&7.5582e-001&1.0793e-003
	\end{tabular}\end{center}
	\begin{center}
	\begin{tabular}{c|c|c|c}
	\hline\multicolumn{4}{c}{Formula dei trapezi adattativa}\\\hline
	$tol$ & $I$ & $E_1^{(n)}$ & punti\\\hline
		1.0e-001&7.5143e-001&5.4696e-003&159\\
		1.0e-002&7.5563e-001&1.2676e-003&471\\
		1.0e-003&7.5657e-001&3.3005e-004&1567\\
		1.0e-004&7.5684e-001&6.5936e-005&4851
	\end{tabular}\end{center}
\end{sol}

\sectionline{black}{88}

\section{Esercizio 10}
\label{sub:es10}
\emph{Utilizzare le \lstinline{function} degli Esercizi 5.5 e 5.7 per il calcolo dell'integrale
      $$\int_{\frac{1}{2}}^{100}-2x^{-3}\cos\left(x^{-2}\right)\;\mathrm{d}x\equiv\sin\left(10^{-4}\right)-\sin(4),$$
			indicando gli errori commessi.
      Si utilizzi $n=1000,2000,\dots,10000$ per la formula di Simpson composita e
      $tol=10^{-1},10^{-2},\dots,10^{-5}$ per la formula di Simpson adattativa (indicando anche il numero di punti).}
\begin{sol}
	\lstinputlisting[caption={Esercizio 5.10}, label=lst:es5.10]{code/es5_10.m}
	\normalfont
	$ $\\
	\begin{center}\begin{tabular}{c|c|c}
	\hline\multicolumn{3}{c}{Formula composita di Simpson}\\\hline
	$n$ & $I$ & $E_1^{(n)}$\\\hline
	1000 	  &7.0132e-001 	  &5.5580e-002\\
	2000 	  &7.5303e-001 	  &3.8753e-003\\
	3000 	  &7.5617e-001 	  &7.2977e-004\\
	4000 	  &7.5668e-001 	  &2.2403e-004\\
	5000 	  &7.5681e-001 	  &9.0209e-005\\
	6000 	  &7.5686e-001 	  &4.3062e-005\\
	7000 	  &7.5688e-001 	  &2.3094e-005\\
	8000 	  &7.5689e-001 	  &1.3479e-005\\
	9000 	  &7.5689e-001 	  &8.3892e-006\\
	10000 	  &7.5690e-001 	  &5.4921e-006
	\end{tabular}\end{center}\begin{center}
	\begin{tabular}{c|c|c|c}
	\hline\multicolumn{4}{c}{Formula di Simpson adattativa}\\\hline
	$tol$ & $I$ & $E_1^{(n)}$ & punti\\\hline
	1.0e-001 	  &7.5701e-001 	  &1.1164e-004 	  &49\\
	1.0e-002 	  &7.5671e-001 	  &1.9384e-004 	  &65\\
	1.0e-003 	  &7.5690e-001 	  &4.8068e-006 	  &93\\
	1.0e-004 	  &7.5688e-001 	  &1.7808e-005 	  &181\\
	1.0e-005 	  &7.5690e-001 	  &4.8337e-006 	  &309
	\end{tabular}\end{center}
\end{sol}
