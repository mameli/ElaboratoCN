\documentclass[a4paper, 10pt]{book}
\usepackage{amsmath}
\usepackage[italian]{babel}
\usepackage[utf8x]{inputenc}
\usepackage{mathtools}
\usepackage{epigraph}
\usepackage{listings}
\usepackage{amsfonts, mathtools, placeins, floatflt, xfrac, multirow, indentfirst, fancyhdr,
            amssymb, latexsym, amsthm, eucal, float, tabularx, mdframed, bigdelim, blkarray,
            relsize, lettrine, xcolor, tikz, caption, geometry, perpage}




\title{ELABORATO DI CALCOLO NUMERICO}
\author{Federico Schipani \\ Tommaso Ceccarini \\ Giuliano Gambacorta \\}



\begin{document}
\maketitle
\tableofcontents

\cleardoublepage

\thispagestyle{empty}
\vspace{\stretch{1}}
\begin{flushright}
\epigraph{L'errore assoluto da un informazione relativa, l'errore relativo da un informazione assoluta!}{Sestini, Brugnano, Magherini}
\end{flushright}
\vspace{\stretch{2}}
\cleardoublepage
\chapter{Errori ed aritmetica finita}
\section{Esercizi del libro}
Qua ci sono gli esercizi contenuti del libro
\subsection{Esercizio 1.1}
\emph{Sia $ x = \pi \approx 3.1415 = \tilde{x} $. Calcolare il corrispondente errore relativo $\epsilon_{x}$.
Verificare che il numero di cifre decimali corrette nella rappresentazione approssimata di x mediante $\tilde{x}$ è all'incirca dato da}  \\

\center$-\log_{10}|\epsilon_{x}|$
\flushleft \underline{Soluzione:}\\
\center
$ x = \pi \approx 3.1415, \qquad \tilde{x} = 3.1415 $ \\ \vspace{1ex}
$ \Delta_{x} = x - \tilde{x} = 0,0000926$ \\ \vspace{1ex}
$ \epsilon_{x} = \frac{\Delta_{x}}{x} = 0,3 \cdot 10^{-4} $\\ \vspace{1ex}
$ -log_{10} | \epsilon_{x} | = -log_{10} | 0,3 \cdot 10^{-4} | \approx 4,5  $\\ \vspace{1ex}
\flushleft Arrotondando $ 4,5 $ si nota che il numero delle cifre decimali corrette nella rappresentazione è 4

\subsection{Esercizio 1.2}
\emph{Dimostrare che, se $f(x)$ è sufficientemente regolare e $ h> 0 $ è una quantità "piccola", allora:}
\center$\frac{f(x_{0}+h)-f(x_{0}-h)}{2h} = f'(x_{0})+O(h^2)$ \\
\vspace{1em}
$\frac{f(x_{0}+h)-2f(x_{0})+f(x_{0}-h)}{h^2} = f''(x_{0})+O(h^2) $
\flushleft

\subsection{Esercizio 1.3}
\emph{Dimostrare che il metodo iterativo $(1.1) $, convergente a $ x^* \ (vedi(1.2)) $, deve verificare la condizione di consistenza}
$$ x^* =  \Phi(x^*)$$
\emph{Ovvero la soluzione cercata deve essere un \underline{punto fisso} per la funzione di iterazione che definisce il metodo.}

\subsection{Esercizio 1.4}

\emph{Il metodo iterativo}

$$x_{n+1} = \frac{x_{n}x_{n-1}+2}{x_{n}+x_{n-1}}, \qquad n = 1, 2, ..., x_{0} = 2,\qquad x_{1} = 1.5 $$

\emph{definisce una successione di approssimazioni convergente a} $\sqrt{2}$. \emph{Calcolare a quale valore di $ n $ bisogna arrestare l'iterazione, per avere un errore di convergenza}
$\approx 10^{-22}$ \emph{(comparare con i risultati in Tabella 1.1)}

\subsection{Esercizio 1.5}
\emph{Il codice Fortran}
\lstinputlisting[language=Fortran, frame=none, stepnumber=0, nolol=true]{code/fortran1_5.txt}
\emph{Produce il seguente output:}
\begin{enumerate}
\item $32765$
\item $32766$
\item $32767$
\item $-32768$
\item $-32767$
\item $-32766$
\item $-32765$
\item $-32764$
\item $-32763$
\item $-32762$
\end{enumerate}
\emph{Spiegarne il motivo}


\subsection{Esercizio 1.6}
\emph{Dimostrare i teoremi 1.1 e 1.3}
\subsection{Esercizio 1.7}
\emph{Completare la dimostrazione del teorema 1.4}

\subsection{Esercizio 1.8}
\emph{Quante cifre binarie sono utilizzate per rappresentare, mediante arrotondamento, la mantissa di un numero, sapendo che la precisione di machina è $ u \approx 4.66 \cdot 10^{-10} $}

\subsection{Esercizio 1.9}
\emph{Dimostrare che, detta $ u $ la precisione di macchina utilizzata, \\
\center{$-log_{10} u $}
\flushleft fornisce, approssimativamente, il numero di cifre decimali correttamente rappresentate dalla mantissa.}

\subsection{Esercizio 1.10}
Con riferimento allo standard IEEE 754 determinare, relativamente alla doppia precisione:\\
\begin{enumerate}
  	\item il più grande numero di macchina
  	\item il più piccolo numero di macchina normalizzato positivo
	\item il più piccolo numero di macchina denormalizzato positivo
	\item la precisione di macchina
\end{enumerate}
Confrontare le risposte ai primi due quesiti col risultato fornito dalle $\mathtt{function}$ $Matlab$ $\mathtt{realmax}$ e $\mathtt{realmin}$

\subsection{Esercizio 1.11}
\emph{Eseguire le seguenti istruzioni $Matlab$:}

\lstinputlisting[frame=none, stepnumber=0, nolol=true]{code/matlab1_11.m}
\emph{Spiegarne il (non) funzionamento}

\subsection{Esercizio 1.12}
\emph{Individuare l'algoritmo più efficace per calcolare, in aritmetica finita, l'espressione $\sqrt{x^{2} + y^{2}}$}

\subsection{Esercizio 1.13}
\emph{Eseguire le seguenti istruzioni $Matlab$:}

\lstinputlisting[frame=none, stepnumber=0, nolol=true]{code/matlab1_13.m}
\emph{Concludere che la somma algebrica non gode, in aritmetica finita, della proprietà associativa.}

\subsection{Esercizio 1.14}
\emph{Eseguire e discutere il risultato delle seguenti istruzioni $Matlab$}
\emph{\center $(1e300-1e300)*1e300$, \qquad $(1e300*1e300)-(1e300*1e300)$}
\flushleft

\subsection{Esercizio 1.15}
\emph{Eseguire l'analisi dell'errore (relativo), dei due seguenti algoritmi per calcolare la somma di tre numeri:}
\center 1) $ (x\oplus y) \oplus z) $, \qquad 2) $x\oplus(y\oplus z)$
\flushleft

\subsection{Esercizio 1.17}
\emph{(Cancellazione numerica) Si supponga di dover calcolare l'espressione }
\center $ y = 0.12345678-0.12341234 \equiv 0.0000444$,
\flushleft utilizzando una rappresentazione decimale con arrotondamento alla quarta cifra significativa. Comparare il risultato esatto con quello ottenuto in aritmetica finita, e determinare la perdita di cifre significative derivante dalla operazione effettuata. Verificare che questo risultato è in accordo con l'analisi di condizionamento.

\subsection{Esercizio 1.18}
\emph{(Cancellazione Numerica) Eseguire le seguenti istruzioni $Matlab$}

\lstinputlisting[frame=none, stepnumber=0, nolol=true]{code/matlab1_18.m}
\emph{Valutare l'errore relativo sui dati di ingresso e l'errore relativo sul risultato ottenuto.}

\section{Esercizi integrativi}
Questi sono gli esercizi integrativi sul capitolo 1


\subsection{Esercizio 1}
\emph{ un'approssimazione del secondo ordine di $f'(x_{0})$ utilizzando il passo di discretizzazione $ h $ e i seguenti tre valori di funzione $ f(x_{0}), f(x_{0}+h), f(x_{0}+2h) $ (molecola a tre punti in avanti).}

\subsection{Esercizio 2}
\emph{Dimostrare che se un numero reale $x $ viene approssimato da $\tilde{x}$ con un certo errore relativo $\epsilon_{x}$, la quantità $-log_{10}|\epsilon_{x}|$ fornisce approssimativamente il numero di cifre decimali esatte di $\tilde{x}$.}

\subsection{Esercizio 3}
\emph{Calcolare il più grande e il più piccolo numero reale di macchina positivo normalizzato che si può rappresentare utilizzando lo standard IEEE 754 nel formato della singola precisione e in quello della doppia precisione.}\\
\underline{Soluzione:}\\
Il numero più piccolo è uguale a: $$R_{1} = b^{\nu} $$ \\
Per cui nel caso di questo esercizio: $$2^{-126}$$\\
Il numero più grande è uguale a $$R_{2} = (1-2^{-24}) 2^{\varphi} $$ con $$ \varphi = 2^{8}-127 $$ quindi $$ R_{2} = 6.805646933 \cdot 10^{38}$$.

\subsection{Esercizio 4}
\emph{Siano $ x = 2.7352 \cdot 10^{2}, y = 4.8017 \cdot 10^{-2}$ e $z = 3.6152 \cdot 10^{-2}$. Utilizzando un'aritmetica finita che lavora in base 10 con arrotondamento e che riserva $ m = 4 $ cifre alla mantissa, confrontare gli errori assoluti $ R_{1}-R$ e $R_{2}-R$, dove $ R = x+y+z$ e \\}
\center{$R_{1} = (x \oplus y) \oplus z, \qquad  R_{2} = x \oplus ( y \oplus z).$} \\
\flushleft
\underline{Soluzione:}\\
Per facilitare i calcoli si portano $ x $ e $ y $ da $ 10 ^ {-2} $ a $ 10^{2} $:\\
$$ y = 4.8017 \cdot 10^{-2} = 0.00048017 \cdot 10^{2} $$ \\
$$ z = 3.6152 \cdot 10^{-2} = 0.00031652 \cdot 10^{2} $$ \\
Successivamente si calcola $R_{1}$, $R_{2}$ ed $R$:\\
$$ R_{1} = (x \oplus y) \oplus z)  = (2.7352 + 0.00048017) + 0.00031652 = 2,7359 \cdot 10^{2}$$\\
$$ R_{2} = x \oplus ( y \oplus z) = 2.7352 + (0.00048017 + 0.00031652) = 2.7360 \cdot 10^{2} $$ \\
$$ R = x+y+z = 2.73604169 $$ \\
Attraverso R si calcolano gli errori assoluti su $R_{1}$ ed $R_{2}$: \\
$$R_{1}-R = -0.000014169 \cdot 10^{2} $$\\
$$R_{2}-R = 0.000004169 \cdot 10^{2} $$ \\


\subsection{Esercizio 5}
\emph{Un'aritmetica finita utilizza la base 10, l'arrotondamento, $m = 5$ cifre per la mantissa, $s = 2$ cifre per l'esponente e lo shift $\nu = 50$. Per gli interi esso utilizza $N = 7$ cifre decimali. Dire se il numero intero $x = 136726$ è un numero intero di macchina e come viene convertito in reale di macchina. Dire quindi se il numero intero $x = 78345 \cdot 10^{40} $ un reale di macchina e/o se è un intero di macchina.}
\\
\underline{Soluzione:}\\
Si verifica facilmente che il numero $ x = 136726$ è un intero di macchina. \\
Il numero $ x $ viene convertito in reale di macchina in questo modo: $x_{r} = 1,36726 \cdot 10^{5} $\\
Invece il numero $ x = 78345 \cdot 10^{40} $ è un reale di macchina in quanto non bastano 5 cifre per rappresentarlo.\\
Per un maggiore sicurezza si calcola il più grande reale di macchina:\\ $$ R_{2} = (1-10^{-5}) 10^{50} = 9.9999 \cdot 10^{49} $$\\
Essendo $x < R_{2} $ è confermato che è un reale di macchina.


\subsection{Esercizio 6}
\emph{Dimostrare che il numero di condizionamento del problema di calcolo $ \sqrt[n]{x} $ è $\frac{1}{n}$.\\}
\underline{Soluzione:} \\
Per prima cosa si riscrive la funzione:\\ $$f(x) = x^{\frac{1}{n}}$$\\
Il numero di condizionamento è uguale a $$\kappa = \lvert f'(x) \frac{x}{y} \rvert $$\\
Si calcola la derivata di $ f(x) $ che è uguale a \\ $$ f'(x) = \frac{x^ { \frac{1}{n}-1 } }{n}$$\\
Quindi: \\ $$ \kappa = \lvert \frac{x^{\frac{1}{n}-1}}{n} \frac{x}{\sqrt[n]{x} }\rvert$$ \\
Da cui con passaggi algebrici:\\
$$\lvert \frac{\sqrt[n]{x} \hspace{1ex} x^{-1}}{n} \frac{x}{\sqrt[n]{x} } \rvert = \lvert \frac{1}{n} \rvert = \frac{1}{n} $$.

\subsection{Esercizio 7}
\emph{Individuare l'algoritmo più efficace per valutare, in aritmetica finita, la funzione $f(x)=\ln{x^{4}}$\\}
\underline{Soluzione:} \\
Bisogna semplicemente riscrivere la funzione per evitare problemi di overflow e underflow e poi valutarla.
$f(x) = \ln{x^{4}} = 4 \cdot \ln{x}$ \\
L'algoritmo è questo: \\
SCRIVERE ALGORITMO

\subsection{Esercizio 8}
\emph{Individuare una forma algebrica equivalente ma preferibile in aritmetica finita per il calcolo dell'espressione $ (x+2)^{3} - x^{3}$\\}
\underline{Soluzione:}\\
$$(x+2)^{3}-x^{3} = (x+2)(x+2)^{2} - x^{3} = (x+2)(x^{2}+4x+4)-x^{3} = x^{3} +4x^{2}+4x+2x^{2}+8x+8-x^{3} = 6x^{2} +12x+8 $$

\subsection{Esercizio 9}
\emph{Si calcoli l'approssimazione $\tilde{y}$ della differenza tra $ y$ fra $x_{2} = 3.5555$ e $x_{1} = 3.5554$ utilizzando un'aritmetica finita che lavora con arrotondamento in base 10 con 4 cifre per la mantissa normalizzata. Se ne calcoli quindi il corrispondente errore relativo e la maggiorazione di esso che si ottiene utilizzando il numero di condizionamento della somma algebrica.\\}
\underline{Soluzione:}\\
Prima si calcola un'approssimazione di $x_{1}$ e $x_{2}$ poi si calcola un approssimazione $\tilde{y}$ della differenza $y$.
$$\tilde{x_{1}} = 3.555$$\\
$$\tilde{x_{2}} = 3.556$$\\
$$\tilde{y} = \tilde{x_{1}} - \tilde{x_{2}} = 0.001$$\\
$$ y = x_{2}-x_{1} = 0.0001$$\\
L'errore relativo è uguale a: $\varepsilon_{y} = \frac{0.001-0.0001}{0.0001} = 9$\\
Per la maggiorazione serve  $max \{ \lvert \varepsilon_{x_{1}} \rvert, \lvert \varepsilon_{x_{2}} \rvert  \}$\\
$$\varepsilon_{x_{1}} = \frac{\tilde{x_{1}}-x_{1}}{x_{1}} = -1.125 \cdot 10^{-4}$$\\
$$\varepsilon_{x_{2}} = \frac{\tilde{x_{2}}-x_{2}}{x_{2}} = 1.406 \cdot 10^{-4}$$\\ %SOSTITUIRE LA CAZZO DI K
Quindi:\\ $$ k = \frac{ \lvert -3.5554 \rvert + \lvert 3.5555 \rvert}{\lvert 3.5555-3.5554\rvert} \cdot 1.406 \cdot 10^{-4} = 9.9979254 $$

\subsection{Esercizio 10}
\emph{Dimostrare che il numero razionale $ 0.1$  (espresso in base 10) non può essere un numero di macchina in un'aritmetica finita che utilizza la base 2 indipendentemente da come viene fissato il numero $\mathit{m}$ di bit riservati alla mantissa. Dare una maggiorazione del corrispondente errore relativo di rappresentazione supponendo di utilizzare l'aritmetica finita binaria che utilizza l'arrotondamento e assume $ \mathit{m} = 7$.}

\chapter{Errori ed aritmetica finita}
\section{Esercizi del libro}
Qua ci sono gli esercizi contenuti del libro
\subsection{Esercizio 1.1}
\emph{Sia $ x = \pi \approx 3.1415 = \tilde{x} $. Calcolare il corrispondente errore relativo $\epsilon_{x}$.
Verificare che il numero di cifre decimali corrette nella rappresentazione approssimata di x mediante $\tilde{x}$ è all'incirca dato da}  \\

\center$-\log_{10}|\epsilon_{x}|$
\flushleft \underline{Soluzione:}\\
\center
$ x = \pi \approx 3.1415, \qquad \tilde{x} = 3.1415 $ \\ \vspace{1ex}
$ \Delta_{x} = x - \tilde{x} = 0,0000926$ \\ \vspace{1ex}
$ \epsilon_{x} = \frac{\Delta_{x}}{x} = 0,3 \cdot 10^{-4} $\\ \vspace{1ex}
$ -log_{10} | \epsilon_{x} | = -log_{10} | 0,3 \cdot 10^{-4} | \approx 4,5  $\\ \vspace{1ex}
\flushleft Arrotondando $ 4,5 $ si nota che il numero delle cifre decimali corrette nella rappresentazione è 4

\subsection{Esercizio 1.2}
\emph{Dimostrare che, se $f(x)$ è sufficientemente regolare e $ h> 0 $ è una quantità "piccola", allora:}
\center$\frac{f(x_{0}+h)-f(x_{0}-h)}{2h} = f'(x_{0})+O(h^2)$ \\
\vspace{1em}
$\frac{f(x_{0}+h)-2f(x_{0})+f(x_{0}-h)}{h^2} = f''(x_{0})+O(h^2) $
\flushleft

\subsection{Esercizio 1.3}
\emph{Dimostrare che il metodo iterativo $(1.1) $, convergente a $ x^* \ (vedi(1.2)) $, deve verificare la condizione di consistenza}
$$ x^* =  \Phi(x^*)$$
\emph{Ovvero la soluzione cercata deve essere un \underline{punto fisso} per la funzione di iterazione che definisce il metodo.}

\subsection{Esercizio 1.4}

\emph{Il metodo iterativo}

$$x_{n+1} = \frac{x_{n}x_{n-1}+2}{x_{n}+x_{n-1}}, \qquad n = 1, 2, ..., x_{0} = 2,\qquad x_{1} = 1.5 $$

\emph{definisce una successione di approssimazioni convergente a} $\sqrt{2}$. \emph{Calcolare a quale valore di $ n $ bisogna arrestare l'iterazione, per avere un errore di convergenza}
$\approx 10^{-22}$ \emph{(comparare con i risultati in Tabella 1.1)}

\subsection{Esercizio 1.5}
\emph{Il codice Fortran}
\lstinputlisting[language=Fortran, frame=none, stepnumber=0, nolol=true]{code/fortran1_5.txt}
\emph{Produce il seguente output:}
\begin{enumerate}
\item $32765$
\item $32766$
\item $32767$
\item $-32768$
\item $-32767$
\item $-32766$
\item $-32765$
\item $-32764$
\item $-32763$
\item $-32762$
\end{enumerate}
\emph{Spiegarne il motivo}


\subsection{Esercizio 1.6}
\emph{Dimostrare i teoremi 1.1 e 1.3}
\subsection{Esercizio 1.7}
\emph{Completare la dimostrazione del teorema 1.4}

\subsection{Esercizio 1.8}
\emph{Quante cifre binarie sono utilizzate per rappresentare, mediante arrotondamento, la mantissa di un numero, sapendo che la precisione di machina è $ u \approx 4.66 \cdot 10^{-10} $}

\subsection{Esercizio 1.9}
\emph{Dimostrare che, detta $ u $ la precisione di macchina utilizzata, \\
\center{$-log_{10} u $}
\flushleft fornisce, approssimativamente, il numero di cifre decimali correttamente rappresentate dalla mantissa.}

\subsection{Esercizio 1.10}
Con riferimento allo standard IEEE 754 determinare, relativamente alla doppia precisione:\\
\begin{enumerate}
  	\item il più grande numero di macchina
  	\item il più piccolo numero di macchina normalizzato positivo
	\item il più piccolo numero di macchina denormalizzato positivo
	\item la precisione di macchina
\end{enumerate}
Confrontare le risposte ai primi due quesiti col risultato fornito dalle $\mathtt{function}$ $Matlab$ $\mathtt{realmax}$ e $\mathtt{realmin}$

\subsection{Esercizio 1.11}
\emph{Eseguire le seguenti istruzioni $Matlab$:}

\lstinputlisting[frame=none, stepnumber=0, nolol=true]{code/matlab1_11.m}
\emph{Spiegarne il (non) funzionamento}

\subsection{Esercizio 1.12}
\emph{Individuare l'algoritmo più efficace per calcolare, in aritmetica finita, l'espressione $\sqrt{x^{2} + y^{2}}$}

\subsection{Esercizio 1.13}
\emph{Eseguire le seguenti istruzioni $Matlab$:}

\lstinputlisting[frame=none, stepnumber=0, nolol=true]{code/matlab1_13.m}
\emph{Concludere che la somma algebrica non gode, in aritmetica finita, della proprietà associativa.}

\subsection{Esercizio 1.14}
\emph{Eseguire e discutere il risultato delle seguenti istruzioni $Matlab$}
\emph{\center $(1e300-1e300)*1e300$, \qquad $(1e300*1e300)-(1e300*1e300)$}
\flushleft

\subsection{Esercizio 1.15}
\emph{Eseguire l'analisi dell'errore (relativo), dei due seguenti algoritmi per calcolare la somma di tre numeri:}
\center 1) $ (x\oplus y) \oplus z) $, \qquad 2) $x\oplus(y\oplus z)$
\flushleft

\subsection{Esercizio 1.17}
\emph{(Cancellazione numerica) Si supponga di dover calcolare l'espressione }
\center $ y = 0.12345678-0.12341234 \equiv 0.0000444$,
\flushleft utilizzando una rappresentazione decimale con arrotondamento alla quarta cifra significativa. Comparare il risultato esatto con quello ottenuto in aritmetica finita, e determinare la perdita di cifre significative derivante dalla operazione effettuata. Verificare che questo risultato è in accordo con l'analisi di condizionamento.

\subsection{Esercizio 1.18}
\emph{(Cancellazione Numerica) Eseguire le seguenti istruzioni $Matlab$}

\lstinputlisting[frame=none, stepnumber=0, nolol=true]{code/matlab1_18.m}
\emph{Valutare l'errore relativo sui dati di ingresso e l'errore relativo sul risultato ottenuto.}

\section{Esercizi integrativi}
Questi sono gli esercizi integrativi sul capitolo 1


\subsection{Esercizio 1}
\emph{ un'approssimazione del secondo ordine di $f'(x_{0})$ utilizzando il passo di discretizzazione $ h $ e i seguenti tre valori di funzione $ f(x_{0}), f(x_{0}+h), f(x_{0}+2h) $ (molecola a tre punti in avanti).}

\subsection{Esercizio 2}
\emph{Dimostrare che se un numero reale $x $ viene approssimato da $\tilde{x}$ con un certo errore relativo $\epsilon_{x}$, la quantità $-log_{10}|\epsilon_{x}|$ fornisce approssimativamente il numero di cifre decimali esatte di $\tilde{x}$.}

\subsection{Esercizio 3}
\emph{Calcolare il più grande e il più piccolo numero reale di macchina positivo normalizzato che si può rappresentare utilizzando lo standard IEEE 754 nel formato della singola precisione e in quello della doppia precisione.}\\
\underline{Soluzione:}\\
Il numero più piccolo è uguale a: $$R_{1} = b^{\nu} $$ \\
Per cui nel caso di questo esercizio: $$2^{-126}$$\\
Il numero più grande è uguale a $$R_{2} = (1-2^{-24}) 2^{\varphi} $$ con $$ \varphi = 2^{8}-127 $$ quindi $$ R_{2} = 6.805646933 \cdot 10^{38}$$.

\subsection{Esercizio 4}
\emph{Siano $ x = 2.7352 \cdot 10^{2}, y = 4.8017 \cdot 10^{-2}$ e $z = 3.6152 \cdot 10^{-2}$. Utilizzando un'aritmetica finita che lavora in base 10 con arrotondamento e che riserva $ m = 4 $ cifre alla mantissa, confrontare gli errori assoluti $ R_{1}-R$ e $R_{2}-R$, dove $ R = x+y+z$ e \\}
\center{$R_{1} = (x \oplus y) \oplus z, \qquad  R_{2} = x \oplus ( y \oplus z).$} \\
\flushleft
\underline{Soluzione:}\\
Per facilitare i calcoli si portano $ x $ e $ y $ da $ 10 ^ {-2} $ a $ 10^{2} $:\\
$$ y = 4.8017 \cdot 10^{-2} = 0.00048017 \cdot 10^{2} $$ \\
$$ z = 3.6152 \cdot 10^{-2} = 0.00031652 \cdot 10^{2} $$ \\
Successivamente si calcola $R_{1}$, $R_{2}$ ed $R$:\\
$$ R_{1} = (x \oplus y) \oplus z)  = (2.7352 + 0.00048017) + 0.00031652 = 2,7359 \cdot 10^{2}$$\\
$$ R_{2} = x \oplus ( y \oplus z) = 2.7352 + (0.00048017 + 0.00031652) = 2.7360 \cdot 10^{2} $$ \\
$$ R = x+y+z = 2.73604169 $$ \\
Attraverso R si calcolano gli errori assoluti su $R_{1}$ ed $R_{2}$: \\
$$R_{1}-R = -0.000014169 \cdot 10^{2} $$\\
$$R_{2}-R = 0.000004169 \cdot 10^{2} $$ \\


\subsection{Esercizio 5}
\emph{Un'aritmetica finita utilizza la base 10, l'arrotondamento, $m = 5$ cifre per la mantissa, $s = 2$ cifre per l'esponente e lo shift $\nu = 50$. Per gli interi esso utilizza $N = 7$ cifre decimali. Dire se il numero intero $x = 136726$ è un numero intero di macchina e come viene convertito in reale di macchina. Dire quindi se il numero intero $x = 78345 \cdot 10^{40} $ un reale di macchina e/o se è un intero di macchina.}
\\
\underline{Soluzione:}\\
Si verifica facilmente che il numero $ x = 136726$ è un intero di macchina. \\
Il numero $ x $ viene convertito in reale di macchina in questo modo: $x_{r} = 1,36726 \cdot 10^{5} $\\
Invece il numero $ x = 78345 \cdot 10^{40} $ è un reale di macchina in quanto non bastano 5 cifre per rappresentarlo.\\
Per un maggiore sicurezza si calcola il più grande reale di macchina:\\ $$ R_{2} = (1-10^{-5}) 10^{50} = 9.9999 \cdot 10^{49} $$\\
Essendo $x < R_{2} $ è confermato che è un reale di macchina.


\subsection{Esercizio 6}
\emph{Dimostrare che il numero di condizionamento del problema di calcolo $ \sqrt[n]{x} $ è $\frac{1}{n}$.\\}
\underline{Soluzione:} \\
Per prima cosa si riscrive la funzione:\\ $$f(x) = x^{\frac{1}{n}}$$\\
Il numero di condizionamento è uguale a $$\kappa = \lvert f'(x) \frac{x}{y} \rvert $$\\
Si calcola la derivata di $ f(x) $ che è uguale a \\ $$ f'(x) = \frac{x^ { \frac{1}{n}-1 } }{n}$$\\
Quindi: \\ $$ \kappa = \lvert \frac{x^{\frac{1}{n}-1}}{n} \frac{x}{\sqrt[n]{x} }\rvert$$ \\
Da cui con passaggi algebrici:\\
$$\lvert \frac{\sqrt[n]{x} \hspace{1ex} x^{-1}}{n} \frac{x}{\sqrt[n]{x} } \rvert = \lvert \frac{1}{n} \rvert = \frac{1}{n} $$.

\subsection{Esercizio 7}
\emph{Individuare l'algoritmo più efficace per valutare, in aritmetica finita, la funzione $f(x)=\ln{x^{4}}$\\}
\underline{Soluzione:} \\
Bisogna semplicemente riscrivere la funzione per evitare problemi di overflow e underflow e poi valutarla.
$f(x) = \ln{x^{4}} = 4 \cdot \ln{x}$ \\
L'algoritmo è questo: \\
SCRIVERE ALGORITMO

\subsection{Esercizio 8}
\emph{Individuare una forma algebrica equivalente ma preferibile in aritmetica finita per il calcolo dell'espressione $ (x+2)^{3} - x^{3}$\\}
\underline{Soluzione:}\\
$$(x+2)^{3}-x^{3} = (x+2)(x+2)^{2} - x^{3} = (x+2)(x^{2}+4x+4)-x^{3} = x^{3} +4x^{2}+4x+2x^{2}+8x+8-x^{3} = 6x^{2} +12x+8 $$

\subsection{Esercizio 9}
\emph{Si calcoli l'approssimazione $\tilde{y}$ della differenza tra $ y$ fra $x_{2} = 3.5555$ e $x_{1} = 3.5554$ utilizzando un'aritmetica finita che lavora con arrotondamento in base 10 con 4 cifre per la mantissa normalizzata. Se ne calcoli quindi il corrispondente errore relativo e la maggiorazione di esso che si ottiene utilizzando il numero di condizionamento della somma algebrica.\\}
\underline{Soluzione:}\\
Prima si calcola un'approssimazione di $x_{1}$ e $x_{2}$ poi si calcola un approssimazione $\tilde{y}$ della differenza $y$.
$$\tilde{x_{1}} = 3.555$$\\
$$\tilde{x_{2}} = 3.556$$\\
$$\tilde{y} = \tilde{x_{1}} - \tilde{x_{2}} = 0.001$$\\
$$ y = x_{2}-x_{1} = 0.0001$$\\
L'errore relativo è uguale a: $\varepsilon_{y} = \frac{0.001-0.0001}{0.0001} = 9$\\
Per la maggiorazione serve  $max \{ \lvert \varepsilon_{x_{1}} \rvert, \lvert \varepsilon_{x_{2}} \rvert  \}$\\
$$\varepsilon_{x_{1}} = \frac{\tilde{x_{1}}-x_{1}}{x_{1}} = -1.125 \cdot 10^{-4}$$\\
$$\varepsilon_{x_{2}} = \frac{\tilde{x_{2}}-x_{2}}{x_{2}} = 1.406 \cdot 10^{-4}$$\\ %SOSTITUIRE LA CAZZO DI K
Quindi:\\ $$ k = \frac{ \lvert -3.5554 \rvert + \lvert 3.5555 \rvert}{\lvert 3.5555-3.5554\rvert} \cdot 1.406 \cdot 10^{-4} = 9.9979254 $$

\subsection{Esercizio 10}
\emph{Dimostrare che il numero razionale $ 0.1$  (espresso in base 10) non può essere un numero di macchina in un'aritmetica finita che utilizza la base 2 indipendentemente da come viene fissato il numero $\mathit{m}$ di bit riservati alla mantissa. Dare una maggiorazione del corrispondente errore relativo di rappresentazione supponendo di utilizzare l'aritmetica finita binaria che utilizza l'arrotondamento e assume $ \mathit{m} = 7$.}


\chapter{Radici di una equazione}
\section{Esercizi del libro}
\subsection{Esercizio 2.1}
Definire una procedura iterativa basata sul metodo di Newton per determinare $\sqrt{a}$, per un assegnato $a>0$. Costruire una tabella dell'approssimazioni relativa al caso $a=x_{0}=2$ (Comparare con la tabella 1.1)
\subsection{Esercizio 2.2}
Generalizzare il risultato del precedente esercizio, derivando una procedura iterativa basata sul metodo di Newton per determinare $\sqrt[n]{a}$ per un assegnato $a>0$
\subsection{Esercizio 2.3}
In analogia con quanto visto nell'Esercizio 2.1, definire una procedura iterativa basata sul metodo delle secanti per determinare $\sqrt{a}$. Confrontare con l'esercizio 1.4.
\subsection{Esercizio 2.4}
Discutere la convergenza del metodo di Newton, applicato per determinare le radici dell'equazione (2.11) in funzione della scelta del punto iniziale $x_{0}$
\subsection{Esercizio 2.5}
Comparare il metodo di Newton (2.9), il metodo di Newton modificato (2.16) ed il metodo di accellerazione di Aitken (2.17), per approsimare gli zeri delle funzioni:\\
\center $f_{1}(x) = (x-1)^{10}, \qquad f_{2}(x)=(x-1)^{10}e^{x} $
\flushleft per valori decrescenti della tolleranza $\mathtt{tolx}$. Utilizzare, in tutti i casi, il punto iniziale $x_{0}=10$.
\subsection{Esercizio 2.6}
� possibile, nel caso delle funzioni del precedente esercizio utilizzare il metodo di bisezione per determinare lo zero?
\subsection{Esercizio 2.7}
Costruire una tabella in cui si comparano, a partire dallo stesso punto iniziale $\mathtt{x_{0} = 0}$, e per valori decrescenti della tolleranza $\mathtt{tolx}$, il numero di iterazioni richieste per la convergenza dei metodi di Newton, corde e secanti, utilizzati per determinare lo zero della funzione\\
\center $f(x) = x - \cos{x}$
\flushleft
\subsection{Esercizio 2.8}
Completare i confronti del precedente esercizio inserendo quelli con il metodo di bisezione, con intervallo di confidenza iniziale $ [0,1]$.
\subsection{Esercizio 2.9}
Quali controlli introdurreste, negli algoritmi 2.4-2.6, al fine di rendere pi� "robuste" le corrispondenti iterazioni?


\chapter{Radici di una equazione}
\section{Esercizi del libro}
\subsection{Esercizio 2.1}
Definire una procedura iterativa basata sul metodo di Newton per determinare $\sqrt{a}$, per un assegnato $a>0$. Costruire una tabella dell'approssimazioni relativa al caso $a=x_{0}=2$ (Comparare con la tabella 1.1)
\subsection{Esercizio 2.2}
Generalizzare il risultato del precedente esercizio, derivando una procedura iterativa basata sul metodo di Newton per determinare $\sqrt[n]{a}$ per un assegnato $a>0$
\subsection{Esercizio 2.3}
In analogia con quanto visto nell'Esercizio 2.1, definire una procedura iterativa basata sul metodo delle secanti per determinare $\sqrt{a}$. Confrontare con l'esercizio 1.4.
\subsection{Esercizio 2.4}
Discutere la convergenza del metodo di Newton, applicato per determinare le radici dell'equazione (2.11) in funzione della scelta del punto iniziale $x_{0}$
\subsection{Esercizio 2.5}
Comparare il metodo di Newton (2.9), il metodo di Newton modificato (2.16) ed il metodo di accellerazione di Aitken (2.17), per approsimare gli zeri delle funzioni:\\
\center $f_{1}(x) = (x-1)^{10}, \qquad f_{2}(x)=(x-1)^{10}e^{x} $
\flushleft per valori decrescenti della tolleranza $\mathtt{tolx}$. Utilizzare, in tutti i casi, il punto iniziale $x_{0}=10$.
\subsection{Esercizio 2.6}
� possibile, nel caso delle funzioni del precedente esercizio utilizzare il metodo di bisezione per determinare lo zero?
\subsection{Esercizio 2.7}
Costruire una tabella in cui si comparano, a partire dallo stesso punto iniziale $\mathtt{x_{0} = 0}$, e per valori decrescenti della tolleranza $\mathtt{tolx}$, il numero di iterazioni richieste per la convergenza dei metodi di Newton, corde e secanti, utilizzati per determinare lo zero della funzione\\
\center $f(x) = x - \cos{x}$
\flushleft
\subsection{Esercizio 2.8}
Completare i confronti del precedente esercizio inserendo quelli con il metodo di bisezione, con intervallo di confidenza iniziale $ [0,1]$.
\subsection{Esercizio 2.9}
Quali controlli introdurreste, negli algoritmi 2.4-2.6, al fine di rendere pi� "robuste" le corrispondenti iterazioni?

\chapter{Capitolo 3}
\section{Esercizio 3.1}
\emph{Riscrivere gli Algoritmi 3.1-3.4 in modo da controllare che la matrice dei coefficenti sia non singolare.}
\section{Esercizio 3.2}
\emph{Dimostrare che la somma ed il prodotto di matrici triangolari inferiori (superiori), è una matrice triangolare inferiore (superiore).}
\section{Esercizio 3.3}
\emph{Dimostrare che il prodotto di due matrici triangolari inferiori a diagonale unitaria è a sua volta una matrice triangolare inferiore a diagonale unitaria.}
\section{Esercizio 3.4}
\emph{Dimostrare che la matrice inversa di una matrice triangolare inferiore è as ua volta triangolare inferiore. Dimostrare inoltre che, se la matrice ha diagonale unitaria, tale è anche la diagonale della sua inversa.}
\section{Esercizio 3.5}
\emph{Dimostrare i lemmi 3.2 e 3.3.}
\section{Esercizio 3.6}
\emph{Dimostrare che il numero di flop richiesti dall'algoritmo 3.5 è dato da (3.25)}
\section{Esercizio 3.7}
\emph{Scrivere una function matlab che implementi efficientemente l'algoritmo 3.5 per calcolare la fattorizzazione LU di una matrice.}
\section{Esercizio 3.8}
\emph{Scrivere una function Matlab che, avendo in ingresso la matrice A riscritta dall'algoritmo 3.5, ed un vettore x contenente i termini noti del sistema lineare (3.1), ne calcoli efficientemente la soluzione.}
\section{Esercizio 3.9}
\emph{Dimostrare i lemmi 3.4 e 3.5}
\section{Esercizio 3.10}
\emph{Completare la dimostrazione del Teorema 3.6}
\section{Esercizio 3.11}
\emph{Dimostrare che , se A è non singolare, le matrici $A^{T}A$ e $AA^{T}$ sono sdp.}
\section{Esercizio 3.12}
\emph{Dimostrare che se A $\in \mathbb{R}^{m \times n} $ con $ m \geq n = rank(A)$, allora la matrice $A^{T} A$ è sdp.}
\section{Esercizio 3.13}
\emph{Data una matrice A $\in \mathbb{R}^{n \times n} $, dimostrare che essa può essere scritta come
$$A = \frac{1}{2} (A+A^{T})+\frac{1}{2} (A-A^{T}) \equiv A_{s} + A_{a}$$
dove $ A_{s} = A_{s}^{T} $ è detta parte simmetrica di A, mentre $A_{a} = - A^{T}_{a} $ è detta parte asimmetrica di A. Dimostrare inoltre che, dato un generico vettore $ x \in \mathbb{R}^{n}$, risulta} $$ x^{T} Ax = x^{T} A_{s} x $$
 \section{Esercizio 3.14}
\emph{Dimostrare la consistenza delle formule (3.29).}
\section{Esercizio 3.15}
\emph{Dimostrare che il numero di flop richiesti dall'algoritmo 3.6 è dato da 3.30 }
\section{Esercizio 3.16}
\emph{Scrivere una function Matlab che implementi efficientemente l'algoritmo di fattorizzazione $LDL^{T}$ per matrici sdp}
\section{Esercizio 3.17}
\emph{Scrivere una funzione Matlab che, avendo in ingresso la matrice A prodotta dalla precedente funziona, contenente la fattorizzazione $LDL^{T}$ della matrice sdp originaria, ed un vettore di termini noti, x, calcoli efficientemente la soluzione del corrispondente sistema lineare.}
\section{Esercizio 3.18}
\emph{Utilizzare la function dell'esercizio 3.16 per verificare che la matrice}\\
   $$\begin{pmatrix}
      1 & 1 & 1 & 1 \\
      1 & 2 & 2 & 2 \\
      1 & 2 & 1 & 1 \\
      1 & 2 & 1 & 2 \\
   \end{pmatrix}$$
\emph{non è sdp.}
\section{Esercizio 3.19}
\emph{Dimostrare che, al passo i-esimo di eliminazione di Gauss con pivoting parziale, si ha $a_{k_{i}i}^{(i)} \neq 0 $, se A è non singolare.}
\section{Esercizio 3.20}
\emph{Con riferimento alla matrice A definita nella (3.24), qual è la matrice di permutazione P che rende PA fattorizzatile LU? Chi sono, in tal caso, i fattori L e U?}
\section{Esercizio 3.21}
\emph{Scrivere una function Matlab che implementi efficientemente l'algoritmo di fattorizzazione LU con pivoting parziale.}
\section{Esercizio 3.22}
\emph{Scrivere una funzione matlab che, avendo in ingresso la matrice A prodotta dalla precedente function, contenente la fattorizzazione LU della matrice permutata, il vettore p contenente l'informazione relativa alla corrispondente matrice di permutazione, ed un vettore di termini noti, x, calcoli efficientemente la soluzione del corrispondente sistema lineare.}
\section{Esercizio 3.23}
\emph{Costruire alcuni esempi di applicazione delle funzioni degli esercizi 3.21 e 3.22}
\section{Esercizio 3.24}
\emph{Le seguenti istruzioni \textsc{Matlab} sono equivalenti a risolvere il dato sistema $2\times 2$ con l'utilizzo del pivoting e non, rispettivamente. Spiegarne il differente risultato ottenuto. Concludere che l'utilizzo del pivoting migliora, in generale, la prognosi degli errori in aritmetica finita.}
\lstinputlisting[frame=none, stepnumber=0, nolol=true]{code/matlab3_24.m}
\section{Esercizio 3.25}
\emph{Si consideri la seguente matrice \textit{bidiagonale inferiore}}
\[
  A=
  \begin{pmatrix}
    1 & & &\\
    100 & 1 & &\\
    & \ddots & \ddots &\\
    & & 100 & 1
  \end{pmatrix}_{10\times 10}.
\]
\emph{Calcolare $k_{\infty}(A)$. Confrontate il risultato con quello fornito dalla \lstinline{function cond} di \textsc{Matlab}. Dimostrare, e verificare, che $k_{\infty}(A)=k_1(A)$.}
\section{Esercizio 3.26}
\emph{Si consideri i seguenti vettori di $\mathbb{R}^{10}$,}
\[
  \underline{b}=
  \begin{pmatrix}
    1\\
    101\\
    \vdots\\
    101
  \end{pmatrix},
  \quad \underline{c}=0.1\cdot
  \begin{pmatrix}
    1\\
    101\\
    \vdots\\
    101
  \end{pmatrix},
\]
\emph{ed i seguenti sistemi lineari
$$A\underline{x}=\underline{b},\qquad A\underline{y}=\underline{c},$$
in cui $A$ è la matrice definita nel precedente Esercizio \ref{es:3.25}. Verificare che le soluzioni di questi sistemi lineari sono, rispettivamente, date da:}
\[
  \underline{x}=
  \begin{pmatrix}
    1\\
    \vdots\\
    1
  \end{pmatrix},
  \qquad \underline{y}=
  \begin{pmatrix}
    0.1\\
    \vdots\\
    0.1
  \end{pmatrix}.
\]
\emph{Confrontare questi vettori con quelli calcolato dalle seguenti due serie di istruzioni \textsc{Matlab},}
\lstinputlisting[frame=none, stepnumber=0, nolol=true]{code/matlab3_26.m}
\emph{che implementano, rispettivamente, le risoluzioni dei due sistemi lineari. Spiegare i risultati ottenuti, alla luce di quanto visto in Sezione 3.7.}
\section{Esercizio 3.27}
\emph{Dimostrare che il numero di \texttt{flop} richiesti dall'Algoritmo di fattorizzazione $QR$ di Householder è dato da 3.60.}
\section{Esercizio 3.28}
\emph{Definendo il vettore $\underline{\hat{v}}=\frac{\underline{v}}{v_1}$, verificare che \lstinline{beta}, come definito nel seguente algoritmo per la fattorizzazione $QR$ di Householder, corrisponde alla quantità $\frac{2}{\underline{\hat{v}}^T\underline{\hat{v}}}$}
\section{Esercizio 3.29}
\emph{Scrivere una \lstinline{function} \textsc{Matlab} che implementi efficientemente l'algoritmo di fattorizzazione $QR$, mediante il metodo di Householder (vedi l'Algoritmo descritto nell'Esercizio precedente).}
\section{Esercizio 3.30}
\emph{Scrivere una \lstinline{function} \textsc{Matlab} che, avendo in ingresso la matrice $A$ prodotta dalla \lstinline{function} del precedente Esercizio, contenente la fattorizzazione $QR$ della matrice originaria, e un corrispondente vettore di termini noti $\underline{b}$, calcoli efficientemente la soluzione del sistema lineare sovradeterminato.}
\section{Esercizio 3.31}
\emph{Utilizzare le \lstinline{function} degli Esercizi 3.29 e 3.30 per calcolare la soluzione ai minimi quadrati di (3.51), ed il corrispondente residuo, nel caso in cui}
\[
  A=
  \begin{pmatrix}
    3 & 2 & 1\\
    1 & 2 & 3\\
    1 & 2 & 1\\
    2 & 1 & 2
  \end{pmatrix},
  \qquad \underline{b}=
  \begin{pmatrix}
    10\\
    10\\
    10\\
    10
  \end{pmatrix}.
\]
\section{Esercizio 3.32}
\emph{Calcolare i coefficienti della equazione della retta
$$r(x)=a_1x+a_2,$$
che meglio approssima i dati prodotti dalle seguenti istruzioni \textsc{Matlab}:}
\lstinputlisting[frame=none, stepnumber=0, nolol=true]{code/matlab3_32a.m}
\emph{Riformulare il problema come minimizzazione della norma Euclidea di un corrispondente vettore residuo. Calcolare la soluzione utilizzando le \lstinline{function} sviluppate negli Esercizi 3.29 e 3.30, e confrontarla con quella ottenuta dalle seguenti istruuzioni \textsc{Matlab}:}
\lstinputlisting[frame=none, stepnumber=0, nolol=true]{code/matlab3_32b.m}
\emph{Confrontare i risultati che si ottengono per i seguenti valori del parametro \lstinline{gamma}:
$$0.5,0.1,0.05,0.01,0.005,0.001.$$
Quale è la soluzione che si ottiene nel limite \lstinline{gamma} $\rightarrow 0$?}
\section{Esercizio 3.33}
\emph{Determinare il punto di minimo della funzione $f(x_1,x_2)=x_1^4+x_1(x_1+x_2)+(1-x_2)^2$, utilizzando il metodo di Newton (3.63) per calcolarne il punto stazionario.}
\section{Esercizio 3.34}
\emph{Uno dei metodi di base per la risoluzione di equazioni differenziali ordinarie, $y'(t)=f(t,y(t))$, $t\in[t_0,T]$, $y(t_0)=y_0$ (problema di Cauchy di prim'ordine), è il metodo di Eulero implicito,
$$y_n=y_{n-1}+hf_n,\qquad n=1,2,\dots,N\equiv\frac{T-t_0}{h},$$
in cui $y_n\approx y(t_n)$, $f_n=f(t_n,y_n)$, $t_n=t_0+nh$. Utilizzare questo metodo nel caso in cui $t_0=0$, $T=10$, $N=100$, $y_0=(1,2)^T$ e, se $y=(x_1,x_2)^T$, $f(t,y)=(-10^3x_1+\sin x_1\cos x_2, -2x_2+\sin x_1 \cos x_2)^T$. Utilizzare il metodo (3.64) per la risoluzione dei sistemi nonlineari richiesti.}

\chapter{Capitolo 3}
\section{Esercizio 3.1}
\emph{Riscrivere gli Algoritmi 3.1-3.4 in modo da controllare che la matrice dei coefficenti sia non singolare.}
\section{Esercizio 3.2}
\emph{Dimostrare che la somma ed il prodotto di matrici triangolari inferiori (superiori), è una matrice triangolare inferiore (superiore).}
\section{Esercizio 3.3}
\emph{Dimostrare che il prodotto di due matrici triangolari inferiori a diagonale unitaria è a sua volta una matrice triangolare inferiore a diagonale unitaria.}
\section{Esercizio 3.4}
\emph{Dimostrare che la matrice inversa di una matrice triangolare inferiore è as ua volta triangolare inferiore. Dimostrare inoltre che, se la matrice ha diagonale unitaria, tale è anche la diagonale della sua inversa.}
\section{Esercizio 3.5}
\emph{Dimostrare i lemmi 3.2 e 3.3.}
\section{Esercizio 3.6}
\emph{Dimostrare che il numero di flop richiesti dall'algoritmo 3.5 è dato da (3.25)}
\section{Esercizio 3.7}
\emph{Scrivere una function matlab che implementi efficientemente l'algoritmo 3.5 per calcolare la fattorizzazione LU di una matrice.}
\section{Esercizio 3.8}
\emph{Scrivere una function Matlab che, avendo in ingresso la matrice A riscritta dall'algoritmo 3.5, ed un vettore x contenente i termini noti del sistema lineare (3.1), ne calcoli efficientemente la soluzione.}
\section{Esercizio 3.9}
\emph{Dimostrare i lemmi 3.4 e 3.5}
\section{Esercizio 3.10}
\emph{Completare la dimostrazione del Teorema 3.6}
\section{Esercizio 3.11}
\emph{Dimostrare che , se A è non singolare, le matrici $A^{T}A$ e $AA^{T}$ sono sdp.}
\section{Esercizio 3.12}
\emph{Dimostrare che se A $\in \mathbb{R}^{m \times n} $ con $ m \geq n = rank(A)$, allora la matrice $A^{T} A$ è sdp.}
\section{Esercizio 3.13}
\emph{Data una matrice A $\in \mathbb{R}^{n \times n} $, dimostrare che essa può essere scritta come
$$A = \frac{1}{2} (A+A^{T})+\frac{1}{2} (A-A^{T}) \equiv A_{s} + A_{a}$$
dove $ A_{s} = A_{s}^{T} $ è detta parte simmetrica di A, mentre $A_{a} = - A^{T}_{a} $ è detta parte asimmetrica di A. Dimostrare inoltre che, dato un generico vettore $ x \in \mathbb{R}^{n}$, risulta} $$ x^{T} Ax = x^{T} A_{s} x $$
 \section{Esercizio 3.14}
\emph{Dimostrare la consistenza delle formule (3.29).}
\section{Esercizio 3.15}
\emph{Dimostrare che il numero di flop richiesti dall'algoritmo 3.6 è dato da 3.30 }
\section{Esercizio 3.16}
\emph{Scrivere una function Matlab che implementi efficientemente l'algoritmo di fattorizzazione $LDL^{T}$ per matrici sdp}
\section{Esercizio 3.17}
\emph{Scrivere una funzione Matlab che, avendo in ingresso la matrice A prodotta dalla precedente funziona, contenente la fattorizzazione $LDL^{T}$ della matrice sdp originaria, ed un vettore di termini noti, x, calcoli efficientemente la soluzione del corrispondente sistema lineare.}
\section{Esercizio 3.18}
\emph{Utilizzare la function dell'esercizio 3.16 per verificare che la matrice}\\
   $$\begin{pmatrix}
      1 & 1 & 1 & 1 \\
      1 & 2 & 2 & 2 \\
      1 & 2 & 1 & 1 \\
      1 & 2 & 1 & 2 \\
   \end{pmatrix}$$
\emph{non è sdp.}
\section{Esercizio 3.19}
\emph{Dimostrare che, al passo i-esimo di eliminazione di Gauss con pivoting parziale, si ha $a_{k_{i}i}^{(i)} \neq 0 $, se A è non singolare.}
\section{Esercizio 3.20}
\emph{Con riferimento alla matrice A definita nella (3.24), qual è la matrice di permutazione P che rende PA fattorizzatile LU? Chi sono, in tal caso, i fattori L e U?}
\section{Esercizio 3.21}
\emph{Scrivere una function Matlab che implementi efficientemente l'algoritmo di fattorizzazione LU con pivoting parziale.}
\section{Esercizio 3.22}
\emph{Scrivere una funzione matlab che, avendo in ingresso la matrice A prodotta dalla precedente function, contenente la fattorizzazione LU della matrice permutata, il vettore p contenente l'informazione relativa alla corrispondente matrice di permutazione, ed un vettore di termini noti, x, calcoli efficientemente la soluzione del corrispondente sistema lineare.}
\section{Esercizio 3.23}
\emph{Costruire alcuni esempi di applicazione delle funzioni degli esercizi 3.21 e 3.22}
\section{Esercizio 3.24}
\emph{Le seguenti istruzioni \textsc{Matlab} sono equivalenti a risolvere il dato sistema $2\times 2$ con l'utilizzo del pivoting e non, rispettivamente. Spiegarne il differente risultato ottenuto. Concludere che l'utilizzo del pivoting migliora, in generale, la prognosi degli errori in aritmetica finita.}
\lstinputlisting[frame=none, stepnumber=0, nolol=true]{code/matlab3_24.m}
\section{Esercizio 3.25}
\emph{Si consideri la seguente matrice \textit{bidiagonale inferiore}}
\[
  A=
  \begin{pmatrix}
    1 & & &\\
    100 & 1 & &\\
    & \ddots & \ddots &\\
    & & 100 & 1
  \end{pmatrix}_{10\times 10}.
\]
\emph{Calcolare $k_{\infty}(A)$. Confrontate il risultato con quello fornito dalla \lstinline{function cond} di \textsc{Matlab}. Dimostrare, e verificare, che $k_{\infty}(A)=k_1(A)$.}
\section{Esercizio 3.26}
\emph{Si consideri i seguenti vettori di $\mathbb{R}^{10}$,}
\[
  \underline{b}=
  \begin{pmatrix}
    1\\
    101\\
    \vdots\\
    101
  \end{pmatrix},
  \quad \underline{c}=0.1\cdot
  \begin{pmatrix}
    1\\
    101\\
    \vdots\\
    101
  \end{pmatrix},
\]
\emph{ed i seguenti sistemi lineari
$$A\underline{x}=\underline{b},\qquad A\underline{y}=\underline{c},$$
in cui $A$ è la matrice definita nel precedente Esercizio \ref{es:3.25}. Verificare che le soluzioni di questi sistemi lineari sono, rispettivamente, date da:}
\[
  \underline{x}=
  \begin{pmatrix}
    1\\
    \vdots\\
    1
  \end{pmatrix},
  \qquad \underline{y}=
  \begin{pmatrix}
    0.1\\
    \vdots\\
    0.1
  \end{pmatrix}.
\]
\emph{Confrontare questi vettori con quelli calcolato dalle seguenti due serie di istruzioni \textsc{Matlab},}
\lstinputlisting[frame=none, stepnumber=0, nolol=true]{code/matlab3_26.m}
\emph{che implementano, rispettivamente, le risoluzioni dei due sistemi lineari. Spiegare i risultati ottenuti, alla luce di quanto visto in Sezione 3.7.}
\section{Esercizio 3.27}
\emph{Dimostrare che il numero di \texttt{flop} richiesti dall'Algoritmo di fattorizzazione $QR$ di Householder è dato da 3.60.}
\section{Esercizio 3.28}
\emph{Definendo il vettore $\underline{\hat{v}}=\frac{\underline{v}}{v_1}$, verificare che \lstinline{beta}, come definito nel seguente algoritmo per la fattorizzazione $QR$ di Householder, corrisponde alla quantità $\frac{2}{\underline{\hat{v}}^T\underline{\hat{v}}}$}
\section{Esercizio 3.29}
\emph{Scrivere una \lstinline{function} \textsc{Matlab} che implementi efficientemente l'algoritmo di fattorizzazione $QR$, mediante il metodo di Householder (vedi l'Algoritmo descritto nell'Esercizio precedente).}
\section{Esercizio 3.30}
\emph{Scrivere una \lstinline{function} \textsc{Matlab} che, avendo in ingresso la matrice $A$ prodotta dalla \lstinline{function} del precedente Esercizio, contenente la fattorizzazione $QR$ della matrice originaria, e un corrispondente vettore di termini noti $\underline{b}$, calcoli efficientemente la soluzione del sistema lineare sovradeterminato.}
\section{Esercizio 3.31}
\emph{Utilizzare le \lstinline{function} degli Esercizi 3.29 e 3.30 per calcolare la soluzione ai minimi quadrati di (3.51), ed il corrispondente residuo, nel caso in cui}
\[
  A=
  \begin{pmatrix}
    3 & 2 & 1\\
    1 & 2 & 3\\
    1 & 2 & 1\\
    2 & 1 & 2
  \end{pmatrix},
  \qquad \underline{b}=
  \begin{pmatrix}
    10\\
    10\\
    10\\
    10
  \end{pmatrix}.
\]
\section{Esercizio 3.32}
\emph{Calcolare i coefficienti della equazione della retta
$$r(x)=a_1x+a_2,$$
che meglio approssima i dati prodotti dalle seguenti istruzioni \textsc{Matlab}:}
\lstinputlisting[frame=none, stepnumber=0, nolol=true]{code/matlab3_32a.m}
\emph{Riformulare il problema come minimizzazione della norma Euclidea di un corrispondente vettore residuo. Calcolare la soluzione utilizzando le \lstinline{function} sviluppate negli Esercizi 3.29 e 3.30, e confrontarla con quella ottenuta dalle seguenti istruuzioni \textsc{Matlab}:}
\lstinputlisting[frame=none, stepnumber=0, nolol=true]{code/matlab3_32b.m}
\emph{Confrontare i risultati che si ottengono per i seguenti valori del parametro \lstinline{gamma}:
$$0.5,0.1,0.05,0.01,0.005,0.001.$$
Quale è la soluzione che si ottiene nel limite \lstinline{gamma} $\rightarrow 0$?}
\section{Esercizio 3.33}
\emph{Determinare il punto di minimo della funzione $f(x_1,x_2)=x_1^4+x_1(x_1+x_2)+(1-x_2)^2$, utilizzando il metodo di Newton (3.63) per calcolarne il punto stazionario.}
\section{Esercizio 3.34}
\emph{Uno dei metodi di base per la risoluzione di equazioni differenziali ordinarie, $y'(t)=f(t,y(t))$, $t\in[t_0,T]$, $y(t_0)=y_0$ (problema di Cauchy di prim'ordine), è il metodo di Eulero implicito,
$$y_n=y_{n-1}+hf_n,\qquad n=1,2,\dots,N\equiv\frac{T-t_0}{h},$$
in cui $y_n\approx y(t_n)$, $f_n=f(t_n,y_n)$, $t_n=t_0+nh$. Utilizzare questo metodo nel caso in cui $t_0=0$, $T=10$, $N=100$, $y_0=(1,2)^T$ e, se $y=(x_1,x_2)^T$, $f(t,y)=(-10^3x_1+\sin x_1\cos x_2, -2x_2+\sin x_1 \cos x_2)^T$. Utilizzare il metodo (3.64) per la risoluzione dei sistemi nonlineari richiesti.}

\chapter{Capitolo 4. Approssimazione di funzioni}
\section{Esercizi del libro}
\subsection{Esercizio 1}
Sia $f(x) = 4x^{2} - 12x +1$. Determinare $p(x) \in \Pi_{4} $ che interpola $f(x)$ sulle ascisse $x_{i} = i, i = 0, \ldots 4$.
\subsection{Esercizio 2} Dimostrare che il seguente algoritmo, \\
SCRIVERE ALGORITMO\\
valuta il polinomio (4.4) nel punto x, se il vettore $a$ contiene i coefficienti del polinomio $p(x)$ (Osservare che in Matlab i vettori hanno indice che parte da 1, invece che da 0).
\subsection{Esercizio 3}
Dimostrare il lemma 4.1
\subsection{Esercizio 4}
Dimostrare il lemma 4.2
\subsection{Esercizio 5}
Dimostrare il lemma 4.4
\subsection{Esercizio 6}
Costruire una function Matlab che implementi in modo efficiente l'Algoritmo 4.1
\subsection{Esercizio 7}
Dimostrare che il seguente algoritmo, che riceve in ingresso i vettori $x$ e $f$ prodotti dalla function dell'Esercizio 4.6, valuta il corrispondente polinomio interpolante di Newton in un punto $xx$
assegnato.\\
SCRIVERE ALGORITMO\\
Quale è il suo costo computazionale? Confrontarlo con quello dell'Algoritmo 4.1. Costruire, quindi, una corrispondente function Matlab che lo implementi efficientemente (complementare la possibilità che $xx$ sia un vettore)
\subsection{Esercizio 8}
Costruire una funciton Matlab che implementi efficientemente l'Algoritmo 4.2.
\chapter{Capitolo 4. Approssimazione di funzioni}
\section{Esercizi del libro}
\subsection{Esercizio 1}
Sia $f(x) = 4x^{2} - 12x +1$. Determinare $p(x) \in \Pi_{4} $ che interpola $f(x)$ sulle ascisse $x_{i} = i, i = 0, \ldots 4$.
\subsection{Esercizio 2} Dimostrare che il seguente algoritmo, \\
SCRIVERE ALGORITMO\\
valuta il polinomio (4.4) nel punto x, se il vettore $a$ contiene i coefficienti del polinomio $p(x)$ (Osservare che in Matlab i vettori hanno indice che parte da 1, invece che da 0).
\subsection{Esercizio 3}
Dimostrare il lemma 4.1
\subsection{Esercizio 4}
Dimostrare il lemma 4.2
\subsection{Esercizio 5}
Dimostrare il lemma 4.4
\subsection{Esercizio 6}
Costruire una function Matlab che implementi in modo efficiente l'Algoritmo 4.1
\subsection{Esercizio 7}
Dimostrare che il seguente algoritmo, che riceve in ingresso i vettori $x$ e $f$ prodotti dalla function dell'Esercizio 4.6, valuta il corrispondente polinomio interpolante di Newton in un punto $xx$
assegnato.\\
SCRIVERE ALGORITMO\\
Quale è il suo costo computazionale? Confrontarlo con quello dell'Algoritmo 4.1. Costruire, quindi, una corrispondente function Matlab che lo implementi efficientemente (complementare la possibilità che $xx$ sia un vettore)
\subsection{Esercizio 8}
Costruire una funciton Matlab che implementi efficientemente l'Algoritmo 4.2.
\chapter{Capitolo 5. Formule di quadratura}
\label{chap:Capitolo 5. Formule di quadratura}

\subsection{Esercizio 1}
\label{sub:es1}
\emph{Calcolare il numero di condizionamento dell'integrale
$$\int_0^{e^{21}}\sin\sqrt{x}\;dx.$$
Questo problema è ben condizionato o è malcondizionato?}


\subsection{Esercizio 2}
\label{sub:es2}
\emph{Derivare, dalla (5.5), i coefficienti della formula dei trapezi (5.6) e della formula di Simpson (5.7).}

\subsection{Esercizio 3}
\label{sub:es3}
\emph{Verificare, utilizzando il risultato del Teorema (5.2) le (5.9) e (5.10).}

\subsection{Esercizio 4}
\label{sub:es4}
\emph{Scrivere una \lstinline{function} \textsc{Matlab} che implementi efficientemente la formula dei trapezi composita (5.11).}

\subsection{Esercizio 5}
\label{sub:es5}
\emph{Scrivere una \lstinline{function} \textsc{Matlab} che implementi efficientemente la formula di Simpson composita (5.13).}

\subsection{Esercizio 6}
\label{sub:es6}
\emph{Implementare efficientemente in \textsc{Matlab} la formula adattativa dei trapezi.}

\subsection{Esercizio 7}
\label{sub:es7}
\emph{Implementare efficientemente in \textsc{Matlab} la formula adattativa di Simpson.}

\subsection{Esercizio 8}
\label{sub:es8}
\emph{Come è classificabile, dal punto di vista del condizionamento, il seguente problema?
			$$\int_{\frac{1}{2}}^{100}-2x^{-3}\cos\left(x^{-2}\right)\;\mathrm{d}x\equiv\sin\left(10^{-4}\right)-\sin(4)$$}

\subsection{Esercizio 9}
\label{sub:es9}
\emph{Utilizzare le \lstinline{function} degli Esercizi 5.4 e 5.6 per il calcolo dell'integrale
      $$\int_{\frac{1}{2}}^{100}-2x^{-3}\cos\left(x^{-2}\right)\;\mathrm{d}x\equiv\sin\left(10^{-4}\right)-\sin(4),$$
			indicando gli errori commessi.
      Si utilizzi $n=1000,2000,\dots,10000$ per la formula dei trapezi composita e
      $tol=10^{-1},10^{-2},\dots,10^{-5}$ per la formula dei trapezi adattativa (indicando anche il numero di punti).}

\subsection{Esercizio 10}
\label{sub:es10}
\emph{Utilizzare le \lstinline{function} degli Esercizi 5.5 e 5.7 per il calcolo dell'integrale
      $$\int_{\frac{1}{2}}^{100}-2x^{-3}\cos\left(x^{-2}\right)\;\mathrm{d}x\equiv\sin\left(10^{-4}\right)-\sin(4),$$
			indicando gli errori commessi.
      Si utilizzi $n=1000,2000,\dots,10000$ per la formula di Simpson composita e
      $tol=10^{-1},10^{-2},\dots,10^{-5}$ per la formula di Simpson adattativa (indicando anche il numero di punti).}

\chapter{Capitolo 5. Formule di quadratura}
\label{chap:Capitolo 5. Formule di quadratura}

\subsection{Esercizio 1}
\label{sub:es1}
\emph{Calcolare il numero di condizionamento dell'integrale
$$\int_0^{e^{21}}\sin\sqrt{x}\;dx.$$
Questo problema è ben condizionato o è malcondizionato?}


\subsection{Esercizio 2}
\label{sub:es2}
\emph{Derivare, dalla (5.5), i coefficienti della formula dei trapezi (5.6) e della formula di Simpson (5.7).}

\subsection{Esercizio 3}
\label{sub:es3}
\emph{Verificare, utilizzando il risultato del Teorema (5.2) le (5.9) e (5.10).}

\subsection{Esercizio 4}
\label{sub:es4}
\emph{Scrivere una \lstinline{function} \textsc{Matlab} che implementi efficientemente la formula dei trapezi composita (5.11).}

\subsection{Esercizio 5}
\label{sub:es5}
\emph{Scrivere una \lstinline{function} \textsc{Matlab} che implementi efficientemente la formula di Simpson composita (5.13).}

\subsection{Esercizio 6}
\label{sub:es6}
\emph{Implementare efficientemente in \textsc{Matlab} la formula adattativa dei trapezi.}

\subsection{Esercizio 7}
\label{sub:es7}
\emph{Implementare efficientemente in \textsc{Matlab} la formula adattativa di Simpson.}

\subsection{Esercizio 8}
\label{sub:es8}
\emph{Come è classificabile, dal punto di vista del condizionamento, il seguente problema?
			$$\int_{\frac{1}{2}}^{100}-2x^{-3}\cos\left(x^{-2}\right)\;\mathrm{d}x\equiv\sin\left(10^{-4}\right)-\sin(4)$$}

\subsection{Esercizio 9}
\label{sub:es9}
\emph{Utilizzare le \lstinline{function} degli Esercizi 5.4 e 5.6 per il calcolo dell'integrale
      $$\int_{\frac{1}{2}}^{100}-2x^{-3}\cos\left(x^{-2}\right)\;\mathrm{d}x\equiv\sin\left(10^{-4}\right)-\sin(4),$$
			indicando gli errori commessi.
      Si utilizzi $n=1000,2000,\dots,10000$ per la formula dei trapezi composita e
      $tol=10^{-1},10^{-2},\dots,10^{-5}$ per la formula dei trapezi adattativa (indicando anche il numero di punti).}

\subsection{Esercizio 10}
\label{sub:es10}
\emph{Utilizzare le \lstinline{function} degli Esercizi 5.5 e 5.7 per il calcolo dell'integrale
      $$\int_{\frac{1}{2}}^{100}-2x^{-3}\cos\left(x^{-2}\right)\;\mathrm{d}x\equiv\sin\left(10^{-4}\right)-\sin(4),$$
			indicando gli errori commessi.
      Si utilizzi $n=1000,2000,\dots,10000$ per la formula di Simpson composita e
      $tol=10^{-1},10^{-2},\dots,10^{-5}$ per la formula di Simpson adattativa (indicando anche il numero di punti).}

\chapter{Calcolo del Google Pagerank}
\label{chap:Google}

\section{Esercizio 1}
\label{sub:Es1}
[Teorema di Gershgorin]
      Dimostrare che gli autovalori di una matrice
      $A=(a_{ij})\in\mathbb{C}^{n n}$
      \emph{sono contenuti nell'insieme}
			\[
				\mathcal{D}=\bigcup_{i=1}^n\mathcal{D}_i,\qquad \mathcal{D}_i=\left\{\lambda\in\mathbb{C}:|\lambda-a_{ii}|\leq\sum_{\begin{subarray}{c}
					j=1\\
					j\neq i
				\end{subarray}}^n|a_{ij}|\right\},\quad i=1,\dots,n.
			\]
\begin{sol}
   Sia $\lambda\in\sigma{A}$ ed $\underline{x}$ il corrispondente autovettore ($\underline{x}\neq\underline{0}$
   e $A\underline{x}=\lambda\underline{x}$) quindi $\underline{e}_i^TA\underline{x}=\lambda\underline{e}_i^T\underline{x}$
   per $i=1,\ldots,n$ cioè $\sum_{j=1}^n{a_{i,j}x_j}=\lambda x_i$;\\
   posto $i$ tale che $|x_i|\geq|x_j|$ ($x_i\neq 0$)
   risulta $$\lambda=\frac{\sum_{j=1}^n{a_{i,j}x_j}}{x_i}=a_{i,i}+\sum_{j=1\: j\neq i}^n{a_{i,j}\frac{x_j}{x_i}}$$ ovvero
   $\lambda-a_{i,i}=\sum_{j=1\: j\neq i}^n{a_{i,j}\frac{x_j}{x_i}}$.\\
   Mettendo poi i valori assuluti
   $$\left|\lambda-a_{i,i}\right|=\left|\sum_{j=1\: j\neq i}^n{a_{i,j}\frac{x_j}{x_i}}\right|\leq\sum_{j=1\: j\neq i}^n{\left|a_{i,j}\frac{x_j}{x_i}\right|}\leq\sum_{j=1\: j\neq i}^n{\left|a_{i,j}\right|}$$ poiché $\left|\frac{x_j}{x_i}\right|\leq 1$
   in quanto $|x_i|\geq|x_j|$. Segue $\lambda\in\bigcup_{i=1}^{n}{\mathcal{D}_i}$ da cui $\sigma{A}\subseteq\mathcal{D}$.
\end{sol}

\sectionline{black}{88}

\section{Esercizio 2}
\label{sub:Es2}
\emph{
      Utilizzare il metodo delle potenze per approssimare l'autovalore dominante della matrice
			\[
				A_n=\begin{pmatrix}
					2 & -1 & &\\
					-1 & 2 & \ddots &\\
					& \ddots & \ddots & -1\\
					& & -1 & 2
				\end{pmatrix}\in\mathbb{R}^{n\times n},
			\]
			per valori crescenti di $n$. Verificare numericamente che questo è dato da $2\left(1+\cos\frac{\pi}{n+1}\right)$.
}
\begin{sol}
  \normalfont
  $A_n$ è una M-matrice,in quanto può essere scritta come : $A_n=2(I_n-B_n)$ dove
  $$B_n=\begin{pmatrix}0&1/2&&\\1/2&0&\ddots&\\&\ddots&\ddots&1/2\\&&1/2&0\end{pmatrix}\in\mathbb{R}^{n\times n} \quad I_n=\begin{pmatrix}2&0&&\\0&2&\ddots&\\&\ddots&\ddots&0\\&&0&2\end{pmatrix}\in\mathbb{R}^{n\times n}.$$
  Risulta $\lambda_j(B_n)=\cos{\frac{j\pi}{n+1}}$, $|\lambda_j|\leq 1$, per $j=1,\ldots,n$;
  segue $\lambda_j(A_n)=2(\lambda_j(I_n)-\lambda_j(B_n))=2(1-\cos{\frac{j\pi}{n+1}})$; il massimo è $\lambda=2(1+\cos{\frac{\pi}{n+1}})$.
  La verifica numerica di questo risultato è riportata nelle seguenti tabelle.
  Codice:\\
  \lstinputlisting[caption={Esercizio 6.2}, label=lst:es6.2]{code/es6_2.m}
  \vspace{0.5em}
  \begin{center}\begin{tabular}{c|c|c|c}
    \hline\multicolumn{4}{c}{Tolleranza $tol=10^{-5}$}\\\hline
    $n$ & $\lambda_1$ & $2\left(1+\cos{\frac{\pi}{n+1}}\right)$ & scostamento\\\hline
    10 & 3.9189 & 3.9190 & 0.0001 \\
    15 & 3.9614 & 3.9616 & 0.0002 \\
    20 & 3.9774 & 3.9777 & 0.0003 \\
    25 & 3.9853 & 3.9854 & 0.0002 \\
    30 & 3.9891 & 3.9897 & 0.0006 \\
    35 & 3.9921 & 3.9924 & 0.0003 \\
    40 & 3.9929 & 3.9941 & 0.0012 \\
    45 & 3.9938 & 3.9953 & 0.0015 \\
    50 & 3.9947 & 3.9962 & 0.0015
  \end{tabular}\end{center}

  \begin{center}\begin{tabular}{c|c|c|c}
    \hline\multicolumn{4}{c}{Tolleranza $tol=10^{-7}$}\\\hline
    $n$ & $\lambda_1$ & $2\left(1+\cos{\frac{\pi}{n+1}}\right)$ & scostamento\\\hline
    5 & 3.7321 & 3.7321 & 0.0000 \\
    10 & 3.9190 & 3.9190 & 0.0000 \\
    15 & 3.9616 & 3.9616 & 0.0000 \\
    20 & 3.9777 & 3.9777 & 0.0000 \\
    25 & 3.9854 & 3.9854 & 0.0000 \\
    30 & 3.9897 & 3.9897 & 0.0000 \\
    35 & 3.9924 & 3.9924 & 0.0000 \\
    40 & 3.9941 & 3.9941 & 0.0000 \\
    45 & 3.9953 & 3.9953 & 0.0000 \\
    50 & 3.9962 & 3.9962 & 0.0000
  \end{tabular}\end{center}
\end{sol}

\sectionline{black}{88}

\section{Esercizio 3}
\label{sub:es3}
\emph{Dimostrare i Corollari 6.2 e 6.3.}
\begin{sol}
  \normalfont
  \underline{Corollario 6.2}\\
  Se $A=\alpha(I-B)$ è una M-matrice e $A=M-N$, con $0\leq N\leq\alpha B$, allora M è nonsingolare
  e lo splitting è regolare. Pertanto, il metodo iterativo per calcolare un’approssimazione
  del vettore di pagerank è convergente.
  \\
  \underline{Dimostrazione}\\
  Sia $A=\alpha(I-B)=M-N$ con $0\leq N\leq\alpha B$ quindi $M=\alpha I-\alpha B+N=\alpha I-\alpha B+\alpha Q=\alpha(I-(B-Q))$
  con $\alpha Q=N\leq\alpha B$ e $0\leq Q\leq B$. Dato che $0\leq B-Q\leq B$, per il Lemma 6.2,
  $\rho(B-Q)\leq\rho(B)\leq 1$. Quindi $M$ è una M-matrice; lo splitting è regolare infatti
  $M^{-1}\geq 0$ ($M$ è una M-matrice) e $B-Q\geq 0$.\\
  \vspace{5mm}
  \underline{Corollario 6.3}\\
  Se $A$ è una M-matrice e la matrice $M$ in ($A=M-N$) è ottenuta ponendo a $0$
  gli elementi extradiagonali di $A$, allora lo splitting ($A=M-N$) è regolare.
  Pertanto il metodo iterativo è convergente.\\
  \underline{Dimostrazione}\\
  Sia $A=M-N$ M-matrice con $M=diag(A)$ allora la matrice $A-M=B$ avrà elementi nulli sulla diagonale e, altrove,
  gli elementi di $A$. Poiché $A=\alpha(I-B)=\alpha I-\alpha B=M-N$, risulta $\alpha B=N$ quindi,
  per il Corollario 6.2, lo splitting è regolare ed il metodo è convergente.
\end{sol}

\sectionline{black}{88}

\section{Esercizio 4}
\label{sub:es4}
\emph{Dimostrare il Teorema 6.9.}
\begin{sol}
  \underline{Teorema 6.9}\\
  Se la matrice $A$ in ($A=D-L-U$) è una M-matrice, allora $D,L,U\geq 0$.
   In particolare $D$ ha elementi diagonali positivi ($D>0$).\\
  \underline{Dimistrazione}\\
  Poiché $A$ è una M-matrice, essa può essere scritta nella forma $A=\alpha(I-B)$ con $B\geq 0$ e $\rho(B)<1$;
   inoltre, per ipotesi, $A=D-L-U$ da cui si deduce che $D=\alpha(I-diag(B))$ e $(L+U)=\alpha(B-diag(B))$.\\
  La matrice $(L+U)=\alpha(B-diag(B))$ risulta maggiore di zero in quanto $B>0\rightarrow (B-diag(B))>0$.
  La matrice $D$ ha elementi positivi sulla diagonale: supponiamo per assurdo $a_{i,i}\leq 0$:
    l'$i$-esima riga di $A$, negativa, è data da $A\underline{e}_i\leq\underline{0}$ dato che $A$ è una M-matrice ed è monotona si può scrivere
    $\underline{e}_i\leq A^{-1}\underline{0}=\underline{0}$ assurdo poiché $\underline{e}_i\geq\underline{0}, \forall i$.
\end{sol}

\sectionline{black}{88}

\section{Esercizio 5}
\label{sub:es5}
\emph{Tenendo conto della (6.10), riformulare il metodo delle potenze (6.11) per il calcolo del \textit{Google pagerank} come metodo iterativo definito da uno splitting regolare.}
\begin{sol}
  Il problema del \textsl{Google pagerank} è $S(p)\hat{\underline{x}}=\hat{\underline{x}}$ dove $S(p)=pS+(1-p)\underline{v}\:\underline{e}^{T}$. Sostituendo, risulta
   $$(pS+(1-p)\underline{v}\:\underline{e}^{T})\hat{\underline{x}}=\hat{\underline{x}}\quad\Rightarrow\quad (I-pS)\hat{\underline{x}}=(1-p)\underline{v}\:\underline{e}^T\hat{\underline{x}}$$
  Dato che $\underline{v}=\frac{1}{n}\underline{e}$ ed $e^T\hat{\underline{x}}=1$ si ricava
  $$(I-pS)\hat{\underline{x}}=\frac{1-p}{n}\underline{e}.$$
  Si può infine definire il seguente metodo iterativo:
  $$I\underline{x}_{k+1}=pS\underline{x}_k+\dfrac{1-p}{n}\underline{e}.$$
  Il metodo è convergente in quanto la matrice di iterazione ha raggio spettrale minore di $1$: $\rho(I^{-1}pS)=\rho(pS)<1$ dato che $p \in (0,1)$ e $\rho(S)=1$.
\end{sol}

\sectionline{black}{88}

\section{Esercizio 6}
\label{sub:es6}
\emph{Dimostrare che il metodo di Jacobi converge asintoticamente in un numero minore di iterazioni, rispetto al metodo delle potenze (6.11) per il calcolo del \textit{Google pagerank}.}
\begin{sol}
  La matrice di iterazione di Jacobi ha raggio spettrale minore di 1 mentre, nel calcolo del \textsl{Google pagerank},
   l'autovalore dominante è $\lambda=1$ quindi il raggio spettrale di tale matrice è esattamente 1.
   Poiché il raggio spettrale della matrice di iterazione del metodo di Jacobi è minore del raggio spettrale della matrice del
   \textsl{Google pagerank}, il metodo di Jacobi converge in asintoticamente in un numero minore di iterazioni, rispetto al metodo delle potenze per il calcolo del \textsl{Google pagerank}.
\end{sol}

\sectionline{black}{88}

\section{Esercizio 7}
\label{sub:es7}
\emph{Dimostrare che, se $A$ è diagonale dominante, per riga o per colonna, il metodo di Jacobi è convergente.}
\begin{sol}
  Nel metodo di Jacobi si ha $A=M-N=D-(L+U)$ quindi
   $$M=\begin{pmatrix}a_{1,1}&&&\\&\ddots&&\\&&\ddots&\\&&&a_{n,n}\end{pmatrix}\mbox{ e } N=\begin{pmatrix}0&a_{1,2}&\ldots&a_{1,n}\\a_{2,1}&\ddots&\ddots&\vdots\\\vdots&\ddots&\ddots&a_{n-1,n}\\a_{n,1}&\ldots&a_{n,n-1}&0\end{pmatrix}.$$
   Si ha $$B=M^{-1}N=\begin{pmatrix}\frac{1}{a_{1,1}}&&&\\&\ddots&&\\&&\ddots&\\&&&\frac{1}{a_{n,n}}\end{pmatrix}\begin{pmatrix}0&a_{1,2}&\ldots&a_{1,n}\\a_{2,1}&\ddots&\ddots&\vdots\\\vdots&\ddots&\ddots&a_{n-1,n}\\
   a_{n,1}&\ldots&a_{n,n-1}&0\end{pmatrix}=$$$$=\begin{pmatrix}0&\frac{a_{1,2}}{a_{1,1}}&\ldots&\frac{a_{1,n}}{a_{1,1}}\\\frac{a_{2,1}}{a_{2,2}}&\ddots&\ddots&\vdots\\\vdots&\ddots&\ddots&\frac{a_{n-1,n}}{a_{n-1,n-1}}\\\frac{a_{n,1}}{a_{n,n}}&\ldots&\frac{a_{n,n-1}}{a_{n,n}}&0\end{pmatrix},$$
   per il Teorema di Gershgorin risulta
   $$\mathcal{D}_i=\left\{\lambda\in\mathbb{C}\::\:|\lambda-b_{i,i}|\leq\sum_{j=1\: j\neq i}^{n}{|b_{i,j}|}\right\}=\left\{\lambda\in\mathbb{C}\::\:|\lambda|\leq\sum_{j=1\: j\neq i}^{n}{\left|\frac{a_{i,j}}{a_{i,i}}\right|}\right\};$$
   supposta $A$ a diagonale dominante per righe, $|a_{i,i}|\geq\sum_{j=1,\: j\neq i}^n{|a_{i,j}|}$,
   risulta $|\lambda|\leq\frac{1}{|a_{i,i}|}\sum_{j=1\: j\neq i}^{n}{a_{i,j}}<1$. Ogni $\mathcal{D}_i$ è centrato in $0$ e ha raggio minore di $1$ quindi $\mathcal{D}=\bigcup_{i=1}^{n}{\mathcal{D}_i}$ è centrato in $0$ e ha raggio pari al raggio massimo dei $\mathcal{D}_i$ ma sempre minore di $1$. Dato che $\lambda(A)=\lambda(A^T)$,
   il risultato vale anche se $A$ è a diagonale dominante per colonne; il metodo di Jacobi è dunque convergente per matrici a diagonale dominante.
\end{sol}

\sectionline{black}{88}

\section{Esercizio 8}
\label{sub:es8}
\emph{Dimostrare che, se $A$ è diagonale dominante, per riga o per colonna, il metodo di Gauss-Seidel è convergente.}
\begin{sol}
  Nel metodo di Gauss-Seidel si ha $A=M-N=(D-L)-U$;
   sia $\lambda\in\sigma(M^-1N)$ quindi $\lambda$ è tale che
   $\det{(M^{-1}N-\lambda I)}=\det{(M^{-1}(N-\lambda M))}=\det{(M^{-1})}\det{(N-\lambda M)}=0$.
   Dato che, per definizione di splitting, $\det{(M^{-1})}\neq 0$ deve risultare
   $\det{(N-\lambda M)}=0=\det{(\lambda M-N)}$; sia $H=\lambda M-N$ matrice singolare e supponiamo, per assurdo,
   $|\lambda|\geq 1$. Risulta $$H=\begin{cases}\lambda a_{i,j}&\mbox{se }i\geq j,\\a_{i,j}&\mbox{altrimenti},\end{cases};$$
   quindi $H$ è a diagonale dominante ma $$\sum_{j=1\: j\neq i}^n{|h_{i,j}|}\leq |\lambda|\sum_{j=1\: j\neq i}^n{|a_{i,j}|}<|\lambda||a_{i,i}|=|h_{i,i}|.$$
   Si ha una contraddizione poiché le matrici a diagonale dominanti sono non singolari,
   dunque $|\lambda|<1$ e quindi il metodo di Gauss-Seidel è convergente per matrici a diagonale dominante.
\end{sol}

\sectionline{black}{88}

\section{Esercizio 9}
\label{sub:es9}
\emph{Se $A$ è \textit{sdp}, il metodo di Gauss-Seidel risulta essere convergente.
      Dimostrare questo risultato nel caso (assai più semplice) in cui l'autovalore di massimo modulo della matrice di iterazione sia reale.\\
			(\underline{Suggerimento:} considerare il sistema lineare equivalente
			$$(D^{-\frac{1}{2}}AD^{-\frac{1}{2}})(D^{\frac{1}{2}}\underline{x})=(D^{-\frac{1}{2}}\underline{b}),\qquad D^{\frac{1}{2}}=diag(\sqrt{a_{11}},\dots,\sqrt{a_{nn}}),$$
			la cui matrice dei coefficienti è ancora \textit{sdp} ma ha diagonale unitaria,
      ovvero del tipo $I-L-L^T$. Osservare quindi che, per ogni vettore reale $\underline{v}$ di norma $1$,si ha:
      $\underline{v}^TL\underline{v}=\underline{v}^TL^T\underline{v}=\frac{1}{2}\underline{v}^T(L+L^T)\underline{v}<\frac{1}{2}$.)}
\begin{sol}
  Scriviamo il sistema $A\underline{x}=\underline{b}$ nella forma equivalente
  $$\left(D^{-1/2}AD^{-1/2}\right)\left(D^{1/2}\underline{x}\right)=\left(D^{-1/2}\underline{b}\right)$$
  con $D=diag(\sqrt{a_{1,1}},\ldots,\sqrt{a_{n,n}})$.
  La matrice $C=\left(D^{-1/2}AD^{-1/2}\right)$ ha diagonale unitaria:
  $$c_{i,i}=d_i^{-1}a_{i,i}d_i^{-1}=\frac{1}{\sqrt{a_{i,i}}}a_{i,i}\frac{1}{\sqrt{a_{i,i}}}=a_{i,i};$$
  inoltre è ancora sdp e scrivibile come $C=I-L-L^T$.\\
  Poiché $C$ è sdp risulta $\underline{v}^TA\underline{v}>0, \forall\underline{v}\neq\underline{0}$
  cioè $$\underline{v}^T\underline{v}>\underline{v}^TL\underline{v}+\underline{v}^TL^T\underline{v}\quad\Rightarrow\quad\underline{v}^TL\underline{v}=\underline{v}^TL^T\underline{v}<\frac{1}{2}.$$
  Sia $|\lambda|=\rho(M_{GS}^{-1}N_{GS})=\rho\left((I-L)^{-1}L^T\right)$
  assunto reale e $\underline{v}$ il corrispondente autovettore, dunque
  $$(I-L)^{-1}L^T=\lambda\underline{v}\quad\Rightarrow\quad\lambda\underline{v}=L^T\underline{v}+\lambda L\underline{v}$$
  ovvero $\lambda=\underline{v}^TL\underline{v}+\lambda\underline{v}^TL\underline{v}=(1+\lambda)\underline{v}^TL)\underline{v}$
  da cui $$\frac{\lambda}{1+\lambda}=\underline{v}^TL\underline{v}<\frac{1}{2}\quad\Rightarrow\quad -1<\lambda<1.$$
  Segue $\rho(M_{GS}^{-1}N_{GS})=|\lambda|<1$.
\end{sol}

\sectionline{black}{88}

\section{Esercizio 10}
\label{sub:es10}
\emph{Con riferimento ai vettori errore (6.16) e residuo (6.17) dimostrare che, se
			\begin{equation}
				\label{criterioArrestoSplitting}
				||\underline{r_k}||\leq\varepsilon||\underline{b}||,
			\end{equation}
			allora
			$$||\underline{e_k}||\leq\varepsilon k(A)||\underline{\hat{x}}||,$$
			dove $k(A)$ denota, al solito, il numero di condizionamento della matrice $A$.
      Concludere che, per sistemi lineari malcondizionati, anche la risoluzione iterativa (al pari di quella diretta)
      risulta essere più problematica.}
\begin{sol}
  Posto $\underline{e}_k=\underline{x}_k-\underline{\hat{x}}$
  e $\underline{r}_k=A\underline{x}_k-\underline{b}$, risulta
  $$A\underline{e}_k=A(\underline{x}_k-\underline{\hat{x}})=A\underline{x}_k-A\underline{\hat{x}}=A\underline{x}_k-b=\underline{r}_k.$$
  Segue, passando alle norme,
  \begin{equation}\begin{split}||\underline{e}_k||=&\left|\left|A^{-1}\underline{r}_k\right|\right|\leq\left|\left|A^{-1}\right|\right|\cdot||\underline{r}_k||\leq\left|\left|A^{-1}\right|\right|\cdot\varepsilon||\underline{b}||\leq\\\leq&\left|\left|A^{-1}\right|\right|\cdot\left|\left|A\right|\right|\cdot||\underline{\hat{x}}||=\varepsilon\kappa(A)||\underline{\hat{x}}||,\end{split}\end{equation}
  ovvero $$\frac{||\underline{e}_k||}{||\underline{\hat{x}}||}\leq\varepsilon\kappa(A).$$
  La risoluzione iterativa, come la risoluzione diretta, di sistemi lineari è ben condizionata per
  $\kappa(A)\approx 1$ mentre risulta malcondizionata per $\kappa(A)\gg 1$.
\end{sol}

\sectionline{black}{88}

\section{Esercizio 11}
\label{sub:es11}
\emph{Calcolare il polinomio caratteristico della matrice
			\[
				\begin{pmatrix}
					0 & \dots & 0 & \alpha\\
					1 & \ddots & & 0\\
					& \ddots & \ddots & \vdots\\
					0 & & 1 & 0
				\end{pmatrix}\in\mathbb{R}^{n\times n}.
			\]}
\begin{sol}
	\normalfont Il polinomio caratteristico è dato del determinante della matrice $A-\lambda I$:
	\begin{equation}
    \begin{split}\det{(A-\lambda I)}=&\det{\begin{pmatrix}-\lambda&\ldots&0&\alpha\\1&\ddots&&0\\&\ddots&\ddots&\vdots\\0&&1&-\lambda\end{pmatrix}}=\\=&(-1)^n\lambda^n+(-1)^{n+1}\alpha=(-1)^n(\lambda^n-\alpha).\end{split}
  \end{equation}
  Le radici di tale polinomio sono $\lambda=\sqrt[n]{\alpha}$.
\end{sol}

\sectionline{black}{88}

\section{Esercizio 12}
\label{sub:es12}
\emph{Dimostrare che i metodi di Jacobi e Gauss-Seidel possono essere utilizzati per la risoluzione del sistema lineare (gli elementi non indicati sono da intendersi nulli)
			\[
				\begin{pmatrix}
					1 & & & -\frac{1}{2}\\
					-1 & 1 & &\\
					& \ddots & \ddots &\\
					& & -1 & 1
				\end{pmatrix}\underline{x}=\begin{pmatrix}
					\frac{1}{2}\\
					0\\
					\vdots\\
					0
				\end{pmatrix}\in\mathbb{R}^n,
			\]
			la cui soluzione è $\underline{x}=(1,\dots,1)^T\in\mathbb{R}^n$.
      Confrontare il numero di iterazioni richieste dai due metodi per soddisfare lo stesso criterio di arresto (6.19),
      per valori crescenti di $n$ e per tolleranze $\varepsilon$ decrescenti. Riportare i risultati ottenuti in una tabella $(n/\varepsilon)$.
      }
  \begin{sol}
        \normalfont
        La matrice è una M-matrice:$$A=\begin{pmatrix}1&&&-1/2\\-1&1&&\\&\ddots&\ddots&\\&&-1&1\end{pmatrix}=I-B\mbox{ con }B=\begin{pmatrix}0&&&1/2\\1&\ddots&&\\&\ddots&\ddots&\\&&1&0\end{pmatrix}>0.$$
        Si dimostra che $\rho(B)<1$ calcolando gli autovalori della matrice $B$:
        $$\det(B-\lambda I)=$$ $$\det\begin{pmatrix}-\lambda&&&1/2\\1&\ddots&&\\&\ddots&\ddots&\\&&1&-\lambda\end{pmatrix}=-\lambda\det\begin{pmatrix}-\lambda&&&\\1&\ddots&&\\&\ddots&\ddots&\\&&1&-\lambda\end{pmatrix}_{(n-1)\times (n-1)}+$$ $$+(-1)^{n+1}\frac{1}{2}\det\begin{pmatrix}1&-\lambda &&\\&\ddots&\ddots&\\&&\ddots&-\lambda\\&&&1\end{pmatrix}_{(n-1)\times (n-1)} = -\lambda(-\lambda)^{n-1}+(-1)^{n+1}\frac{1}{2}=$$$$=(-1)^{n-2}\lambda^n+(-1)^{n+1}\frac{1}{2}=(-1)^n\frac{1}{2}(2\lambda^n-1).$$
        Quindi $\det(B-\lambda I)=0$ se e solo se $2\lambda^n-1=0$ se e solo se $\lambda=2^{-1/n}$. Poiché
        $|\lambda|<1$, $\rho(B)<1$ quindi $A$ è una M-matrice ed è possibile risolvere il sistema lineare tramite i metodo iterativi
        di Jacobi e Gauss-Seidel in quando lo splitting è regolare ed $A$ converge.\\
        Implementando il criterio d'arresto $||\underline{r}_k||\leq\varepsilon||\underline{b}||$, si ha convergenza nel numero di iterazioni riportate nelle seguenti tabelle:
        Codice:\\
        \lstinputlisting[caption={Esercizio 6.12.}, label=lst:es6.12]{code/es6_12.m}
        \vspace{0.5em}
        \footnotesize
        \begin{center}\begin{tabular}{|c||c|c|c|c|c|c|c|c|c|c|}
        \multicolumn{11}{c}{Iterazioni del metodo di Jacobi}\\
        \hline
        $n \backslash \varepsilon $ & $10^{-1}$& $10^{-2}$& $10^{-3}$& $10^{-14}$& $10^{-5}$& $10^{-6}$& $10^{-7}$& $10^{-8}$& $10^{-9}$& $10^{-10}$\\\hline
        5& 25 & 34 & 54 & 74 & 85 & 105 & 124 & 139 & 157 & 171 \\\hline
        10& 50 & 72 & 116 & 145 & 181 & 211 & 250 & 276 & 304 & 342 \\\hline
        15& 87 & 125 & 180 & 230 & 284 & 326 & 379 & 430 & 465 & 524 \\ \hline
        20& 95 & 175 & 239 & 298 & 370 & 433 & 512 & 573 & 628 & 702 \\\hline
        25& 128 & 217 & 299 & 375 & 468 & 549 & 643 & 720 & 800 & 885 \\ \hline
        30& 162 & 265 & 378 & 475 & 570 & 659 & 762 & 870 & 964 & 1066 \\ \hline
        35& 199 & 311 & 436 & 533 & 662 & 775 & 905 & 1014 & 1120 & 1249 \\ \hline
        40& 214 & 384 & 494 & 635 & 755 & 889 & 1037 & 1157 & 1299 & 1429 \\ \hline
        45& 252 & 423 & 556 & 703 & 853 & 1020 & 1165 & 1327 & 1448 & 1607 \\ \hline
        50& 299 & 458 & 622 & 787 & 963 & 1108 & 1290 & 1473 & 1630 & 1789 \\\hline
        \end{tabular}\end{center}
        \begin{center}\begin{tabular}{|c||c|c|c|c|c|c|c|c|c|c|}
        \multicolumn{11}{c}{Iterazioni del metodo di Gauss-Seidel}\\
        \hline
        $n \backslash \varepsilon $ & $10^{-1}$& $10^{-2}$& $10^{-3}$& $10^{-14}$& $10^{-5}$& $10^{-6}$& $10^{-7}$& $10^{-8}$& $10^{-9}$& $10^{-10}$\\\hline
        5 & 3 & 7 & 10 & 13 & 17 & 20 & 22 & 26 & 30 & 33 \\\hline
        10 & 1 & 6 & 10 & 13 & 16 & 20 & 23 & 26 & 27 & 33 \\\hline
        15 & 4 & 6 & 9 & 12 & 17 & 19 & 21 & 27 & 27 & 33 \\ \hline
        20 & 2 & 5 & 10 & 12 & 15 & 20 & 23 & 27 & 28 & 33 \\ \hline
        25 & 4 & 4 & 10 & 12 & 16 & 20 & 24 & 23 & 28 & 33 \\ \hline
        30 & 1 & 7 & 7 & 13 & 17 & 19 & 23 & 27 & 29 & 33 \\ \hline
        35 & 2 & 7 & 10 & 11 & 17 & 19 & 23 & 26 & 30 & 32 \\ \hline
        40 & 1 & 7 & 9 & 12 & 17 & 20 & 18 & 27 & 30 & 25 \\ \hline
        45 & 1 & 3 & 9 & 13 & 15 & 20 & 23 & 27 & 30 & 32 \\ \hline
        50 & 2 & 3 & 10 & 7 & 15 & 19 & 23 & 27 & 27 & 31 \\ \hline
        \end{tabular}\end{center}
        \normalsize Si nota come il numero di iterazioni richieste dal metodo di Gauss-Seidel sia molto minore del numero richiesto dal metodo di Jacobi, per ogni valore della tolleranza.
      \end{sol}

\chapter{Calcolo del Google Pagerank}
\label{chap:Google}

\section{Esercizio 1}
\label{sub:Es1}
[Teorema di Gershgorin]
      Dimostrare che gli autovalori di una matrice
      $A=(a_{ij})\in\mathbb{C}^{n n}$
      \emph{sono contenuti nell'insieme}
			\[
				\mathcal{D}=\bigcup_{i=1}^n\mathcal{D}_i,\qquad \mathcal{D}_i=\left\{\lambda\in\mathbb{C}:|\lambda-a_{ii}|\leq\sum_{\begin{subarray}{c}
					j=1\\
					j\neq i
				\end{subarray}}^n|a_{ij}|\right\},\quad i=1,\dots,n.
			\]
\begin{sol}
   Sia $\lambda\in\sigma{A}$ ed $\underline{x}$ il corrispondente autovettore ($\underline{x}\neq\underline{0}$
   e $A\underline{x}=\lambda\underline{x}$) quindi $\underline{e}_i^TA\underline{x}=\lambda\underline{e}_i^T\underline{x}$
   per $i=1,\ldots,n$ cioè $\sum_{j=1}^n{a_{i,j}x_j}=\lambda x_i$;\\
   posto $i$ tale che $|x_i|\geq|x_j|$ ($x_i\neq 0$)
   risulta $$\lambda=\frac{\sum_{j=1}^n{a_{i,j}x_j}}{x_i}=a_{i,i}+\sum_{j=1\: j\neq i}^n{a_{i,j}\frac{x_j}{x_i}}$$ ovvero
   $\lambda-a_{i,i}=\sum_{j=1\: j\neq i}^n{a_{i,j}\frac{x_j}{x_i}}$.\\
   Mettendo poi i valori assuluti
   $$\left|\lambda-a_{i,i}\right|=\left|\sum_{j=1\: j\neq i}^n{a_{i,j}\frac{x_j}{x_i}}\right|\leq\sum_{j=1\: j\neq i}^n{\left|a_{i,j}\frac{x_j}{x_i}\right|}\leq\sum_{j=1\: j\neq i}^n{\left|a_{i,j}\right|}$$ poiché $\left|\frac{x_j}{x_i}\right|\leq 1$
   in quanto $|x_i|\geq|x_j|$. Segue $\lambda\in\bigcup_{i=1}^{n}{\mathcal{D}_i}$ da cui $\sigma{A}\subseteq\mathcal{D}$.
\end{sol}

\sectionline{black}{88}

\section{Esercizio 2}
\label{sub:Es2}
\emph{
      Utilizzare il metodo delle potenze per approssimare l'autovalore dominante della matrice
			\[
				A_n=\begin{pmatrix}
					2 & -1 & &\\
					-1 & 2 & \ddots &\\
					& \ddots & \ddots & -1\\
					& & -1 & 2
				\end{pmatrix}\in\mathbb{R}^{n\times n},
			\]
			per valori crescenti di $n$. Verificare numericamente che questo è dato da $2\left(1+\cos\frac{\pi}{n+1}\right)$.
}
\begin{sol}
  \normalfont
  $A_n$ è una M-matrice,in quanto può essere scritta come : $A_n=2(I_n-B_n)$ dove
  $$B_n=\begin{pmatrix}0&1/2&&\\1/2&0&\ddots&\\&\ddots&\ddots&1/2\\&&1/2&0\end{pmatrix}\in\mathbb{R}^{n\times n} \quad I_n=\begin{pmatrix}2&0&&\\0&2&\ddots&\\&\ddots&\ddots&0\\&&0&2\end{pmatrix}\in\mathbb{R}^{n\times n}.$$
  Risulta $\lambda_j(B_n)=\cos{\frac{j\pi}{n+1}}$, $|\lambda_j|\leq 1$, per $j=1,\ldots,n$;
  segue $\lambda_j(A_n)=2(\lambda_j(I_n)-\lambda_j(B_n))=2(1-\cos{\frac{j\pi}{n+1}})$; il massimo è $\lambda=2(1+\cos{\frac{\pi}{n+1}})$.
  La verifica numerica di questo risultato è riportata nelle seguenti tabelle.
  Codice:\\
  \lstinputlisting[caption={Esercizio 6.2}, label=lst:es6.2]{code/es6_2.m}
  \vspace{0.5em}
  \begin{center}\begin{tabular}{c|c|c|c}
    \hline\multicolumn{4}{c}{Tolleranza $tol=10^{-5}$}\\\hline
    $n$ & $\lambda_1$ & $2\left(1+\cos{\frac{\pi}{n+1}}\right)$ & scostamento\\\hline
    10 & 3.9189 & 3.9190 & 0.0001 \\
    15 & 3.9614 & 3.9616 & 0.0002 \\
    20 & 3.9774 & 3.9777 & 0.0003 \\
    25 & 3.9853 & 3.9854 & 0.0002 \\
    30 & 3.9891 & 3.9897 & 0.0006 \\
    35 & 3.9921 & 3.9924 & 0.0003 \\
    40 & 3.9929 & 3.9941 & 0.0012 \\
    45 & 3.9938 & 3.9953 & 0.0015 \\
    50 & 3.9947 & 3.9962 & 0.0015
  \end{tabular}\end{center}

  \begin{center}\begin{tabular}{c|c|c|c}
    \hline\multicolumn{4}{c}{Tolleranza $tol=10^{-7}$}\\\hline
    $n$ & $\lambda_1$ & $2\left(1+\cos{\frac{\pi}{n+1}}\right)$ & scostamento\\\hline
    5 & 3.7321 & 3.7321 & 0.0000 \\
    10 & 3.9190 & 3.9190 & 0.0000 \\
    15 & 3.9616 & 3.9616 & 0.0000 \\
    20 & 3.9777 & 3.9777 & 0.0000 \\
    25 & 3.9854 & 3.9854 & 0.0000 \\
    30 & 3.9897 & 3.9897 & 0.0000 \\
    35 & 3.9924 & 3.9924 & 0.0000 \\
    40 & 3.9941 & 3.9941 & 0.0000 \\
    45 & 3.9953 & 3.9953 & 0.0000 \\
    50 & 3.9962 & 3.9962 & 0.0000
  \end{tabular}\end{center}
\end{sol}

\sectionline{black}{88}

\section{Esercizio 3}
\label{sub:es3}
\emph{Dimostrare i Corollari 6.2 e 6.3.}
\begin{sol}
  \normalfont
  \underline{Corollario 6.2}\\
  Se $A=\alpha(I-B)$ è una M-matrice e $A=M-N$, con $0\leq N\leq\alpha B$, allora M è nonsingolare
  e lo splitting è regolare. Pertanto, il metodo iterativo per calcolare un’approssimazione
  del vettore di pagerank è convergente.
  \\
  \underline{Dimostrazione}\\
  Sia $A=\alpha(I-B)=M-N$ con $0\leq N\leq\alpha B$ quindi $M=\alpha I-\alpha B+N=\alpha I-\alpha B+\alpha Q=\alpha(I-(B-Q))$
  con $\alpha Q=N\leq\alpha B$ e $0\leq Q\leq B$. Dato che $0\leq B-Q\leq B$, per il Lemma 6.2,
  $\rho(B-Q)\leq\rho(B)\leq 1$. Quindi $M$ è una M-matrice; lo splitting è regolare infatti
  $M^{-1}\geq 0$ ($M$ è una M-matrice) e $B-Q\geq 0$.\\
  \vspace{5mm}
  \underline{Corollario 6.3}\\
  Se $A$ è una M-matrice e la matrice $M$ in ($A=M-N$) è ottenuta ponendo a $0$
  gli elementi extradiagonali di $A$, allora lo splitting ($A=M-N$) è regolare.
  Pertanto il metodo iterativo è convergente.\\
  \underline{Dimostrazione}\\
  Sia $A=M-N$ M-matrice con $M=diag(A)$ allora la matrice $A-M=B$ avrà elementi nulli sulla diagonale e, altrove,
  gli elementi di $A$. Poiché $A=\alpha(I-B)=\alpha I-\alpha B=M-N$, risulta $\alpha B=N$ quindi,
  per il Corollario 6.2, lo splitting è regolare ed il metodo è convergente.
\end{sol}

\sectionline{black}{88}

\section{Esercizio 4}
\label{sub:es4}
\emph{Dimostrare il Teorema 6.9.}
\begin{sol}
  \underline{Teorema 6.9}\\
  Se la matrice $A$ in ($A=D-L-U$) è una M-matrice, allora $D,L,U\geq 0$.
   In particolare $D$ ha elementi diagonali positivi ($D>0$).\\
  \underline{Dimistrazione}\\
  Poiché $A$ è una M-matrice, essa può essere scritta nella forma $A=\alpha(I-B)$ con $B\geq 0$ e $\rho(B)<1$;
   inoltre, per ipotesi, $A=D-L-U$ da cui si deduce che $D=\alpha(I-diag(B))$ e $(L+U)=\alpha(B-diag(B))$.\\
  La matrice $(L+U)=\alpha(B-diag(B))$ risulta maggiore di zero in quanto $B>0\rightarrow (B-diag(B))>0$.
  La matrice $D$ ha elementi positivi sulla diagonale: supponiamo per assurdo $a_{i,i}\leq 0$:
    l'$i$-esima riga di $A$, negativa, è data da $A\underline{e}_i\leq\underline{0}$ dato che $A$ è una M-matrice ed è monotona si può scrivere
    $\underline{e}_i\leq A^{-1}\underline{0}=\underline{0}$ assurdo poiché $\underline{e}_i\geq\underline{0}, \forall i$.
\end{sol}

\sectionline{black}{88}

\section{Esercizio 5}
\label{sub:es5}
\emph{Tenendo conto della (6.10), riformulare il metodo delle potenze (6.11) per il calcolo del \textit{Google pagerank} come metodo iterativo definito da uno splitting regolare.}
\begin{sol}
  Il problema del \textsl{Google pagerank} è $S(p)\hat{\underline{x}}=\hat{\underline{x}}$ dove $S(p)=pS+(1-p)\underline{v}\:\underline{e}^{T}$. Sostituendo, risulta
   $$(pS+(1-p)\underline{v}\:\underline{e}^{T})\hat{\underline{x}}=\hat{\underline{x}}\quad\Rightarrow\quad (I-pS)\hat{\underline{x}}=(1-p)\underline{v}\:\underline{e}^T\hat{\underline{x}}$$
  Dato che $\underline{v}=\frac{1}{n}\underline{e}$ ed $e^T\hat{\underline{x}}=1$ si ricava
  $$(I-pS)\hat{\underline{x}}=\frac{1-p}{n}\underline{e}.$$
  Si può infine definire il seguente metodo iterativo:
  $$I\underline{x}_{k+1}=pS\underline{x}_k+\dfrac{1-p}{n}\underline{e}.$$
  Il metodo è convergente in quanto la matrice di iterazione ha raggio spettrale minore di $1$: $\rho(I^{-1}pS)=\rho(pS)<1$ dato che $p \in (0,1)$ e $\rho(S)=1$.
\end{sol}

\sectionline{black}{88}

\section{Esercizio 6}
\label{sub:es6}
\emph{Dimostrare che il metodo di Jacobi converge asintoticamente in un numero minore di iterazioni, rispetto al metodo delle potenze (6.11) per il calcolo del \textit{Google pagerank}.}
\begin{sol}
  La matrice di iterazione di Jacobi ha raggio spettrale minore di 1 mentre, nel calcolo del \textsl{Google pagerank},
   l'autovalore dominante è $\lambda=1$ quindi il raggio spettrale di tale matrice è esattamente 1.
   Poiché il raggio spettrale della matrice di iterazione del metodo di Jacobi è minore del raggio spettrale della matrice del
   \textsl{Google pagerank}, il metodo di Jacobi converge in asintoticamente in un numero minore di iterazioni, rispetto al metodo delle potenze per il calcolo del \textsl{Google pagerank}.
\end{sol}

\sectionline{black}{88}

\section{Esercizio 7}
\label{sub:es7}
\emph{Dimostrare che, se $A$ è diagonale dominante, per riga o per colonna, il metodo di Jacobi è convergente.}
\begin{sol}
  Nel metodo di Jacobi si ha $A=M-N=D-(L+U)$ quindi
   $$M=\begin{pmatrix}a_{1,1}&&&\\&\ddots&&\\&&\ddots&\\&&&a_{n,n}\end{pmatrix}\mbox{ e } N=\begin{pmatrix}0&a_{1,2}&\ldots&a_{1,n}\\a_{2,1}&\ddots&\ddots&\vdots\\\vdots&\ddots&\ddots&a_{n-1,n}\\a_{n,1}&\ldots&a_{n,n-1}&0\end{pmatrix}.$$
   Si ha $$B=M^{-1}N=\begin{pmatrix}\frac{1}{a_{1,1}}&&&\\&\ddots&&\\&&\ddots&\\&&&\frac{1}{a_{n,n}}\end{pmatrix}\begin{pmatrix}0&a_{1,2}&\ldots&a_{1,n}\\a_{2,1}&\ddots&\ddots&\vdots\\\vdots&\ddots&\ddots&a_{n-1,n}\\
   a_{n,1}&\ldots&a_{n,n-1}&0\end{pmatrix}=$$$$=\begin{pmatrix}0&\frac{a_{1,2}}{a_{1,1}}&\ldots&\frac{a_{1,n}}{a_{1,1}}\\\frac{a_{2,1}}{a_{2,2}}&\ddots&\ddots&\vdots\\\vdots&\ddots&\ddots&\frac{a_{n-1,n}}{a_{n-1,n-1}}\\\frac{a_{n,1}}{a_{n,n}}&\ldots&\frac{a_{n,n-1}}{a_{n,n}}&0\end{pmatrix},$$
   per il Teorema di Gershgorin risulta
   $$\mathcal{D}_i=\left\{\lambda\in\mathbb{C}\::\:|\lambda-b_{i,i}|\leq\sum_{j=1\: j\neq i}^{n}{|b_{i,j}|}\right\}=\left\{\lambda\in\mathbb{C}\::\:|\lambda|\leq\sum_{j=1\: j\neq i}^{n}{\left|\frac{a_{i,j}}{a_{i,i}}\right|}\right\};$$
   supposta $A$ a diagonale dominante per righe, $|a_{i,i}|\geq\sum_{j=1,\: j\neq i}^n{|a_{i,j}|}$,
   risulta $|\lambda|\leq\frac{1}{|a_{i,i}|}\sum_{j=1\: j\neq i}^{n}{a_{i,j}}<1$. Ogni $\mathcal{D}_i$ è centrato in $0$ e ha raggio minore di $1$ quindi $\mathcal{D}=\bigcup_{i=1}^{n}{\mathcal{D}_i}$ è centrato in $0$ e ha raggio pari al raggio massimo dei $\mathcal{D}_i$ ma sempre minore di $1$. Dato che $\lambda(A)=\lambda(A^T)$,
   il risultato vale anche se $A$ è a diagonale dominante per colonne; il metodo di Jacobi è dunque convergente per matrici a diagonale dominante.
\end{sol}

\sectionline{black}{88}

\section{Esercizio 8}
\label{sub:es8}
\emph{Dimostrare che, se $A$ è diagonale dominante, per riga o per colonna, il metodo di Gauss-Seidel è convergente.}
\begin{sol}
  Nel metodo di Gauss-Seidel si ha $A=M-N=(D-L)-U$;
   sia $\lambda\in\sigma(M^-1N)$ quindi $\lambda$ è tale che
   $\det{(M^{-1}N-\lambda I)}=\det{(M^{-1}(N-\lambda M))}=\det{(M^{-1})}\det{(N-\lambda M)}=0$.
   Dato che, per definizione di splitting, $\det{(M^{-1})}\neq 0$ deve risultare
   $\det{(N-\lambda M)}=0=\det{(\lambda M-N)}$; sia $H=\lambda M-N$ matrice singolare e supponiamo, per assurdo,
   $|\lambda|\geq 1$. Risulta $$H=\begin{cases}\lambda a_{i,j}&\mbox{se }i\geq j,\\a_{i,j}&\mbox{altrimenti},\end{cases};$$
   quindi $H$ è a diagonale dominante ma $$\sum_{j=1\: j\neq i}^n{|h_{i,j}|}\leq |\lambda|\sum_{j=1\: j\neq i}^n{|a_{i,j}|}<|\lambda||a_{i,i}|=|h_{i,i}|.$$
   Si ha una contraddizione poiché le matrici a diagonale dominanti sono non singolari,
   dunque $|\lambda|<1$ e quindi il metodo di Gauss-Seidel è convergente per matrici a diagonale dominante.
\end{sol}

\sectionline{black}{88}

\section{Esercizio 9}
\label{sub:es9}
\emph{Se $A$ è \textit{sdp}, il metodo di Gauss-Seidel risulta essere convergente.
      Dimostrare questo risultato nel caso (assai più semplice) in cui l'autovalore di massimo modulo della matrice di iterazione sia reale.\\
			(\underline{Suggerimento:} considerare il sistema lineare equivalente
			$$(D^{-\frac{1}{2}}AD^{-\frac{1}{2}})(D^{\frac{1}{2}}\underline{x})=(D^{-\frac{1}{2}}\underline{b}),\qquad D^{\frac{1}{2}}=diag(\sqrt{a_{11}},\dots,\sqrt{a_{nn}}),$$
			la cui matrice dei coefficienti è ancora \textit{sdp} ma ha diagonale unitaria,
      ovvero del tipo $I-L-L^T$. Osservare quindi che, per ogni vettore reale $\underline{v}$ di norma $1$,si ha:
      $\underline{v}^TL\underline{v}=\underline{v}^TL^T\underline{v}=\frac{1}{2}\underline{v}^T(L+L^T)\underline{v}<\frac{1}{2}$.)}
\begin{sol}
  Scriviamo il sistema $A\underline{x}=\underline{b}$ nella forma equivalente
  $$\left(D^{-1/2}AD^{-1/2}\right)\left(D^{1/2}\underline{x}\right)=\left(D^{-1/2}\underline{b}\right)$$
  con $D=diag(\sqrt{a_{1,1}},\ldots,\sqrt{a_{n,n}})$.
  La matrice $C=\left(D^{-1/2}AD^{-1/2}\right)$ ha diagonale unitaria:
  $$c_{i,i}=d_i^{-1}a_{i,i}d_i^{-1}=\frac{1}{\sqrt{a_{i,i}}}a_{i,i}\frac{1}{\sqrt{a_{i,i}}}=a_{i,i};$$
  inoltre è ancora sdp e scrivibile come $C=I-L-L^T$.\\
  Poiché $C$ è sdp risulta $\underline{v}^TA\underline{v}>0, \forall\underline{v}\neq\underline{0}$
  cioè $$\underline{v}^T\underline{v}>\underline{v}^TL\underline{v}+\underline{v}^TL^T\underline{v}\quad\Rightarrow\quad\underline{v}^TL\underline{v}=\underline{v}^TL^T\underline{v}<\frac{1}{2}.$$
  Sia $|\lambda|=\rho(M_{GS}^{-1}N_{GS})=\rho\left((I-L)^{-1}L^T\right)$
  assunto reale e $\underline{v}$ il corrispondente autovettore, dunque
  $$(I-L)^{-1}L^T=\lambda\underline{v}\quad\Rightarrow\quad\lambda\underline{v}=L^T\underline{v}+\lambda L\underline{v}$$
  ovvero $\lambda=\underline{v}^TL\underline{v}+\lambda\underline{v}^TL\underline{v}=(1+\lambda)\underline{v}^TL)\underline{v}$
  da cui $$\frac{\lambda}{1+\lambda}=\underline{v}^TL\underline{v}<\frac{1}{2}\quad\Rightarrow\quad -1<\lambda<1.$$
  Segue $\rho(M_{GS}^{-1}N_{GS})=|\lambda|<1$.
\end{sol}

\sectionline{black}{88}

\section{Esercizio 10}
\label{sub:es10}
\emph{Con riferimento ai vettori errore (6.16) e residuo (6.17) dimostrare che, se
			\begin{equation}
				\label{criterioArrestoSplitting}
				||\underline{r_k}||\leq\varepsilon||\underline{b}||,
			\end{equation}
			allora
			$$||\underline{e_k}||\leq\varepsilon k(A)||\underline{\hat{x}}||,$$
			dove $k(A)$ denota, al solito, il numero di condizionamento della matrice $A$.
      Concludere che, per sistemi lineari malcondizionati, anche la risoluzione iterativa (al pari di quella diretta)
      risulta essere più problematica.}
\begin{sol}
  Posto $\underline{e}_k=\underline{x}_k-\underline{\hat{x}}$
  e $\underline{r}_k=A\underline{x}_k-\underline{b}$, risulta
  $$A\underline{e}_k=A(\underline{x}_k-\underline{\hat{x}})=A\underline{x}_k-A\underline{\hat{x}}=A\underline{x}_k-b=\underline{r}_k.$$
  Segue, passando alle norme,
  \begin{equation}\begin{split}||\underline{e}_k||=&\left|\left|A^{-1}\underline{r}_k\right|\right|\leq\left|\left|A^{-1}\right|\right|\cdot||\underline{r}_k||\leq\left|\left|A^{-1}\right|\right|\cdot\varepsilon||\underline{b}||\leq\\\leq&\left|\left|A^{-1}\right|\right|\cdot\left|\left|A\right|\right|\cdot||\underline{\hat{x}}||=\varepsilon\kappa(A)||\underline{\hat{x}}||,\end{split}\end{equation}
  ovvero $$\frac{||\underline{e}_k||}{||\underline{\hat{x}}||}\leq\varepsilon\kappa(A).$$
  La risoluzione iterativa, come la risoluzione diretta, di sistemi lineari è ben condizionata per
  $\kappa(A)\approx 1$ mentre risulta malcondizionata per $\kappa(A)\gg 1$.
\end{sol}

\sectionline{black}{88}

\section{Esercizio 11}
\label{sub:es11}
\emph{Calcolare il polinomio caratteristico della matrice
			\[
				\begin{pmatrix}
					0 & \dots & 0 & \alpha\\
					1 & \ddots & & 0\\
					& \ddots & \ddots & \vdots\\
					0 & & 1 & 0
				\end{pmatrix}\in\mathbb{R}^{n\times n}.
			\]}
\begin{sol}
	\normalfont Il polinomio caratteristico è dato del determinante della matrice $A-\lambda I$:
	\begin{equation}
    \begin{split}\det{(A-\lambda I)}=&\det{\begin{pmatrix}-\lambda&\ldots&0&\alpha\\1&\ddots&&0\\&\ddots&\ddots&\vdots\\0&&1&-\lambda\end{pmatrix}}=\\=&(-1)^n\lambda^n+(-1)^{n+1}\alpha=(-1)^n(\lambda^n-\alpha).\end{split}
  \end{equation}
  Le radici di tale polinomio sono $\lambda=\sqrt[n]{\alpha}$.
\end{sol}

\sectionline{black}{88}

\section{Esercizio 12}
\label{sub:es12}
\emph{Dimostrare che i metodi di Jacobi e Gauss-Seidel possono essere utilizzati per la risoluzione del sistema lineare (gli elementi non indicati sono da intendersi nulli)
			\[
				\begin{pmatrix}
					1 & & & -\frac{1}{2}\\
					-1 & 1 & &\\
					& \ddots & \ddots &\\
					& & -1 & 1
				\end{pmatrix}\underline{x}=\begin{pmatrix}
					\frac{1}{2}\\
					0\\
					\vdots\\
					0
				\end{pmatrix}\in\mathbb{R}^n,
			\]
			la cui soluzione è $\underline{x}=(1,\dots,1)^T\in\mathbb{R}^n$.
      Confrontare il numero di iterazioni richieste dai due metodi per soddisfare lo stesso criterio di arresto (6.19),
      per valori crescenti di $n$ e per tolleranze $\varepsilon$ decrescenti. Riportare i risultati ottenuti in una tabella $(n/\varepsilon)$.
      }
  \begin{sol}
        \normalfont
        La matrice è una M-matrice:$$A=\begin{pmatrix}1&&&-1/2\\-1&1&&\\&\ddots&\ddots&\\&&-1&1\end{pmatrix}=I-B\mbox{ con }B=\begin{pmatrix}0&&&1/2\\1&\ddots&&\\&\ddots&\ddots&\\&&1&0\end{pmatrix}>0.$$
        Si dimostra che $\rho(B)<1$ calcolando gli autovalori della matrice $B$:
        $$\det(B-\lambda I)=$$ $$\det\begin{pmatrix}-\lambda&&&1/2\\1&\ddots&&\\&\ddots&\ddots&\\&&1&-\lambda\end{pmatrix}=-\lambda\det\begin{pmatrix}-\lambda&&&\\1&\ddots&&\\&\ddots&\ddots&\\&&1&-\lambda\end{pmatrix}_{(n-1)\times (n-1)}+$$ $$+(-1)^{n+1}\frac{1}{2}\det\begin{pmatrix}1&-\lambda &&\\&\ddots&\ddots&\\&&\ddots&-\lambda\\&&&1\end{pmatrix}_{(n-1)\times (n-1)} = -\lambda(-\lambda)^{n-1}+(-1)^{n+1}\frac{1}{2}=$$$$=(-1)^{n-2}\lambda^n+(-1)^{n+1}\frac{1}{2}=(-1)^n\frac{1}{2}(2\lambda^n-1).$$
        Quindi $\det(B-\lambda I)=0$ se e solo se $2\lambda^n-1=0$ se e solo se $\lambda=2^{-1/n}$. Poiché
        $|\lambda|<1$, $\rho(B)<1$ quindi $A$ è una M-matrice ed è possibile risolvere il sistema lineare tramite i metodo iterativi
        di Jacobi e Gauss-Seidel in quando lo splitting è regolare ed $A$ converge.\\
        Implementando il criterio d'arresto $||\underline{r}_k||\leq\varepsilon||\underline{b}||$, si ha convergenza nel numero di iterazioni riportate nelle seguenti tabelle:
        Codice:\\
        \lstinputlisting[caption={Esercizio 6.12.}, label=lst:es6.12]{code/es6_12.m}
        \vspace{0.5em}
        \footnotesize
        \begin{center}\begin{tabular}{|c||c|c|c|c|c|c|c|c|c|c|}
        \multicolumn{11}{c}{Iterazioni del metodo di Jacobi}\\
        \hline
        $n \backslash \varepsilon $ & $10^{-1}$& $10^{-2}$& $10^{-3}$& $10^{-14}$& $10^{-5}$& $10^{-6}$& $10^{-7}$& $10^{-8}$& $10^{-9}$& $10^{-10}$\\\hline
        5& 25 & 34 & 54 & 74 & 85 & 105 & 124 & 139 & 157 & 171 \\\hline
        10& 50 & 72 & 116 & 145 & 181 & 211 & 250 & 276 & 304 & 342 \\\hline
        15& 87 & 125 & 180 & 230 & 284 & 326 & 379 & 430 & 465 & 524 \\ \hline
        20& 95 & 175 & 239 & 298 & 370 & 433 & 512 & 573 & 628 & 702 \\\hline
        25& 128 & 217 & 299 & 375 & 468 & 549 & 643 & 720 & 800 & 885 \\ \hline
        30& 162 & 265 & 378 & 475 & 570 & 659 & 762 & 870 & 964 & 1066 \\ \hline
        35& 199 & 311 & 436 & 533 & 662 & 775 & 905 & 1014 & 1120 & 1249 \\ \hline
        40& 214 & 384 & 494 & 635 & 755 & 889 & 1037 & 1157 & 1299 & 1429 \\ \hline
        45& 252 & 423 & 556 & 703 & 853 & 1020 & 1165 & 1327 & 1448 & 1607 \\ \hline
        50& 299 & 458 & 622 & 787 & 963 & 1108 & 1290 & 1473 & 1630 & 1789 \\\hline
        \end{tabular}\end{center}
        \begin{center}\begin{tabular}{|c||c|c|c|c|c|c|c|c|c|c|}
        \multicolumn{11}{c}{Iterazioni del metodo di Gauss-Seidel}\\
        \hline
        $n \backslash \varepsilon $ & $10^{-1}$& $10^{-2}$& $10^{-3}$& $10^{-14}$& $10^{-5}$& $10^{-6}$& $10^{-7}$& $10^{-8}$& $10^{-9}$& $10^{-10}$\\\hline
        5 & 3 & 7 & 10 & 13 & 17 & 20 & 22 & 26 & 30 & 33 \\\hline
        10 & 1 & 6 & 10 & 13 & 16 & 20 & 23 & 26 & 27 & 33 \\\hline
        15 & 4 & 6 & 9 & 12 & 17 & 19 & 21 & 27 & 27 & 33 \\ \hline
        20 & 2 & 5 & 10 & 12 & 15 & 20 & 23 & 27 & 28 & 33 \\ \hline
        25 & 4 & 4 & 10 & 12 & 16 & 20 & 24 & 23 & 28 & 33 \\ \hline
        30 & 1 & 7 & 7 & 13 & 17 & 19 & 23 & 27 & 29 & 33 \\ \hline
        35 & 2 & 7 & 10 & 11 & 17 & 19 & 23 & 26 & 30 & 32 \\ \hline
        40 & 1 & 7 & 9 & 12 & 17 & 20 & 18 & 27 & 30 & 25 \\ \hline
        45 & 1 & 3 & 9 & 13 & 15 & 20 & 23 & 27 & 30 & 32 \\ \hline
        50 & 2 & 3 & 10 & 7 & 15 & 19 & 23 & 27 & 27 & 31 \\ \hline
        \end{tabular}\end{center}
        \normalsize Si nota come il numero di iterazioni richieste dal metodo di Gauss-Seidel sia molto minore del numero richiesto dal metodo di Jacobi, per ogni valore della tolleranza.
      \end{sol}

\end{document}
