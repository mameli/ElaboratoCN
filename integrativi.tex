\chapter{Esercizi integrativi}
\qquad
\textbf{Esercizio 1}:\\Determinare un'approssimazione del secondo ordine
di $f'(x_0)$ utilizzando il passo di discretizzazione $h$ e i seguenti tre valori
di funzione: $f(x_0),f(x_0+h),f(x_0+2h)$ (molecola a tre punti in avanti).\\

\textbf{Soluzione:} \begin{equation*} \begin{split}
f(x_0 + h) &= f(x_0) + f'(x_0)h + \frac{1}{2}f''(x_0)h^2 + o(h^3)\\
f(x_0 + 2h) &= f(x_0)+f'(x_0)2h + \frac{1}{2}f''(x_0)(2h)^2 + o(h^3) \end{split}
\end{equation*} $f(x_0 + 2h) + f(x_0+h) - 2f(x_0) = 3f'(x_0)h +
\frac{3}{2}f''(x_0)h^2 + o(h^3)$ \begin{equation*} \begin{split} f'(x_0) &=
\frac{f(x_0 + 2h)+f(x_0+h)-2f(x_0)-\frac{3}{2}f''(x_0)h^2 + o(h^3)}{3h} =\\ &=
\frac{f(x_0 +2h)+f(x_0+h)-2f(x_0)}{3h}-\frac{1}{2}f''(x_0)h + o(h^2) \end{split}
\end{equation*}

\begin{center}
\line(1,0){250}
\end{center}

\textbf{Esercizio 2}:\\Dimostrare che se un numero reale $x$ viene approssimato
da $\tilde{x}$ con un certo errore relativo $\varepsilon_x$, la quantit\'a
$-log_{10}{|\varepsilon_x|}$ fornisce approssimativamente il numero di cifre
decimali esatte di $\tilde{x}$.\\

\textbf{Soluzione:}\\Per dimostrare l'asserto consideriamo: 
$$x=m\cdot10^{\eta},\qquad \tilde{x}=\tilde{m}\cdot10^{\eta}, \qquad con\:\:\:
1\leq|m|<10$$
Supponendo di avere $r$ cifre esatte, sar\'a:
$$10^{-r}\leq|\tilde{m}-m|\leq10^{-r+1}$$
$$10^{-r-1}\leq|\varepsilon_x|=\frac{|\tilde{m}-m|}{|m|}\leq\frac{10^{-r+1}}{1}$$
$$10^{-r-1}\leq|\varepsilon_x|\leq10^{-r+1}$$
Passando ai logaritmi:
$$-r-1\leq\log_{10}{|\varepsilon_x|}\leq-r+1$$
$$-1-\log_{10}{|\varepsilon_x|}\leq r\leq
1-\log_{10}{|\varepsilon_x|}$$

\begin{center}
\line(1,0){250}
\end{center}

\textbf{Esercizio 3}:\\Calcolare il pi\'u grande ed il pi\'u piccolo numero
reale di macchina positivo normalizzato che si pu\'o rappresentare utilizzando
lo standard IEEE 754 nel formato della singola precisione ed in quello della
doppia precisione.\\

\textbf{Soluzione:}\\Il pi\'u grande numero reale rappresentabile nello
standard IEEE 754 \'e dato dalla formula:$$(1-b^{-m})b^\varphi,\qquad \varphi =
b^s - \nu - 1.$$ Il $-1$ \'e dovuto al fatto che lo standard IEEE 754 riserva il
pi\'u grande esponente rappresentabile al simbolo $+\infty$.

Il pi\'u piccolo reale invece \'e dato dalla formula:$$b^{-\nu + 1}$$
Sostituendo i valori nelle formule per la singola precisione,
cio\'e $m = 24 , b = 2 , s = 8$ e $\nu = 127$ otteniamo:
$$(1-2^{-24})2^{(2^{8}-127-1)}$$ che d\'a come risultato
$\approx3.4028\cdot10^{38}$ per il massimo, e $$2^{-127 + 1}$$ che  d\'a come
risultato $\approx1.1755\cdot10^{-38}$ per il minimo.

Per la doppia precisione, invece, i valori sono: $b = 2 , m = 53 , s = 11$ e
$\nu = 1023$ con i quali otteniamo:$$(1-2^{-53})2^{(2^{11}-1023-1)}$$ che d\'a
come risultato $\approx1.7976\cdot10^{308}$ per il massimo, e $$2^{-1023 + 1}$$
che d\'a come risultato $\approx2.225\cdot10^{-308}$ per il minimo.

\begin{center}
\line(1,0){250}
\end{center}

\textbf{Esercizio 4}:\\Siano $x=2.7352\cdot10^2$, $y=4.8017\cdot10^{-2}$ e
$z=3.6152\cdot10^{-2}$. Utilizzando un'aritmetica finita che lavora in base 10
con arrotondamento e che riserva $m=4$ cifre alla mantissa, confrontare gli
errori assoluti $R_1-R$ e $R_2-R$, dove $R=x+y+z$ e 
$$R_1=(x\oplus{y})\oplus{z}, \qquad R_2=x\oplus{(y\oplus{z})}$$ 

\textbf{Soluzione:}\\$R = 2.7352 \cdot 10^2 + 4.8017 \cdot 10^{-2} +
3.6152 \cdot 10^{-2} = 2.73604169 \cdot 10^{2}$

$R_1 = fl(fl(x)+fl(y))+fl(z)
= fl(2.735 \cdot 10^{2}) + 0 = 2.735 \cdot 10^{2}$
$R_2 =fl(x)+fl(fl(y)+fl(z)) = 2.735 \cdot 10^2+fl(4.802 \cdot 10^{-2}+3.615
\cdot 10^{-2}) =$ 

$= 2.735\cdot10^2+8.417 \cdot 10^{-2} = 2.735 \cdot 10^2 + 0.001 \cdot 10^2 = 2.736 \cdot 10^2$

$$R_1 - R = -0.10417, \qquad\qquad\qquad R_2 - R = -0.00417$$

\begin{center}
\line(1,0){250}
\end{center}

\textbf{Esercizio 5}:\\Un'aritmetica finita utilizza la base 10,
l'arrotondamento, $m = 5$ cifre per la mantissa, $s = 2$ cifre per l'esponente e
lo shift $\nu = 50$. Per gli interi esso utilizza $N = 7$ cifre decimali. Dire
se il numero intero $x = 136726$ \'e un numero intero di macchina e come viene
convertito in reale di macchina. Dire quindi se il numero intero $x =
78345\cdot10^{40}$ \'e un reale di macchina e/o se \'e un intero di macchina.\\

\textbf{Soluzione:}\\Il numero 136726 \'e un intero di macchina in quanto
costituito da 6 cifre, ma non \'e rappresentabile come reale di macchina in
quanto si perde l'ultima cifra e la penultima diventa 3. Il reale con cui si
approssima il numero \'e 1.3673 per la mantissa e 55 per l'esponente.\\
Poich\'e il pi\'u grande intero rappresentabile \'e 9999999 allora il numero
$78345\cdot10^{40}$ non \'e sicuramente un intero di macchina, mentre \'e un
reale di macchina in quanto rappresentabile come un numero con mantissa 7.8345
ed esponente 94.

\begin{center}
\line(1,0){250}
\end{center}

\newpage\textbf{Esercizio 6}:\\Dimostrare che il numero di condizionamento del
problema del calcolo di\\ $\sqrt[n]{x}$ \'e $\frac{1}{n}$.\\

\textbf{Soluzione:}\\ Usando l'equazione 1.21 del libro di testo si ha
\begin{equation*}
\tilde{y} = f(\tilde{x}), \qquad \tilde{x} = x(1+\varepsilon_x), \qquad
\tilde{y} = y(1+\varepsilon_y)
\end{equation*} 
Sviluppando il secondo membro della 1.21 otteniamo
\begin{equation*}
y+y\varepsilon_y = f(x) + f'(x)x\varepsilon_x + O(\varepsilon_{x}^2)
\end{equation*} 
\begin{equation*}
|\varepsilon_y| \approx \left|f'(x)\frac{x}{y}\right|\cdot|\varepsilon_x|
\equiv \kappa|\varepsilon_x|
\end{equation*} 
Sostituendo la funzione data si ricava
\begin{equation*}
|\varepsilon_y| \approx
\left|\frac{\frac{1}{n\sqrt[n]{x^{n-1}}}}{\sqrt[n]{x}}x\right|\cdot|\varepsilon_x|
= \frac{1}{n}|\varepsilon_x|\text{ quindi } 
\kappa = \frac{1}{n}\end{equation*}

\begin{center}
\line(1,0){250}
\end{center}

\textbf{Esercizio 7}:\\Individuare una forma algebrica equivalente ma
preferibile in aritmetica finita per il controllo dell'espressione
$(x+2)^3-x^3$.\\

\textbf{Soluzione:}\\Notiamo che nella formula $(x+2)^3-x^3$, per valori di $x$
molto grandi, si incorre nel problema della cancellazione numerica, dovuto
all'operazione di sottrazione.

Dunque, una forma algebrica equivalente che risolve il problema della
cancellazione numerica, si ottiene sviluppando l'espressione come differenza di
cubi: $$(x+2)^3-x^3 = (x+2-x)[(x+2)^2+(x+2)(x)+x^2] = 2(3x^2+6x +4)$$

\begin{center}
\line(1,0){250}
\end{center}

\textbf{Esercizio 8}:\\Si calcoli l'approssimazione $\tilde{y}$ della
differenza $y$ fra $x_2 = 3.5555$ e $x_1 = 3.5554$ utilizzando un'aritmetica finita che
lavora con arrotondamento in base 10 con 4 cifre per la mantissa normalizzata.
Se ne calcoli quindi il corrispondente errore relativo e la maggiorazione di
esso che si ottiene utilizzando il numero di condizionamento della somma
algebrica.\\

\textbf{Soluzione:}\\Consideriamo le approssimazioni di $x_1$ e $x_2$:
$$\tilde{x}_1 = 3.555,\qquad\tilde{x}_2 = 3.556$$
quindi l'approssimazione della differenza $y = x_1-x_2$ \'e:
$$\tilde{y} = 3.555-3.556 = 0.001$$ 
Gli errori relativi di $x_1$ e $x_2$ sono:
$$\varepsilon_{x_1} = -\frac{3.555 - 3.5554}{3.5554} \approx -1.1\cdot10^{-4}$$
$$\varepsilon_{x_2} = \frac{3.556 - 3.5555}{3.5555} \approx 1.4\cdot10^{-4}$$
L'errore relativo di $y$ \'e:
$$\varepsilon_y = \frac{0.001 - 0.0001}{0.0001} = 9$$
e una sua maggiorazione \'e:
$$|\varepsilon_y| \leq \frac{|x_1|+|x_2|}{|-x_1 + x_2|}\varepsilon_x,\qquad 
\varepsilon_x = max\{|\varepsilon_{x_1}|,|\varepsilon_{x_2}|\} =
1.4\cdot10^{-4}$$ $$|\varepsilon_y| \leq \frac{3.5555 + 3.5554}{0.0001}1.4\cdot10^{-4} =
9.95526$$

\begin{center}
\line(1,0){250}
\end{center}

\textbf{Esercizio 9}:\\Dimostrare che il numero razionale $0.1$ (espresso in
base 10) non pu\'o essere un numero di macchina in un'aritmetica finita che
utilizza la base 2 indipendentemente da come viene fissato il numero $m$ di bit
riservati alla mantissa. Dare una maggiorazione del corrispondente errore
relativo di rappresentazione supponendo di utilizzare l'aritmetica finita
binaria che utilizza l'arrotondamento e assume $m = 7$.\\
 
\textbf{Soluzione:}\\Utilizzando il metodo di trasformazione (
ricordiamo che per convertire la parte frazionaria di un numero decimale in
binario \'e sufficiente:
\begin{enumerate}
\item moltiplicare la parte frazionaria per 2;
\item scrivere il valore della parte intera ottenuto nella posizione pi\'u
significativa;
\item ripetere il procedimento sulla parte frazionaria ottenuta.)
\end{enumerate}
otteniamo:\\
$$\begin{tabular}{l|rl}
0.1 & 0 \\
0.2 & 0 \\
0.4 & 0 \\
0.8 & 1 \\
0.6 & 1 \\
0.2 & 0 \\
0.4 & 0 \\
\ldots & \ldots \\
\end{tabular}$$\\
Si ricava quindi che:
$$0.1\:_{10} = 0.0\overline{0011}\:_2$$
Il numero ottenuto \'e periodico e quindi non rappresentabile in macchina
qualunque sia il numero $m$ di cifre riservate alla mantissa.\\
Per il teorema 1.4 del libro di testo si ha
\begin{equation*}
|\varepsilon_x| \leq u, \qquad u = \frac{1}{2}b^{1-m}|\qquad\varepsilon_x| \leq
\frac{1}{2}2^{1-7} = 2^{-7}
\end{equation*}

\begin{center}
\line(1,0){250}
\end{center}